\documentclass[11pt,a4paper,twoside,addpoints,color,answers]{exam}
\usepackage[utf8]{inputenc}
\usepackage[T1]{fontenc}

%% -> um Ligaturen beim Kopieren aus dem PDF aufzulösen
%
%\usepackage{libertine}
%\input{glyphtounicode}
%
%\pdfglyphtounicode{f_f}{FB00}
%\pdfglyphtounicode{f_f_i}{FB03}
%\pdfglyphtounicode{f_f_l}{FB04}
%\pdfglyphtounicode{f_i}{FB01}
%
%\pdfgentounicode=1
%
%% <- um Ligaturen beim Kopieren aus dem PDF aufzulösen

\usepackage[ngerman]{babel}
\usepackage{paralist}
\usepackage{extarrows}
\usepackage{graphicx}
\usepackage{wrapfig} % Produces figures which text can flow around !!! funktioniert nicht mit exam !!!
\usepackage[export]{adjustbox}
\usepackage{framed}
\usepackage{tabularx}
	\newcolumntype{Y}{>{\centering\arraybackslash}X}
\usepackage{qrcode}%QR-Codes generieren
\usepackage{multicol}%mehrspaltige Listen
\usepackage{xcolor}
\definecolor{shadecolor}{rgb}{0.9,0.9,0.9}
\usepackage{pifont} %Access to PostScript standard Symbol and Dingbats fonts
\usepackage{blindtext}
\usepackage{caption}
\usepackage[verbose]{linegoal}
\usepackage{subcaption}
\captionsetup[subfigure]{font=footnotesize,
	labelfont=up,
	%%singlelinecheck=false,
	%%format=hang,
	justification=centering % <-- new
}

\usepackage{float}
\usepackage{upgreek} % griechische Buchstaben aufrech z. B. für Präfixe
\usepackage{url}							% Verbatim with URL-sensitive line breaks
\usepackage{accents}
\usepackage{relsize} % für {\larger\textcircled{\smaller[2]1}}

\newcommand*{\zb}{z.\,B. }
\newcommand{\bx}[1]{\fbox{\rule[-0.5ex]{0ex}{3ex} \hspace*{#1ex}}}
\newcommand{\bluebox}[1]{\textcolor{blue}{\fbox{\rule[-1ex]{0ex}{3.5ex} \quad #1 \quad}}}

\newcommand{\grid}[1]{
	%		\newcount\G
	%		\G=#1
	%		\divide\G by 2
	%		\advance\G by 
	
	\begin{tikzpicture}[x=1cm,y=.5cm]
	\tikzset{dashdot/.style={dash pattern=on 0.1mm off 0.9mm on 0.1mm off 0.9mm}}
	\draw[%style=help lines ,upright
	dashdot
	%densely dotted
	%loosely dotted
	] (0,0) grid[step=0.5cm] (16,#1);
	\end{tikzpicture}
}

\newcommand{\gridn}[1]{
	%		\newcount\G
	%		\G=#1
	%		\divide\G by 2
	%		\advance\G by 
	
	\begin{tikzpicture}[x=1cm,y=.5cm]
	\tikzset{dashdot/.style={dash pattern=on 0.1mm off 0.9mm on 0.1mm off 0.9mm}}
	\draw[%style=help lines ,
	dashdot
	%densely dotted
	%loosely dotted
	] (0,0) grid[step=0.5cm] (16,#1);
	\end{tikzpicture}
}

\newcommand{\gridxy}[2]{
	
	\begin{tikzpicture}[x=.5cm,y=.5cm]
	\tikzset{dashdot/.style={dash pattern=on 0.1mm off 0.9mm on 0.1mm off 0.9mm}}
	\draw[%style=help lines ,
	dashdot
	%densely dotted
	%loosely dotted
	] (0,0) grid[step=0.5cm] (#1,#2);
	\end{tikzpicture}
}

\usepackage[most]{tcolorbox}
%package provides commands to produce colorful framed boxes which can also be applied to math environments.
\newtcolorbox{mymbox}[1]{
	%enhanced,
	ams align, 
	colback=yellow!5!white, 
	colframe=red!75!black, 
	fonttitle=\bfseries, 
	drop fuzzy shadow=blue!50!black!50!white,
	title=#1}
\newtcolorbox{mymbox*}[1]{
	%enhanced,
	ams align*, 
	colback=yellow!5!white, 
	colframe=red!75!black, 
	fonttitle=\bfseries, 
	drop fuzzy shadow=blue!50!black!50!white,
	title=#1}
\newtcolorbox{mybox}[1]{colback=red!5!white, colframe=red!75!black, fonttitle=\bfseries, title=#1}
\newtcolorbox{myboxyellow}[1]{colback=yellow!5!white, colframe=red!75!black, fonttitle=\bfseries, title=#1}
\newtcolorbox{myboxblue}[1]{colback=blue!5!white, colframe=blue!75!black, fonttitle=\bfseries, title=#1}

\usepackage{hyperref}						% Extensive support for hypertext in LATEX
\hypersetup{
	colorlinks	= true,
	linkcolor 	= black,
	urlcolor	= blue,
	breaklinks = true		
}
\usepackage[hyphenbreaks]{breakurl}			% Line-breakable \url-like links in hyperref when compiling via dvips/ps2pdf

\newcommand{\hervorheben}[1]{\textsl{#1}}

\usepackage{enumitem}% http://ctan.org/pkg/enumitem zur Anpassung von itemize und enumerate-Umgebungen [leftmargin=5mm,noitemsep,topsep=5pt,parsep=5pt,partopsep=0pt]

\newcommand{\mystrut}{\rule[-3ex]{0ex}{8ex}} % für Tabellen mit tabularx

%\renewcommand*{\dh}{d.~h.}

\usepackage{tabto} %https://tex.stackexchange.com/questions/14038/tabbing-inside-itemize-or-itemize-inside-tabbing

\newcommand*{\linie}{
	
	\vspace*{1ex}
	
	\hrulefill}

% Erweiterung zu \phantom{}, so dass auch negative Abstände "eingefügt" werden können:

\makeatletter
\newlength{\negph@wd}
\DeclareRobustCommand{\negphantom}[1]{%
	\ifmmode
	\mathpalette\negph@math{#1}%
	\else
	\negph@do{#1}%
	\fi
}
\newcommand{\negph@math}[2]{\negph@do{$\m@th#1#2$}}
\newcommand{\negph@do}[1]{%
	\settowidth{\negph@wd}{#1}%
	\hspace*{-\negph@wd}%
}
\makeatother

%\usepackage{stackengine,graphicx}
%\def\stacktype{L}
%\def\useanchorwidth{T}
%\newcommand\cancel[1]{\stackon[0pt]{#1}{\rotatebox{30}{\textcolor{red}{\rule{2.9ex}{1pt}}}}}

\usepackage{cancel}
\renewcommand{\CancelColor}{\color{red}}

\usepackage{ulem}			% Package for underlining

\usepackage{emptypage}

%\usepackage{xparse}
%
%\DeclareDocumentCommand{\hcancel}{mO{0pt}O{0pt}O{0pt}O{0pt}}{%
%	\tikz[baseline=(tocancel.base)]{
%		\node[inner sep=0pt,outer sep=0pt] (tocancel) {#1};
%		\draw[red] ($(tocancel.south west)+(#2,#3)$) -- ($(tocancel.north east)+(#4,#5)$);
%	}%
%}%
%

% zum schrägen Durchstreichen:

\usepackage{keycommand}
% Patch by Joseph Wright ("bug in the definition of \ifcommandkey (2010/04/27 v3.1415)"),
% https://tex.stackexchange.com/a/35794
\begingroup
\makeatletter
\catcode`\/=8 %
\@firstofone
{
	\endgroup
	\renewcommand{\ifcommandkey}[1]{%
		\csname @\expandafter \expandafter \expandafter
		\expandafter \expandafter \expandafter \expandafter
		\kcmd@nbk \commandkey {#1}//{first}{second}//oftwo\endcsname
	}
}
%--------%
\usepackage{tikz}
\usetikzlibrary{calc}
\newkeycommand{\hcancel}[hshiftstart=0pt,vshiftstart=0pt,hshiftend=0pt,vshiftend=0pt,color=red][1]{%
	\tikz[baseline=(tocancel.base)]{
		\node[inner sep=0pt,outer sep=0pt] (tocancel) {#1};
		\draw[\commandkey{color}] ($(tocancel.south west)+(\commandkey{hshiftstart},\commandkey{vshiftstart})$) --
		($(tocancel.north east)+(\commandkey{hshiftend},\commandkey{vshiftend})$);
	}%
}%


% farbig unterstreichen:

\newsavebox\MBox
\newcommand\Cline[2][red]{{\sbox\MBox{$#2$}%
		\rlap{\usebox\MBox}\color{#1}\rule[-1.2\dp\MBox]{\wd\MBox}{0.5pt}}}
	

\def\mathunderline#1#2{\color{#1}\underline{{#2}}}


% farbig hervorheben in math

\newcommand{\highlight}[2][red!50]{\mathpalette{\highlightwithstyle[#1]}{#2}}
\newcommand{\highlightwithstyle}[3][red!50]{
	\begingroup                         %% <- limit scope of \box0 and \fboxsep assignment
	\sbox0{$\mathsurround 0pt #2#3$}% %% <- typeset content in box 0
	\setlength{\fboxsep}{1pt}        %% <- set (smaller) framebox margins
	\sbox2{\hspace{-.5pt}%            %% <- create box 2, undo margin
		\colorbox{#1}{\usebox0}%        %% <- print the contents of box 0 in a \colorbox
	}%
	\dp2=\dp0 \ht2=\ht0 \wd2=\wd0     %% <- set dimensions of box 2 to match box 0
	\box2                             %% <- print box 2
	\endgroup                           %% <- revert old definitions of the boxes and \fboxsep
}

% für Tabellen
\usepackage{array,booktabs,tabularx}
\usepackage{ragged2e}

% für dem Befehlt \Longunderstack{} in Tabellen 
% Quelle: https://tex.stackexchange.com/a/183873/232548
\usepackage[usestackEOL]{stackengine}
\edef\tmp{\the\baselineskip}
\setstackgap{L}{\tmp}

% Funktioniert leider nicht:
%% Für das Malteserkreuz - Quelle: https://tex.stackexchange.com/questions/202746/how-to-draw-an-actual-maltese-cross/202856#202856
%\usepackage{fontspec}
%\usepackage{mathtools}
%\newcommand*{\amalfi}{\ensuremath{\text{\fontspec{code2000.ttf}\symbol{"2720}}}}

\usepackage{scalerel}
\newcommand{\maltcross}{\scalerel*{%
		\tikz\fill
		(0.02,0.02)    -- (0.2,0.5)   -- (0,0.4)  -- (-0.2,0.5)  --
		(-0.02,0.02)   -- (-0.5,0.2)  -- (-0.4,0) -- (-0.5,-0.2) --
		(-0.02,-0.02)  -- (-0.2,-0.5) -- (0,-0.4) -- (0.2,-0.5)  --
		(0.02,-0.02)   -- (0.5,-0.2)  -- (0.4,0)  -- (0.5,0.2)   --
		cycle;}% Scale this picture...
	{0}% ...to the size of this symbol.
}

\usepackage{tcolorbox}

%\usepackage[table]{xcolor}

%\usepackage{booktabs} 

% Text mit geschweifter Klammer und Text darunter
\newcommand{\undertext}[2] {$\underbrace{\textrm{#1}}_{\textrm{#2}}$} % Quelle: https://tex.stackexchange.com/questions/128064/how-to-overbrace-underbrace-in-text-mode-using-tikz

\usepackage{pifont}% http://ctan.org/pkg/pifont für \cmark und \xmark
\newcommand{\cmark}{\textcolor{green!50!black}{\ding{51}}}%
\newcommand{\xmark}{\textcolor{red!100!black}{\ding{55}}}%

\usepackage[autostyle,babel=true]{csquotes}	% Context sensitive quotation facilities
\renewcommand{\enquote}[1]{\glqq #1\grqq{}}


%\usepackage[upright]{fourier}				% Für aufrechte Labels der Punkte in tkz-euclide - Quelle: https://tikz.fr/tkz-euclide-simple-example/

\newcommand{\pfeilrunterbeschriftungrechts}[1]{\begin{tikzpicture} \draw[-Stealth, ultra thick] (0,0.5) -- node [right, xshift=2em] {#1} (0,-0.5); \end{tikzpicture}}

\usepackage{nicematrix} % Für schöne Tabellen mit dem Befehl \begin{NiceTabular}{|c|c|c|}[cell-space-limits=4pt] ... \end{NiceTabular}

\usepackage{tkz-euclide}

\usepackage[compat=1.1.0]{tikz-feynman}

\usepackage{tikz}							% Create PostScript and PDF graphics in TEX
	\usetikzlibrary{3d, angles, arrows, arrows.meta, babel, backgrounds, bending, calc, circuits.logic.US, decorations.markings, decorations.pathmorphing, decorations.pathreplacing, decorations.text, decorations, er, fit, graphs, intersections, matrix, mindmap, patterns, plotmarks, positioning, quotes, svg.path, shadows.blur, shapes.misc, tikzmark, trees, shapes, shadows, through, circuits.logic.US, circuits.logic.IEC, circuits.logic.CDH, circuits.ee.IEC}

\usepackage{tikz-qtree}

%%\usepackage[upright]{fourier} 
%\usepackage{fourier} 

\usetikzlibrary{positioning}

\usepackage[american, europeanresistors,americanvoltages,emptydiodes,siunitx]{circuitikz}
	\tikzset{circuit declare symbol = ammeter}
	\tikzset{set ammeter graphic ={draw,generic circle IEC, minimum size=5.5mm,info=center:A}}
	\tikzset{circuit declare symbol = voltmeter}
	\tikzset{set voltmeter graphic ={draw,generic circle IEC, minimum size=5.5mm,info=center:V}}

\usepackage{pgf}
\usepgflibrary{decorations.text}
\usepackage{pgfplots}
\pgfplotsset{compat=newest}

%  Für Schraffur (https://tex.stackexchange.com/questions/164773/graphics-area-between-curves)
\usepgfplotslibrary{fillbetween}
\usetikzlibrary{patterns}

	
\makeatletter
\tikzset{%
	remember picture with id/.style={%
		remember picture,
		overlay,
		save picture id=#1,
	},
	save picture id/.code={%
		\edef\pgf@temp{#1}%
		\immediate\write\pgfutil@auxout{%
			\noexpand\savepointas{\pgf@temp}{\pgfpictureid}}%
	},
	if picture id/.code args={#1#2#3}{%
		\@ifundefined{save@pt@#1}{%
			\pgfkeysalso{#3}%
		}{
			\pgfkeysalso{#2}%
		}
	}
}

\def\savepointas#1#2{%
	\expandafter\gdef\csname save@pt@#1\endcsname{#2}%
}

\def\tmk@labeldef#1,#2\@nil{%
	\def\tmk@label{#1}%
	\def\tmk@def{#2}%
}

\tikzdeclarecoordinatesystem{pic}{%
	\pgfutil@in@,{#1}%
	\ifpgfutil@in@%
	\tmk@labeldef#1\@nil
	\else
	\tmk@labeldef#1,(0pt,0pt)\@nil
	\fi
	\@ifundefined{save@pt@\tmk@label}{%
		\tikz@scan@one@point\pgfutil@firstofone\tmk@def
	}{%
		\pgfsys@getposition{\csname save@pt@\tmk@label\endcsname}\save@orig@pic%
		\pgfsys@getposition{\pgfpictureid}\save@this@pic%
		\pgf@process{\pgfpointorigin\save@this@pic}%
		\pgf@xa=\pgf@x
		\pgf@ya=\pgf@y
		\pgf@process{\pgfpointorigin\save@orig@pic}%
		\advance\pgf@x by -\pgf@xa
		\advance\pgf@y by -\pgf@ya
	}%
}
\makeatother

\newcommand{\tikzmarkindouble}[2][]{%
	\tikz[remember picture,overlay,baseline=1ex]
	\draw[line width=0.5pt,#1,double]
	(pic cs:#2) -- (0,0)
	;}

\newcommand{\tikzmarkin}[2][]{%
	\tikz[remember picture,overlay,baseline=1ex]
	\draw[line width=0.5pt,#1]
	(pic cs:#2) -- (0,0)
	;}

\newcommand\tikzmarkend[2][]{%
	\tikz[remember picture with id=#2,baseline=1ex] 
	#1;}

\newcommand{\tikzmarkindoublebaseline}[3][]{%
	\tikz[remember picture,overlay,baseline=#3ex]
	\draw[line width=0.5pt,#1,double]
	(pic cs:#2) -- (0,0)
	;}

\newcommand{\tikzmarkinbaseline}[3][]{%
	\tikz[remember picture,overlay,baseline=#3ex]
	\draw[line width=0.5pt,#1]
	(pic cs:#2) -- (0,0)
	;}

\newcommand\tikzmarkendbaseline[3][]{%
	\tikz[remember picture with id=#2,baseline=#3ex] 
	#1;}


\newcommand{\UpArrow}{\mathord{~\begin{tikzpicture}[baseline=0ex, line width=0.7, scale=0.12, ->, >=latex]
		\draw (0,0) -- (0,3);
		\end{tikzpicture}}}

\newcommand{\DownArrow}{\mathord{~\begin{tikzpicture}[baseline=0ex, line width=0.7, scale=0.12, <-, >=latex]
		\draw (0,0) -- (0,3);
		\end{tikzpicture}}}
	
\newcommand{\RightDownArrow}{\mathord{~\begin{tikzpicture}[baseline=0ex, line width=0.7, scale=0.12, ->, >=latex]
		\draw (0,3) -- (3,0);
		\draw (1,4) -- (4,1);
		\end{tikzpicture}}}

\newcommand{\RightUpArrow}{\mathord{~\begin{tikzpicture}[baseline=0ex, line width=0.7, scale=0.12, <-, >=latex]
		\draw (0,0) -- (0,3);
		\end{tikzpicture}}}

%%%%%%%%%%%%%%%%%%%%%%%%%%%
%% LowUpDep %%%%%%%%%%%%%%%%%%%%
%\tikzset{dependent/.style={annotation arrow/.style = {>=}}}
%%
%%
%%LowDep %%%%%
%\tikzset{LowDep/.style args={#1}{
%		append after command={%
%			\bgroup
%			[current point is local=true]
%			[every LowDep/.try]
%			[annotation arrow,-]
%			(-2.5\tikzcircuitssizeunit,-1.5\tikzcircuitssizeunit) edge[line to]
%			(-1.5\tikzcircuitssizeunit,-1.5\tikzcircuitssizeunit) node[xshift=3.0\tikzcircuitssizeunit]{#1}
%			\egroup%
%	}},
%	%
%	LowDep'/.style args={#1}{
%		append after command={%
%			\bgroup
%			[current point is local=true, yscale=-1]
%			[every LowDep/.try]
%			[annotation arrow,-]
%			(-2.5\tikzcircuitssizeunit,-1.5\tikzcircuitssizeunit) edge[line to]
%			(-1.5\tikzcircuitssizeunit,-1.5\tikzcircuitssizeunit) node[xshift=3.0\tikzcircuitssizeunit]{#1}
%			\egroup%
%	}}
%}
%%
%%
%%
%%UpDep %%%%%
%\tikzset{UpDep/.style args={#1}{
%		append after command={%
%			\bgroup
%			[current point is local=true]
%			[every UpDep/.try]
%			[annotation arrow,-]
%			%
%			(2.5\tikzcircuitssizeunit,1.5\tikzcircuitssizeunit)  edge[line to]
%			(1.5\tikzcircuitssizeunit,1.5\tikzcircuitssizeunit) node[xshift=-3.0\tikzcircuitssizeunit]{#1}
%			\egroup%
%	}},
%	%
%	UpDep'/.style args={#1}{
%		append after command={%
%			\bgroup
%			[current point is local=true, yscale=-1]
%			[every UpDep/.try]
%			[annotation arrow,-]
%			%
%			(2.5\tikzcircuitssizeunit,1.5\tikzcircuitssizeunit)  edge[line to]
%			(1.5\tikzcircuitssizeunit,1.5\tikzcircuitssizeunit) node[xshift=-3.0\tikzcircuitssizeunit]{#1}
%			\egroup%
%	}}
%}
%%%%%%%%%%%%%%%%%%%%%%%%%%%%


%\newcommand\circlearound[1]{\tikz[baseline]\node[draw,shape=circle,anchor=base] {#1} ;}

%\usepackage{relsize} % für {\larger\textcircled{\smaller[2]1}}

\newcommand*\circled[1]{\tikz[baseline=(char.base)]{ \node[shape=circle,draw,inner sep=2pt, minimum width=3ex, line width=0.75pt] (char) {\textsf{#1}};}}


%my-Meter - eigenes Messgerät zum Eintragen:
\tikzset{circuit declare symbol = mymeter}
\tikzset{set mymeter graphic ={draw,generic circle IEC, minimum size=7mm,info=center:}}


%\tikzset{circuit declare symbol = AC source}
%\tikzset{AC source IEC graphic/.style={
%		circuit symbol lines,
%		circuit symbol size=width 2 height 2,
%		shape=generic circle IEC,
%		/pgf/generic circle IEC/before background={
%			\pgfpathmoveto{\pgfpoint{-0.8pt}{0pt}}
%			\pgfpathsine{\pgfpoint{0.4pt}{0.4pt}}
%			\pgfpathcosine{\pgfpoint{0.4pt}{-0.4pt}}
%			\pgfpathsine{\pgfpoint{0.4pt}{-0.4pt}}
%			\pgfpathcosine{\pgfpoint{0.4pt}{0.4pt}}
%			\pgfusepath{stroke}
%		},
%		transform shape, draw
%	}
%}
%\tikzset{circuit ee IEC/.append style=
%	{set AC source graphic = AC source IEC graphic}
%}
%
%\tikzset{circuit declare symbol = AC generator}
%\tikzset{set AC generator graphic ={draw, minimum size=5.5mm,transform shape, info=center:{$\underset{\mathbf{\sim}}{\mathsf{G}}$}}}
%
%%%%%%%%%%%%%%%%%%%%%%%%%%%%%%%%%%%




%%%%% geschweifte Klammern in Tabellen - Aternative		

\usepackage{array, multirow, bigdelim, makecell, booktabs}
%& \hspace{-10em}\rdelim\}{3.5}{*}[\quad $\Rightarrow W_\text{th}  = \SI{4,182}{\kilo\joule}$] 

%%%%% geschweifte Klammern in Tabellen - Aternative		

% Zum extrahieren von Punktkoordinaten und Weiterverwendung
\newdimen\XCoord
\newdimen\YCoord
\newcommand*{\ExtractCoordinate}[1]{\path (#1); \pgfgetlastxy{\XCoord}{\YCoord};}


% für eine schöne Spule (wird aber aktuell nicht genutzt)
% \draw (0,-1) -- node[inductor={0}]{} ++(2,0) node[right]{\verb|inductor={0}|};
% Quelle: https://tex.stackexchange.com/questions/212077/circuitikz-inductor-style

\tikzset{
	inductor/.style args={#1}{
		sloped,
		allow upside down,
		minimum height=4mm,
		minimum width=10mm,
		path picture={%
			\pgfmathsetmacro{\addangle}{#1}
			\pgfmathsetmacro{\radius}{.8/(2+6*cos(\addangle))}
			\draw[white] (-4.5mm,0) -- (4.5mm,0);
			\draw[line join=bevel] (-5mm,0)-- (-4mm,0)
			arc (180:-\addangle:\radius)
			arc (180+\addangle:-\addangle:\radius)
			arc (180+\addangle:-\addangle:\radius)
			arc (180+\addangle:0:\radius)
			-- (5mm,0);
		}
	},
	inductor/.default={40}
}

\tikzstyle{background grid}=[draw, black!20,step=1*5mm]

%TikZ Grid Paper With Node Coordinates

\makeatletter% from https://tex.stackexchange.com/a/39698/121799
\def\grd@save@target#1{%
	\def\grd@target{#1}}
\def\grd@save@start#1{%
	\def\grd@start{#1}}
\tikzset{
	labeled grid/.style={
		to path={%
			\pgfextra{%
				\edef\grd@@target{(\tikztotarget)}%
				\tikz@scan@one@point\grd@save@target\grd@@target\relax
				\edef\grd@@start{(\tikztostart)}%
				\tikz@scan@one@point\grd@save@start\grd@@start\relax
				\draw[minor help lines] (\tikztostart) grid (\tikztotarget);
				\draw[major help lines] (\tikztostart) grid (\tikztotarget);
				\grd@start
				\pgfmathsetmacro{\grd@xa}{\the\pgf@x/1cm}
				\pgfmathsetmacro{\grd@ya}{\the\pgf@y/1cm}
				\grd@target
				\pgfmathsetmacro{\grd@xb}{\the\pgf@x/1cm}
				\pgfmathsetmacro{\grd@yb}{\the\pgf@y/1cm}
				\pgfmathsetmacro{\grd@xc}{\grd@xa + \pgfkeysvalueof{/tikz/grid with coordinates/major step}}
				\pgfmathsetmacro{\grd@yc}{\grd@ya + \pgfkeysvalueof{/tikz/grid with coordinates/major step}}
				%\foreach \x in {\grd@xa,\grd@xc,...,\grd@xb}
				%\node[anchor=north] at (\x,\grd@ya) {\pgfmathprintnumber{\x}};
				%\foreach \y in {\grd@ya,\grd@yc,...,\grd@yb}
				%\node[anchor=east] at (\grd@xa,\y) {\pgfmathprintnumber{\y}};
				\path foreach \x in {\grd@xa,\grd@xc,...,\grd@xb}
				{foreach \y in {\grd@ya,\grd@yc,...,\grd@yb}
					{ (\x,\y) node[grid with coordinates/grid label] {$(\pgfmathprintnumber{\x},\pgfmathprintnumber{\y})$}}};
			}
		}
	},
	minor help lines/.style={
		help lines,
		step=\pgfkeysvalueof{/tikz/grid with coordinates/minor step},
		draw=none
	},
	major help lines/.style={
		help lines,
		line width=\pgfkeysvalueof{/tikz/grid with coordinates/major line width},
		step=\pgfkeysvalueof{/tikz/grid with coordinates/major step},
		draw=none
	},
	grid with coordinates/.cd,
	minor step/.initial=.2,
	major step/.initial=1,
	major line width/.initial=0pt,
	grid label/.style={below,scale=0.3,color = black!20}
}
\makeatother

\newcommand{\lamp}{\!
\begin{tikzpicture}[scale=0.5, every node/.style={transform shape}]
\draw (0,0) to [lamp] (2,0);
\end{tikzpicture}\; }

\tikzstyle{background grid}=[draw, black!20,step=1*5mm]

\makeatletter
\pgfdeclareshape{neon lamp shape}
{
	\inheritsavedanchors[from=circle ee]
	\inheritanchor[from=circle ee]{center}
	\inheritanchor[from=circle ee]{north}
	\inheritanchor[from=circle ee]{south}
	\inheritanchor[from=circle ee]{east}
	\inheritanchor[from=circle ee]{west}
	\inheritanchor[from=circle ee]{north east}
	\inheritanchor[from=circle ee]{north west}
	\inheritanchor[from=circle ee]{south east}
	\inheritanchor[from=circle ee]{south west}
	\inheritanchor[from=circle ee]{input}
	\inheritanchor[from=circle ee]{output}
	\inheritanchorborder[from=circle ee]
	
	\backgroundpath{
		\pgf@process{\radius}
		\pgfutil@tempdima=\radius
		
		\pgfpathcircle{\centerpoint}{\pgfutil@tempdima}
		
		\pgfpathmoveto{\pgfpoint{-\pgfutil@tempdima}{0pt}}
		\pgfpathlineto{\pgfpoint{-0.4\pgfutil@tempdima}{0pt}}
		\pgfpathmoveto{\pgfpoint{-0.4\pgfutil@tempdima}{0.6\pgfutil@tempdima}}
		\pgfpathlineto{\pgfpoint{-0.4\pgfutil@tempdima}{-0.6\pgfutil@tempdima}}
		
		\pgfpathmoveto{\pgfpoint{\pgfutil@tempdima}{0pt}}
		\pgfpathlineto{\pgfpoint{0.4\pgfutil@tempdima}{0pt}}
		\pgfpathmoveto{\pgfpoint{0.4\pgfutil@tempdima}{0.6\pgfutil@tempdima}}
		\pgfpathlineto{\pgfpoint{0.4\pgfutil@tempdima}{-0.6\pgfutil@tempdima}}
		\pgfusepath{stroke}
		
		\pgfpathcircle{\pgfqpoint{0\pgfutil@tempdima}{0.55\pgfutil@tempdima}}{.1\pgfutil@tempdima}
		\pgfusepath{stroke}
	}
}
\makeatother

\newcommand{\mathdirectcurrent}{\mathrel{\mathpalette\mathdirectcurrentinner\relax}}
\newcommand{\mathdirectcurrentinner}[2]{%
	\settowidth{\dimen0}{$#1=$}%
	\vbox to .85ex {\offinterlineskip
		\hbox to \dimen0{\hss\leaders\hrule height 1pt\hskip.85\dimen0\hss}
		\vskip.35ex
		\hbox to \dimen0{\hss
			\leaders\hrule height 1pt\hskip.34\dimen0
			\hskip.17\dimen0
			\leaders\hrule height 1pt\hskip.34\dimen0
			%			\hskip.17\dimen0
			%			\leaders\hrule\hskip.17\dimen0
			\hss}
		%		\vfill
	}%
}
\newcommand{\textdirectcurrent}{\mathdirectcurrentinner{\textstyle}{}}

\newcommand\Ground{%
	\mathbin{\text{\begin{tikzpicture}[circuit ee IEC,yscale=0.6,xscale=0.5]
			\draw (0,2ex) to (0,0) node[ground,rotate=-90,xshift=.65ex] {};
			\end{tikzpicture}}}%
}


% schöne Pfeilspitzen
\tikzset{%
	>=latex,
	inner sep=0pt,%
	outer sep=0pt,%         Das ist auch der Abstand zwischen Messgeräten und Leitungen!!!
	mark coordinate/.style={inner sep=0pt,outer sep=0pt,minimum size=3pt,
		fill=black,circle}%
}


\newcommand\pgfmathsinandcos[3]{%
	\pgfmathsetmacro#1{sin(#3)}%
	\pgfmathsetmacro#2{cos(#3)}%
}
\newcommand\LongitudePlane[3][current plane]{%
	\pgfmathsinandcos\sinEl\cosEl{#2} % elevation
	\pgfmathsinandcos\sint\cost{#3} % azimuth
	\tikzset{#1/.style={cm={\cost,\sint*\sinEl,0,\cosEl,(0,0)}}}
}

\newcommand\LatitudePlane[3][current plane]{%
	\pgfmathsinandcos\sinEl\cosEl{#2} % elevation
	\pgfmathsinandcos\sint\cost{#3} % latitude
	\pgfmathsetmacro\yshift{\RadiusSphere*\cosEl*\sint}
	\tikzset{#1/.style={cm={\cost,0,0,\cost*\sinEl,(0,\yshift)}}} %
}
\newcommand\NewLatitudePlane[4][current plane]{%
	\pgfmathsinandcos\sinEl\cosEl{#3} % elevation
	\pgfmathsinandcos\sint\cost{#4} % latitude
	\pgfmathsetmacro\yshift{#2*\cosEl*\sint}
	\tikzset{#1/.style={cm={\cost,0,0,\cost*\sinEl,(0,\yshift)}}} %
}
\newcommand\DrawLongitudeCircle[2][1]{
	\LongitudePlane{\angEl}{#2}
	\tikzset{current plane/.prefix style={scale=#1}}
	% angle of "visibility"
	\pgfmathsetmacro\angVis{atan(sin(#2)*cos(\angEl)/sin(\angEl))} %
	\draw[current plane] (\angVis:1) arc (\angVis:\angVis+180:1);
	\draw[current plane,opacity=0.4] (\angVis-180:1) arc (\angVis-180:\angVis:1);
}
\newcommand\DrawLongitudeArc[4][black]{
	\LongitudePlane{\angEl}{#2}
	\tikzset{current plane/.prefix style={scale=1}}
	\pgfmathsetmacro\angVis{atan(sin(#2)*cos(\angEl)/sin(\angEl))} %
	\pgfmathsetmacro\angA{mod(max(\angVis,#3),360)} %
	\pgfmathsetmacro\angB{mod(min(\angVis+180,#4),360} %
	\draw[current plane,#1,opacity=0.4] (#3:\RadiusSphere) arc (#3:#4:\RadiusSphere);
	\draw[current plane,#1]  (\angA:\RadiusSphere) arc (\angA:\angB:\RadiusSphere);
}%
\newcommand\DrawLatitudeCircle[2][1]{
	\LatitudePlane{\angEl}{#2}
	\tikzset{current plane/.prefix style={scale=#1}}
	\pgfmathsetmacro\sinVis{sin(#2)/cos(#2)*sin(\angEl)/cos(\angEl)}
	% angle of "visibility"
	\pgfmathsetmacro\angVis{asin(min(1,max(\sinVis,-1)))}
	\draw[current plane] (\angVis:1) arc (\angVis:-\angVis-180:1);
	\draw[current plane,opacity=0.4] (180-\angVis:1) arc (180-\angVis:\angVis:1);
}

\newcommand\DrawLatitudeArc[4][black]{
	\LatitudePlane{\angEl}{#2}
	\tikzset{current plane/.prefix style={scale=1}}
	\pgfmathsetmacro\sinVis{sin(#2)/cos(#2)*sin(\angEl)/cos(\angEl)}
	% angle of "visibility"
	\pgfmathsetmacro\angVis{asin(min(1,max(\sinVis,-1)))}
	\pgfmathsetmacro\angA{max(min(\angVis,#3),-\angVis-180)} %
	\pgfmathsetmacro\angB{min(\angVis,#4)} %
	\draw[current plane,#1,opacity=0.4] (#3:\RadiusSphere) arc (#3:#4:\RadiusSphere);
	\draw[current plane,#1] (\angA:\RadiusSphere) arc (\angA:\angB:\RadiusSphere);
}


% Fading:
\makeatletter
\tikzset{ % based on https://tex.stackexchange.com/q/328433/121799
	/tikz/render blur shadow/.code={
		\pgfbs@savebb
		\pgfsyssoftpath@getcurrentpath{\pgfbs@input@path}%
		\pgfbs@compute@shadow@bbox
		\pgfbs@process@rounding{\pgfbs@input@path}{\pgfbs@fadepath}%
		\pgfbs@apply@canvas@transform
		\colorlet{pstb@shadow@color}{white!\pgfbs@opacity!\my@shadow@color}%
		\pgfdeclarefading{shadowfading}{\pgfbs@paint@fading}%
		\pgfsetfillcolor{\my@shadow@color}%
		\pgfsetfading{shadowfading}%
		{\pgftransformshift{\pgfpoint{\pgfbs@midx}{\pgfbs@midy}}}%
		\pgfbs@usebbox{fill}%
		\pgfbs@restorebb
	},}
\tikzset{
	/tikz/shadow color/.store in=\my@shadow@color,
	/tikz/shadow color=gray,
}
\makeatother

% Text entlang Winkel-Markierung nach 
%	https://tex.stackexchange.com/questions/234826/build-text-path-using-pic-from-library-angles

\newcommand{\escala}{.02} % escala
\tikzset{MeuAngulo/.style={draw=blue,<-,angle eccentricity=1.2,angle radius=25pt,semithick}}

\makeatletter
\tikzset{
	pics/deco angle/.style n args = 2{
		setup code  = \tikz@lib@angle@parse#1\pgf@stop,
		background code = \tikz@lib@angle@background#1\pgf@stop,
		foreground code = \tikz@lib@decoangle@foreground#1{#2}\pgf@stop,
	},
	pics/angle/.default=A--B--C,
	angle eccentricity/.initial=.6,
	angle radius/.initial=5mm
}

\def\tikz@lib@decoangle@foreground#1--#2--#3#4\pgf@stop{%
	\path [name prefix ..] [pic actions, fill=none, shade=none]
	([shift={(\tikz@start@angle@temp:\tikz@lib@angle@rad pt)}]#2.center)
	arc [start angle=\tikz@start@angle@temp, end
	angle=\tikz@end@angle@temp, radius=\tikz@lib@angle@rad pt];
	\path [name prefix ..] [fill=none, shade=none, decorate, decoration={text along path, text align=center, text={#4}}]
	([shift={(\tikz@end@angle@temp:\tikz@lib@angle@rad+2 pt)}]#2.center)
	arc [start angle=\tikz@end@angle@temp, end angle=\tikz@start@angle@temp, radius=\tikz@lib@angle@rad+2 pt];
}
\makeatother
% und jetzt noch für Text auf dem Kopf...
\makeatletter
\tikzset{
	pics/deco angle usd/.style n args = 2{
		setup code  = \tikz@lib@angle@parse#1\pgf@stop,
		background code = \tikz@lib@angle@background#1\pgf@stop,
		foreground code = \tikz@lib@decoangle@foreground#1{#2}\pgf@stop,
	},
	pics/angle/.default=A--B--C,
	angle eccentricity/.initial=.6,
	angle radius/.initial=5mm
}

\def\tikz@lib@decoangle@foreground#1--#2--#3#4\pgf@stop{%
	\path [name prefix ..] [pic actions, fill=none, shade=none] ([shift={(\tikz@start@angle@temp:\tikz@lib@angle@rad pt)}]#2.center)
	arc [start angle=\tikz@start@angle@temp, end angle=\tikz@end@angle@temp, radius=\tikz@lib@angle@rad pt];
	\path [name prefix ..] [fill=none, shade=none, decorate, decoration={text along path, text align=center, text={$|\scriptstyle|~#4~||$}}]
	([shift={(\tikz@end@angle@temp:\tikz@lib@angle@rad+3 pt)}]#2.center)
	arc [start angle=\tikz@end@angle@temp, end angle=\tikz@start@angle@temp, radius=\tikz@lib@angle@rad+2 pt];
}
\makeatother

%%%%%%%%%%%%%%% Highlighter
%%%%%%%%%%%%%%% Quelle. https://tex.stackexchange.com/questions/5959/cool-text-highlighting-in-latex

\usepackage{soulutf8}
\usepackage{atbegshi}
\usepackage{etoolbox}

\usetikzlibrary{tikzmark,calc,decorations.pathmorphing}

\colorlet{tdcolor}{yellow!35}

\makeatletter

\newlength{\txtdec@depth}
\setlength{\txtdec@depth}{.5ex}
\newlength{\txtdec@height}
\setlength{\txtdec@height}{\f@size pt} % sensible default

\newcounter{txtdec@hyphmark}
\newcounter{txtdec@decormark}
\setcounter{txtdec@decormark}{1}
\newcounter{txtdec@hyphdraw}
\setcounter{txtdec@hyphdraw}{1}
\newcounter{txtdec@decordraw}
\setcounter{txtdec@decordraw}{1}

\newcounter{txtdec@thenextpage}

\newtoggle{txtdec@unfinisheddecor}
\newtoggle{txtdec@stayonpage}
\newtoggle{txtdec@stayondecor}
\newtoggle{txtdec@stayonline}

% from https://tex.stackexchange.com/a/33765/105447
\newcommand{\gettikzxy}[3]{%
	\tikz@scan@one@point\pgfutil@firstofone#1\relax
	\edef#2{\the\pgf@x}%
	\edef#3{\the\pgf@y}%
}

% the drawing macros

\tikzset{%
	defaultdecor/.style={%
		fill=tdcolor,
		decoration = {random steps, amplitude=0.5pt, segment length=15pt},
		outer sep = -15pt,
		inner sep = 0pt,
		decorate}%
}

\newcommand{\txtdec@draw@all}{%
	\tikzset{thisdecor/.style/.expanded=\csuse{decor@tikz@style@\thetxtdec@decordraw}}%
	\path[defaultdecor, thisdecor]
	($(\Xbegin,\Ybegin)+(0,-\txtdec@depth)$) rectangle
	($(\Xend,\Yend)+(0,\txtdec@height-\txtdec@depth)$) ;
}

\newcommand{\txtdec@draw@begin}{%
	\tikzset{thisdecor/.style/.expanded=\csuse{decor@tikz@style@\thetxtdec@decordraw}}%
	\path[defaultdecor, thisdecor]
	($(\Xbegin,\Ybegin)+(0,-\txtdec@depth)$) rectangle
	($(\Xlineend,\Ylineend)+(0,\txtdec@height-\txtdec@depth)$) ;
}

\newcommand{\txtdec@draw@middle}{%
	\tikzset{thisdecor/.style/.expanded=\csuse{decor@tikz@style@\thetxtdec@decordraw}}%
	\path[defaultdecor, thisdecor]
	($(\Xlinebegin,\Ylinebegin)+(0,-\txtdec@depth)$) rectangle
	($(\Xlineend,\Ylineend)+(0,\txtdec@height-\txtdec@depth)$) ;
}

\newcommand{\txtdec@draw@end}{%
	\tikzset{thisdecor/.style/.expanded=\csuse{decor@tikz@style@\thetxtdec@decordraw}}%
	\path[defaultdecor, thisdecor]
	($(\Xlinebegin,\Ylinebegin)+(0,-\txtdec@depth)$) rectangle
	($(\Xend,\Yend)+(0,\txtdec@height-\txtdec@depth)$) ;
}


% using soul to set tikzmarks

\def\SOUL@tdleaders{%
	\stepcounter{txtdec@hyphmark}%
	\tikzmark{p\thepage.d\arabic{txtdec@decormark}.\arabic{txtdec@hyphmark}}%
	\leaders\hrule\@depth\z@\@height\z@\relax
}

\def\SOUL@tdunderline#1{{%
		\setbox\z@\hbox{#1}%
		\dimen@=\wd\z@
		\dimen@i=\SOUL@uloverlap
		\advance\dimen@2\dimen@i
		\rlap{%
			\null
			\kern-\dimen@i
			\SOUL@ulcolor{\SOUL@tdleaders\hskip\dimen@}%
			\hskip\dimen@
		}%
		\unhcopy\z@
}}

\def\SOUL@tdpreamble{%
	\spaceskip\SOUL@spaceskip
	\setcounter{txtdec@hyphmark}{0}%
	\tikzmark{p\thepage.d\arabic{txtdec@decormark}.begin}%
}
\def\SOUL@tdeverysyllable{%
	\SOUL@tdunderline{%
		\the\SOUL@syllable
		\SOUL@setkern\SOUL@charkern
	}%
	\stepcounter{txtdec@hyphmark}%
	\tikzmark{p\thepage.d\arabic{txtdec@decormark}.\arabic{txtdec@hyphmark}}%
}
\def\SOUL@tdeveryhyphen{%
	\discretionary{%
		\unkern
		\SOUL@tdunderline{%
			\SOUL@setkern\SOUL@hyphkern
			\SOUL@sethyphenchar
		}%
		\stepcounter{txtdec@hyphmark}%
		\tikzmark{p\thepage.d\arabic{txtdec@decormark}.\arabic{txtdec@hyphmark}}%
	}{}{}%
}
\def\SOUL@tdeveryexhyphen#1{%
	\SOUL@setkern\SOUL@hyphkern
	\SOUL@tdunderline{#1}%
	\stepcounter{txtdec@hyphmark}%
	\tikzmark{p\thepage.d\arabic{txtdec@decormark}.\arabic{txtdec@hyphmark}}%
	\discretionary{}{}{%
		\SOUL@setkern\SOUL@charkern
	}%
}
\def\SOUL@tdpostamble{%
	% create an extra mark, vertically displaced, to create an exit condition for the last line
	\stepcounter{txtdec@hyphmark}%
	\raisebox{-5pt}{\tikzmark{p\thepage.d\arabic{txtdec@decormark}.\arabic{txtdec@hyphmark}}}%
	\tikzmark{p\thepage.d\arabic{txtdec@decormark}.end}%
	\stepcounter{txtdec@decormark}%
}
\def\SOUL@tdsetup{%
	\SOUL@setup
	\let\SOUL@preamble\SOUL@tdpreamble
	\let\SOUL@everysyllable\SOUL@tdeverysyllable
	\let\SOUL@everyhyphen\SOUL@tdeveryhyphen
	\let\SOUL@everyexhyphen\SOUL@tdeveryexhyphen
	\let\SOUL@postamble\SOUL@tdpostamble
}
\DeclareRobustCommand*\textdecor[1][]{%
	\csxdef{decor@tikz@style@\thetxtdec@decormark}{#1}%
	\csxdef{decor@fsize@\thetxtdec@decormark}{\f@size pt}%
	\SOUL@tdsetup\SOUL@}


% get the drawing done AtBeginShipout

\AtBeginShipout{%
	\AtBeginShipoutUpperLeft{%
		% getting the number of the next page
		\setcounter{txtdec@thenextpage}{\thepage}%
		\stepcounter{txtdec@thenextpage}%
		% if the current decoration occurs on this page, stay on it
		\iftikzmark{p\thepage.d\arabic{txtdec@decordraw}.\arabic{txtdec@hyphdraw}}{%
			\toggletrue{txtdec@stayonpage}}{}%
		\whileboolexpr{togl {txtdec@stayonpage}}{%
			\begin{tikzpicture}[remember picture, overlay]
			\setlength{\txtdec@height}{\csuse{decor@fsize@\thetxtdec@decordraw}}%
			\iftikzmark{p\thepage.d\arabic{txtdec@decordraw}.begin}{%
				% if current decor begins in current page, get coordinates
				\gettikzxy{(pic cs:p\thepage.d\arabic{txtdec@decordraw}.begin)}{\Xbegin}{\Ybegin}}{%
				% if current decor begins in previous page, set to top left of the page
				\gettikzxy{(current page.north west)}{\Xbegin}{\Ybegin}}%
			\iftikzmark{p\thepage.d\arabic{txtdec@decordraw}.end}{%
				% if current decor ends in current page, get coordinates
				\gettikzxy{(pic cs:p\thepage.d\arabic{txtdec@decordraw}.end)}{\Xend}{\Yend}}{%
				% if current decor ends in future page, set to bottom right of the page
				\gettikzxy{(current page.south east)}{\Xend}{\Yend}}%
			\ifdim\Ybegin=\Yend % the simplest case, a single line
			\txtdec@draw@all
			\stepcounter{txtdec@decordraw}%
			\else % current textdecor has a line break          
			\toggletrue{txtdec@stayondecor}%
			\whileboolexpr{togl {txtdec@stayondecor}}{%
				\gettikzxy{(pic cs:p\thepage.d\arabic{txtdec@decordraw}.\arabic{txtdec@hyphdraw})}{\Xlinebegin}{\Ylinebegin}%
				\edef\Xcurrent{\Xlinebegin}%
				\edef\Ycurrent{\Ylinebegin}%
				\edef\Xnext{\Xcurrent}%
				\edef\Ynext{\Ycurrent}%
				\toggletrue{txtdec@stayonline}%
				\whileboolexpr{togl {txtdec@stayonline}}{%
					\ifdim\Ycurrent=\Ynext
					\stepcounter{txtdec@hyphdraw}%
					% if the following tikzmark exists, we are at a page break
					\iftikzmark{p\arabic{txtdec@thenextpage}.d\arabic{txtdec@decordraw}.\arabic{txtdec@hyphdraw}}{%
						\edef\Xcurrent{\Xnext}%
						\edef\Ycurrent{\Ynext}%
						\gettikzxy{(current page.south east)}{\Xnext}{\Ynext}
						\togglefalse{txtdec@stayondecor}%
					}{% else, we remain on the same page
						\iftikzmark{p\thepage.d\arabic{txtdec@decordraw}.\arabic{txtdec@hyphdraw}}{%
							\edef\Xcurrent{\Xnext}%
							\edef\Ycurrent{\Ynext}%
							\gettikzxy{(pic cs:p\thepage.d\arabic{txtdec@decordraw}.\arabic{txtdec@hyphdraw})}{\Xnext}{\Ynext}}{}%
					}%
					\else
					\edef\Xlineend{\Xcurrent}%
					\edef\Ylineend{\Ycurrent}%
					% if we are on the first line of the current decoration
					\ifdim\Ylinebegin=\Ybegin
					\txtdec@draw@begin
					\else
					% if we are on the last line of the current decoration
					\ifdim\Ycurrent=\Yend
					\txtdec@draw@end
					\stepcounter{txtdec@decordraw}%
					\setcounter{txtdec@hyphdraw}{1}%
					\togglefalse{txtdec@stayondecor}%
					% if we are in a middle line of the decoration
					\else
					\txtdec@draw@middle
					\fi
					\fi
					\togglefalse{txtdec@stayonline}%  
					\fi
				}%
			}%
			\fi
			\end{tikzpicture}%
			% if the beginning of the next decor does not exist in this page, leave this page
			\iftikzmark{p\thepage.d\arabic{txtdec@decordraw}.begin}{}{%
				\togglefalse{txtdec@stayonpage}}%
			% if the continuation of the current decor exists in the next page, leave this page
			\iftikzmark{p\arabic{txtdec@thenextpage}.d\arabic{txtdec@decordraw}.\arabic{txtdec@hyphdraw}}{%
				\togglefalse{txtdec@stayonpage}}{}% 
		}%
	}%
}

\makeatother

%%%%% geschweifte Klammern von A nach B

%\newcommand{\tikzmark}[2][-3pt]{\tikz[remember picture, overlay, baseline=-0.5ex]\node[#1](#2){};}
%
%\tikzset{brace/.style={decorate, decoration={brace}},
%	brace mirrored/.style={decorate, decoration={brace,mirror}},
%}
%
%\newcounter{brace}
%\setcounter{brace}{0}
%\newcommand{\drawbrace}[3][brace]{%
%	\refstepcounter{brace}
%	\tikz[remember picture, overlay]\draw[#1] (#2.center)--(#3.center)node[pos=0.5, name=brace-\thebrace]{};
%}
%
%\newcounter{arrow}
%\setcounter{arrow}{0}
%\newcommand{\drawcurvedarrow}[3][]{%
%	\refstepcounter{arrow}
%	\tikz[remember picture, overlay]\draw (#2.center)edge[#1]node[coordinate,pos=0.5, name=arrow-\thearrow]{}(#3.center);
%}
%
%% #1 options, #2 position, #3 text 
%\newcommand{\annote}[3][]{%
%	\tikz[remember picture, overlay]\node[#1] at (#2) {#3};
%}

\renewcommand{\tikzmark}[2][-3pt]{\tikz[remember picture, overlay, baseline=-0.5ex]\node[#1](#2){};}

\tikzset{brace/.style={decorate, decoration={brace}},
	brace mirrored/.style={decorate, decoration={brace,mirror}},
}

\newcounter{brace}
\setcounter{brace}{0}
\newcommand{\drawbrace}[3][brace]{%
	\refstepcounter{brace}
	\tikz[remember picture, overlay]\draw[#1] (#2.center)--(#3.center)node[pos=0.5, name=brace-\thebrace]{};
}

\newcounter{arrow}
\setcounter{arrow}{0}
\newcommand{\drawcurvedarrow}[3][]{%
	\refstepcounter{arrow}
	\tikz[remember picture, overlay]\draw (#2.center)edge[#1]node[coordinate,pos=0.5, name=arrow-\thearrow]{}(#3.center);
}

\newcommand{\annote}[3][]{%
	\tikz[remember picture, overlay]\node[#1] at (#2) {#3};
}


%%%%% geschweifte Klammern von A nach B

\newcommand*\mycirc[1]{%
	\begin{tikzpicture}[baseline=(C.base)]
	\node[draw,circle,inner sep=1pt](C) {#1};
	\end{tikzpicture}}

%%%%% Lüchentext \fib nach 
%%%%% https://tobiw.de/tbdm/lueckentexte

\makeatletter

	\newlength\fib@width
	
	\newcommand\fib@widthfactor{1.34}
	
	\newlength\shortfib@width
	
	\newcommand\shortfib@widthfactor{1} % war 0.7
	
	\newif\iffibhideanswer
	\fibhideanswertrue
	
	\tikzset{
		every fill in box/.style={
			inner xsep=5pt,
			minimum height=4ex,
			align=center, %% war "center"
			font={\sffamily\slshape\bfseries},
			color=magenta,
			fill=black!5!white%dazu gedichtet
		},
		colored box/.style={
			every fill in box,
			fill=yellow!50!white,
		},
		framed box/.style={
			every fill in box,
			draw,
		},
		underline style/.style={},
		underlined box/.style={
			every fill in box,
			append after command={%
				\pgfextra{\begin{pgfinterruptpath}
						\draw [underline style,dashed,color=gray] (\tikzlastnode.south west)
						-- (\tikzlastnode.south east);
				\end{pgfinterruptpath}}
			},
		},
		bracket style/.style={},
		underbracked box/.style={
			every fill in box,
			append after command={%
				\pgfextra{\begin{pgfinterruptpath}
						\draw [bracket style]
						($(\tikzlastnode.south west)+(0,2pt)$)
						|- (\tikzlastnode.south)
						-| ($(\tikzlastnode.south east)+(0,2pt)$);
				\end{pgfinterruptpath}}
			},
		},
		%%%
		bracket style/.style={},
		simple box/.style={
			every fill in box,
			append after command={
				\pgfextra{
					%			\begin{pgfinterruptpath}
					%				\draw [bracket style]
					%				($(\tikzlastnode.south west)+(0,2pt)$)
					%				|- (\tikzlastnode.south)
					%				-| ($(\tikzlastnode.south east)+(0,2pt)$);
					%			\end{pgfinterruptpath}
				}
			},
		},
		%%%
		% von mir dazu gedichtet, um z. B. in Merkkästen Lingsbündig zu schreiben
		bracket style/.style={},
		underleft box/.style={
			every fill in box,
			align=left,
			append after command={%
				\pgfextra{\begin{pgfinterruptpath}
						\draw [bracket style]
						($(\tikzlastnode.south west)+(0,2pt)$)
						|- (\tikzlastnode.south)
						-| ($(\tikzlastnode.south east)+(0,2pt)$);
				\end{pgfinterruptpath}}
			},
		},
		%%%%%%%%%%%%%%%%%%%%%%%%%%%%%%%%%%%%%%%%%%%%%%%%%%%%%%%%%%%%%%%%%%%%%%%%%
		underoverbracked box/.style={
			every fill in box,
			append after command={%
				\pgfextra{\begin{pgfinterruptpath}
						\draw [bracket style]
						($(\tikzlastnode.north west)-(0,2pt)$)
						|- (\tikzlastnode.north)
						-| ($(\tikzlastnode.north east)-(0,2pt)$);
						\draw [bracket style]
						($(\tikzlastnode.south west)+(0,2pt)$)
						|- (\tikzlastnode.south)
						-| ($(\tikzlastnode.south east)+(0,2pt)$);
				\end{pgfinterruptpath}}
			},
		},
		fill in/.style={
			colored box,
		},
	}
	\tikzset{
		every fill in box/.style={
			inner xsep=4pt,			% wie groß ist der x-Abstand der Box vom Text
			inner ysep=2pt,			% wie groß ist der y-Abstand der Box vom Text
			minimum height=4ex,
			align=center, %% war "center"
			font={\sffamily\bfseries}, % war: 		font={\sffamily\slshape\bfseries},
			color=magenta,
			color=blue,
			fill=black!5!white%dazu gedichtet
		},
		%%%
		bracket style/.style={},
		simple box/.style={
			every fill in box,
			append after command={
				\pgfextra{
					%			\begin{pgfinterruptpath}
					%				\draw [bracket style]
					%				($(\tikzlastnode.south west)+(0,2pt)$)
					%				|- (\tikzlastnode.south)
					%				-| ($(\tikzlastnode.south east)+(0,2pt)$);
					%			\end{pgfinterruptpath}
				}%
			},
		},
		%%%
		fill in/.style={
			colored box,
		},
	}
		
	\NewDocumentCommand { \fib@hide } { m } {%
		\iffibhideanswer
		\phantom{#1}%
		\else
			{\boldmath\bfseries#1}%
		\fi
	}
	
	\NewDocumentCommand { \shortfib@hide } { m } {%
		\iffibhideanswer
		\phantom{#1}%
		\else
			{\boldmath\bfseries#1}%
		\fi
	}
	
	\NewDocumentCommand { \fib@makebox }{ m }{%
		\settowidth{\fib@width}{\tikz\node[fill in]{#1};}%
		\begin{tikzpicture}[baseline=(fill in node.base)]
		\node (fill in node)
		[text width=\fib@widthfactor*\fib@width,fill in]
		{%
			\fib@hide{#1}%
		};
		\end{tikzpicture}%
	}
	
	\NewDocumentCommand { \shortfib@makebox }{ m }{%
		\settowidth{\shortfib@width}{\tikz\node[fill in]{#1};}%
		\begin{tikzpicture}[baseline=(fill in node.base)]
		\node (fill in node)
		[text width=\shortfib@widthfactor*\shortfib@width,fill in]
		{%
			\shortfib@hide{#1}%
		};
		\end{tikzpicture}%
	}
	
	\NewDocumentCommand { \fib } { s o m }{{%
			\IfBooleanT{#1}{\fibhideanswerfalse}%
			\IfValueT{#2}{\tikzset{fill in/.style={#2}}}%
			\ifmmode
				\mathchoice
					{\fib@makebox{$\displaystyle#3$}}
					{\fib@makebox{$\textstyle#3$}}
					{\fib@makebox{$\scriptstyle#3$}}
					{\fib@makebox{$\scriptscriptstyle#3$}}
			\else
				\fib@makebox{{\boldmath\bfseries#3}}% <---- Dieses Kommentarzeichen war für die Lücken hinter den fibs verantwortlich!!!!
				%https://tex.stackexchange.com/questions/288561/how-to-boldify-a-part-of-text-with-maths
				%https://tex.stackexchange.com/questions/253105/question-about-bolding-math-and-text-simultaneously
			\fi
	}}
	
	\NewDocumentCommand { \shortfib } { s o m }{{%
			\IfBooleanT{#1}{\fibhideanswerfalse}%
			\IfValueT{#2}{\tikzset{fill in/.style={#2}}}%
			\ifmmode
				\mathchoice
					{\shortfib@makebox{$\displaystyle#3$}}
					{\shortfib@makebox{$\textstyle#3$}}
					{\shortfib@makebox{$\scriptstyle#3$}}
					{\shortfib@makebox{$\scriptscriptstyle#3$}}
			\else
				\shortfib@makebox{{\boldmath\bfseries#3}}% <---- Dieses Kommentarzeichen war für die Lücken hinter den fibs verantwortlich!!!!
				%https://tex.stackexchange.com/questions/288561/how-to-boldify-a-part-of-text-with-maths
				%https://tex.stackexchange.com/questions/253105/question-about-bolding-math-and-text-simultaneously
			\fi
	}}

\makeatother

%%%%%%%%%%%%%%%
%oberhalb gibt es wohl den Grund für die Lücke nach einer fib-Lücke!
%%%%%%%%%%%%%%%

% Definition des Standards:

%\tikzset{fill in/.style={colored box}}% = Voreinstellung
%\tikzset{fill in/.style={framed box}}
%\tikzset{fill in/.style={underlined box}}
%\tikzset{fill in/.style={underbracked box}}
\tikzset{fill in/.style={underleft box}}
%\tikzset{fill in/.style={underoverbracked box}}

\tikzset{colored box/.append style={fill=black!5!white}}
\tikzset{underline style/.style={dotted, thick}}
\tikzset{bracket style/.style={gray, thick}}

%%%%%%%%%%%%%%%%%%%%%%%%%%%%%%%%%%%
%https://tobiw.de/tbdm/lueckentexte
%%%%%%%%%%%%%%%%%%%%%%%%%%%%%%%%%%%

% Einfärben eines Ringes:

\tikzset{
	ring shading/.code args={from #1 at #2 to #3 at #4}{
		\pgfmathsetmacro{\inner}{25*#2/#4}
		\pgfmathsetmacro{\innerlow}{\inner-1}
		\pgfdeclareradialshading{ring}{\pgfpoint{0cm}{0cm}}%
		{
			color(0bp)=(white);
			color(\innerlow bp)=(white);
			color(\inner bp)=(#1);
			color(25bp)=(#3);
			color(50bp)=(black)
		}
		\pgfkeysalso{/tikz/shading=ring}
	},
}


%%% Für Baumdiagramme
%\tikzset{
%	basic/.style  = 	{draw, font=\sffamily, rectangle},
%	root/.style   = 	{basic, text width=4cm, rounded corners=2pt, thick, align=center, fill=white!30, minimum height=1cm},
%	level 1/.style = 	{basic, text width=6cm, rounded corners=2pt, thick, align=center, fill=white!60, minimum height=1.5cm, sibling distance = 7cm},
%	level 2/.style = 	{basic, text width=5cm, rounded corners=2pt, thick, align=center, fill=white!60, minimum height=1.5cm, sibling distance = 6cm},
%}

\makeatletter
\newif\if@showgrid@grid
\newif\if@showgrid@left
\newif\if@showgrid@right
\newif\if@showgrid@below
\newif\if@showgrid@above
\tikzset{%
	every show grid/.style={},
	show grid/.style={execute at end picture={\@showgrid{grid=true,#1}}},%
	show grid/.default={true},
	show grid/.cd,
	labels/.style={font={\sffamily\small},help lines},
	xlabels/.style={},
	ylabels/.style={},
	keep bb/.code={\useasboundingbox (current bounding box.south west) rectangle (current bounding box.north west);},
	true/.style={left,below},
	false/.style={left=false,right=false,above=false,below=false,grid=false},
	none/.style={left=false,right=false,above=false,below=false},
	all/.style={left=true,right=true,above=true,below=true},
	grid/.is if=@showgrid@grid,
	left/.is if=@showgrid@left,
	right/.is if=@showgrid@right,
	below/.is if=@showgrid@below,
	above/.is if=@showgrid@above,
	false,
}

\def\@showgrid#1{%
	\begin{scope}[every show grid,show grid/.cd,#1]
		\if@showgrid@grid
		\begin{pgfonlayer}{background}
			\draw [help lines]
			(current bounding box.south west) grid
			(current bounding box.north east);
			%
			\pgfpointxy{1}{1}%
			\edef\xs{\the\pgf@x}%
			\edef\ys{\the\pgf@y}%
			\pgfpointanchor{current bounding box}{south west}
			\edef\xa{\the\pgf@x}%
			\edef\ya{\the\pgf@y}%
			\pgfpointanchor{current bounding box}{north east}
			\edef\xb{\the\pgf@x}%
			\edef\yb{\the\pgf@y}%
			\pgfmathtruncatemacro\xbeg{ceil(\xa/\xs)}
			\pgfmathtruncatemacro\xend{floor(\xb/\xs)}
			\if@showgrid@below
			\foreach \X in {\xbeg,...,\xend} {
				\node [below,show grid/labels,show grid/xlabels] at (\X,\ya) {\X};
			}
			\fi
			\if@showgrid@above
			\foreach \X in {\xbeg,...,\xend} {
				\node [above,show grid/labels,show grid/xlabels] at (\X,\yb) {\X};
			}
			\fi
			\pgfmathtruncatemacro\ybeg{ceil(\ya/\ys)}
			\pgfmathtruncatemacro\yend{floor(\yb/\ys)}
			\if@showgrid@left
			\foreach \Y in {\ybeg,...,\yend} {
				\node [left,show grid/labels,show grid/ylabels] at (\xa,\Y) {\Y};
			}
			\fi
			\if@showgrid@right
			\foreach \Y in {\ybeg,...,\yend} {
				\node [right,show grid/labels,show grid/ylabels] at (\xb,\Y) {\Y};
			}
			\fi
		\end{pgfonlayer}
		\fi
	\end{scope}
}
\makeatother

\usepackage{eqparbox}
\usepackage{expl3}							% für Berechnungen - siehe https://tex.stackexchange.com/questions/196171/inaccurate-pgf-calculation
\ExplSyntaxOn
\cs_set_eq:NN \fpeval \fp_eval:n
\ExplSyntaxOff

\usepackage{pgf}
\usepackage{tkz-euclide}
%\usepackage{amsfonts}						% TEX fonts from the American Mathematical Society
\usepackage{wasysym}   						% Electrical and physical symbols \AC \HF \VHF \photon \gluon
%\usepackage{mathtools}						% Mathematical tools to use with amsmath 
\usepackage{amsmath}						% AMS mathematical facilities for LATEX
\usepackage{amssymb}
\usepackage{isotope}						% A package for typesetting isotopes
%\usepackage{bohr}							% simple atom representation according to the Bohr model
%\usepackage{elements}
\usepackage[version=4]{mhchem}				% \ce{H2O}
\usepackage{fixmath}						% für $\mathbold{x}$
\usepackage{dsfont}							% Type­set math­e­mat­i­cal dou­ble stroke sym­bols
\usepackage{float}
%\usepackage{xfp}

\usepackage{scalerel}
\usepackage{stackengine}

%\usepackage{bm} % für fette Mathematik

\newcommand\reallywidetilde[1]{\ThisStyle{%
		\setbox0=\hbox{$\SavedStyle#1$}%
		\stackengine{-.5\LMpt}{$\SavedStyle#1$}{%
			\stretchto{\scaleto{\SavedStyle\mkern.2mu\frown}{.4\wd0}}{.5\ht0}%
		}{O}{c}{F}{T}{S}%
}}

\def\bogen#1{\ensuremath{\reallywidetilde{\text{#1}}}}

\usepackage{mathtools}						% für \begin{spreadlines}{2ex} Abstand in align ändern \end{spreadlines}


\usepackage{xstring}

%\newcommand{\pkt}[3]{%
%		\FPset\xpkt{#2}%
%		\FPset\ypkt{#3}%
%	\ensuremath{\text{#1}(\StrLeft{#2}{1}[\firstchar]\IfStrEq{\firstchar}{-}{\,}{\,}\num{\FPprint\xpkt}\,|\StrLeft{#3}{1}[\firstchari]\IfStrEq{\firstchari}{-}{\,}{\,}\num{\FPprint\ypkt}\,)}}

\newcommand{\pkt}[3]{\ensuremath{\mathrm{#1}\left(\,\SIx{#2}{}\,\left|\,\SIx{#3}{}\right.\,\right)}}
	
%\newcommand{\pkti}[3]{\ensuremath{\text{#1}(\;#3\;|\;#2\;)}}
%
%\newcommand{\pktii}[3]{P$(-2\,|-8\,)$}


%\newcommand{\pkt}[3]{% Funktioniert leider nur für Zahlen als Koordinaten...
%	\FPset\x{#2}%
%	\FPset\y{#3}%
%	\ensuremath{\text{#1}%
%	(%
%	\FPiflt{\x}{0}
%	%
%	\else
%	\,
%	\fi%
%	\num[round-mode=places,round-precision=2,zero-decimal-to-integer]{\FPprint\x}\,|%
%	\FPiflt{\y}{0}
%	\else
%	\,
%	\fi%
%	\num[round-mode=places,round-precision=2,zero-decimal-to-integer]{\FPprint\y}\,)}
%}

%Taschenrechnerbefehl für Quad. ergänzen und Nullstellen:
\newcommand{\trquad}[3]{
	\FPset\a{#1}
	\FPset\b{#2}
	\FPset\c{#3}
	\ensuremath{\textsf{TR  \; Menu \; A} \rightarrow 2 \rightarrow 2  \quad\; a=\num{\a}\; \rightarrow \; b=\num{\b}\; \rightarrow \; c=\num{\c}}}


%\usepackage{xfp}					% Interface to the LATEX3 floating point unit
%Taschenrechnerbefehl für Nullstellen und Extremwerte:
\newcommand{\trquadl}[3]{			% "Taschenrechner quadratische Gleichung mit Lösung" Eingabe der Koeffizienten mit Dezimalpunkt! 
	\FPset\a{#1}
	\FPset\b{#2}
	\FPset\c{#3}
	\FPeval\d{b*b-4*a*c}			% Berechnung der Diskriminante
	\ensuremath{
	\textsf{TR  \; Menu \; A} \rightarrow 2 \rightarrow 2  \quad\; a=\num{\a}\; \rightarrow \; b=\num{\b}\; \rightarrow \; c=\num{\c} \\[0.5ex]  			% Eingabe am TaschenRechner
%		(a=\a \quad b=\b \quad c=\c \quad D=\num[round-mode=places,round-precision=2]{\k}) \quad 
	\FPifgt{\d}{0} 					% wenn D>0
			\FPeval\e{round((-b+d^0.5)/(2*a),2)} % ohne round(...,2) kann auch mal "-0" rauskommen...
			\FPeval\f{round((-b-d^0.5)/(2*a),2)}
			\sisetup{round-mode=places,round-integer-to-decimal}
			x_1=\num[round-mode=places,round-precision=2,zero-decimal-to-integer]{\FPprint\e} \quad x_2=\num[round-mode=places,round-precision=2,zero-decimal-to-integer]{\FPprint\f} \qquad \mathds{L}=\{\num[round-mode=places,round-precision=2,zero-decimal-to-integer]{\FPprint\e}; \num[round-mode=places,round-precision=2,zero-decimal-to-integer]{\FPprint\f}\} \quad (D=\num[round-mode=places,round-precision=2,zero-decimal-to-integer]{\FPprint\d}>0)
		\else 
		\fi
	\FPifeq{\d}{0} 					% wenn D=0
			\FPeval\e{round((-b)/(2*a),2)} % 
			x=\num[round-mode=places,round-precision=2,zero-decimal-to-integer]{\FPprint\e} \qquad \mathds{L}=\{\num[round-mode=places,round-precision=2,zero-decimal-to-integer]{\FPprint\e}\} \quad (D=\num[round-mode=places,round-precision=2,zero-decimal-to-integer]{\FPprint\d})
		\else 
		\fi
	\FPiflt{\d}{0} 					% wenn D<0
			\mathds{L}=\varnothing \quad (D=\num[round-mode=places,round-precision=2,zero-decimal-to-integer]{\FPprint\d}<0)
		\else 
		\fi
}	
}

\newcommand{\trquadnl}[5]{			% "Taschenrechner quadratische Gleichung mit Lösung" Eingabe der Koeffizienten mit Dezimalpunkt! 
	\FPset\a{#1}
	\FPset\b{#2}
	\FPset\c{#3}
	\FPeval\d{b*b-4*a*c}			% Berechnung der Diskriminante
	\ensuremath{
		\textsf{TR  \; Menu \; A} \rightarrow 2 \rightarrow 2  \quad\; a=\num{\a}\; \rightarrow \; b=\num{\b}\; \rightarrow \; c=\num{\c} \\[0.5ex]  			% Eingabe am TaschenRechner
		%		(a=\a \quad b=\b \quad c=\c \quad D=\num[round-mode=places,round-precision=2]{\k}) \quad 
		\FPifgt{\d}{0} 					% wenn D>0
		\FPeval\e{round((-b+d^0.5)/(2*a),2)} % ohne round(...,2) kann auch mal "-0" rauskommen...
		\FPeval\f{round((-b-d^0.5)/(2*a),2)}
		\sisetup{round-mode=places,round-integer-to-decimal}
		x_#4=\num[round-mode=places,round-precision=2,zero-decimal-to-integer]{\FPprint\e} \quad x_#5=\num[round-mode=places,round-precision=2,zero-decimal-to-integer]{\FPprint\f} \qquad \mathds{L}=\{\num[round-mode=places,round-precision=2,zero-decimal-to-integer]{\FPprint\e}; \num[round-mode=places,round-precision=2,zero-decimal-to-integer]{\FPprint\f}\} \quad (D=\num[round-mode=places,round-precision=2,zero-decimal-to-integer]{\FPprint\d}>0)
		\else 
		\fi
		\FPifeq{\d}{0} 					% wenn D=0
		\FPeval\e{round((-b)/(2*a),2)} % 
		x=\num[round-mode=places,round-precision=2,zero-decimal-to-integer]{\FPprint\e} \qquad \mathds{L}=\{\num[round-mode=places,round-precision=2,zero-decimal-to-integer]{\FPprint\e}\} \quad (D=\num[round-mode=places,round-precision=2,zero-decimal-to-integer]{\FPprint\d})
		\else 
		\fi
		\FPiflt{\d}{0} 					% wenn D<0
		\mathds{L}=\varnothing \quad (D=\num[round-mode=places,round-precision=2,zero-decimal-to-integer]{\FPprint\d}<0)
		\else 
		\fi
	}	
}

%Taschenrechnerbefehl für Lineares Gleichungssystem:
\newcommand{\trlgs}[6]{\ensuremath{\textsf{TR  \; Menu \; A} \rightarrow 1 \rightarrow 2  \quad\; #1 \; \rightarrow \; #2 \; \rightarrow \; #3 \quad \rightarrow \quad  #4 \; \rightarrow \; #5 \; \rightarrow \; #6}}

\usepackage{siunitx}						% A comprehensive (SI) units package
	\sisetup{output-decimal-marker = {,}}	% Komma als Dezimaltrennzeichen für siunitx
	\sisetup{exponent-product = \cdot}		% Punkt statt x zwischen Zahl und Zehnerpotenz
	\sisetup{per-mode = fraction, fraction-function=\tfrac}			% Bruchstrich in Einheit
%	\sisetup{inter-unit-product = \ensuremath{{}\cdot{}}}
	\sisetup{inter-unit-product = \cdot}
	\sisetup{detect-weight=true, detect-family=true} 
	\sisetup{detect-all=true} % für fettes SI - Quelle: https://tex.stackexchange.com/questions/610211/how-to-have-bold-unit-with-siunitx
%	\sisetup{detect-weight=true, detect-family=true, mode=text} 
		% für fettes SI Quelle: https://tex.stackexchange.com/questions/612182/siunitx-bold-units-in-math-mode
		% Quelle: https://tex.stackexchange.com/questions/193389/siunitx-per-mode-fraction-in-displaystyle-with-tfrac
%	\sisetup{round-integer-to-decimal}
	\DeclareSIUnit\As{\text{As}}
	\DeclareSIUnit\Ah{\text{Ah}}
	\DeclareSIUnit\mAh{\text{mAh}}
	\DeclareSIUnit{\liter}{\ell}
	\DeclareSIPrefix\hekto{h}{h}
	\DeclareSIPrefix\zpmzwei{10^{-2}\,}{-2}
	\DeclareSIPrefix\zpmdrei{10^{-3}\,}{-3}
	\DeclareSIPrefix\zpmvier{10^{-4}\,}{-4}
	\DeclareSIPrefix\zpmsechs{10^{-6}\,}{-6}
%
	\DeclareSIPrefix\zppzwei{10^{2}\,}{2}
	\DeclareSIPrefix\zppdrei{10^{3}\,}{3}
	\DeclareSIPrefix\zppvier{10^{4}\,}{4}
	\DeclareSIPrefix\zppzweicenti{10^{2}\,c}{4}
	\DeclareSIPrefix\zppzweideci{10^{2}\,d}{4}
	\DeclareSIPrefix\zppdreideci{10^{3}\,d}{4}
	\DeclareSIPrefix\zppviercenti{10^{4}\,c}{4}
	\DeclareSIPrefix\zppfuenf{10^{5}\,}{5}
	\DeclareSIPrefix\zppsechs{10^{6}\,}{6}
\newcommand{\SIx}[2]{\SI[parse-numbers = false]{\text{$#1$}}{#2}}%damit gehen auch beliebige Terme als \num in \SI{\num}{\unit}

\AtBeginDocument{\DeclareSIUnit{\Wh}{Wh}}
\AtBeginDocument{\DeclareSIUnit{\kWh}{kWh}}
\AtBeginDocument{\DeclareSIUnit{\MWh}{MWh}}
\AtBeginDocument{\DeclareSIUnit{\GWh}{GWh}}
\AtBeginDocument{\DeclareSIUnit{\ct}{ct}}
\AtBeginDocument{\DeclareSIUnit{\EUR}{EUR}}
\AtBeginDocument{\DeclareSIUnit{\impuls}{Impulse}}
\AtBeginDocument{\DeclareSIUnit{\year}{a}}
%\AtBeginDocument{\DeclareSIUnit{\hekto}{h}}
\AtBeginDocument{\DeclareSIUnit{\c}{c}}
\AtBeginDocument{\DeclareSIUnit{\e}{e}}
\AtBeginDocument{\DeclareSIUnit{\MeV}{MeV}}
%\AtBeginDocument{\DeclareSIUnit{\giga}{G}}
%\AtBeginDocument{\DeclareSIUnit{\mega}{M}}
%\AtBeginDocument{\DeclareSIUnit{\kilo}{k}}
%\AtBeginDocument{\DeclareSIUnit{\deca}{da}}
%\AtBeginDocument{\DeclareSIUnit{\deka}{da}}
%\AtBeginDocument{\DeclareSIUnit{\deci}{d}}
%\AtBeginDocument{\DeclareSIUnit{\centi}{c}}
%\AtBeginDocument{\DeclareSIUnit{\milli}{m}}
%\AtBeginDocument{\DeclareSIUnit{\micro}{\text{$\umu$}}}
%\AtBeginDocument{\DeclareSIUnit{\nano}{n}}









%Befehl für (leere) Koordinatensysteme: \kos{links}{unten}{rechts}{oben}. Beschriftet mit der Eins in jede Richtung und dem minimalen bzw. maximalen Wert.   
\newcommand{\kos}[4]{
	\clip (#1-0.5,#2-0.5) rectangle (#3+1,#4+1);
	\draw [color=gray!50]  [step=5mm] (#1-1,#2-1) grid (#3+1,#4+1); %Raster zeichnen
	\draw[->,thick] (#1-0.5,0) -- (#3+0.5,0) node[right] {$x$}; % x-Achse zeichnen
	\draw[->,thick] (0,#2-0.5) -- (0,#4+0.5) node[above] {$y$}; % y-Achse zeichnen
	\foreach \x in {#1,-1,1,#3} \draw (\x,-.1)--(\x,.1) node[below=4pt] {$\scriptstyle\x$};%x-Achse beschriften
	\foreach \y in {#2,-1,1,#4} \draw (-.1,\y)--(.1,\y) node[left=4pt] {$\scriptstyle\y$};   % y-Achse beschriften
	\draw[color=black] (-4pt,-6pt) node[] {$0$};     % schöner Ursprung
	\path[name path=rahmen,clip](#1-0.5,#2-0.5) rectangle (#3+0.5,#4+0.5);
}

%Lineares Gleichungssystem, 4 Argumente: 1. Zeile links, rechts, 2. Zeile links rechts:
\newcommand{\lgs}[4]{
	\begin{tabular}{|lrcl}
		&$#1$&=&$#2$\\$\land$&$#3$&=&$#4$\\\hline
	\end{tabular}}



\newcommand{\kommando}{\qquad\rule[-1ex]{0.5pt}{3ex}\hspace*{1ex}}
\usepackage{etoolbox}% for \ifblank{#3}{true}{false}
%\lings wie \lgs nur mit zwei zusätzlichen Argumenten für etwaige Äquivalenzumformungen
\newcommand{\lings}[6]{
	\begin{tabular}{rrcll}
				I	&$#1$\!\!\!&=&\!\!\!$#2$&$\ensuremath{\ifblank{#3}{}{\qquad\rule[-1ex]{0.5pt}{3ex}\hspace*{1ex} #3}}$\\[0.5ex] 
	$\wedge$	II	&$#4$\!\!\!&=&\!\!\!$#5$&$\ensuremath{\ifblank{#6}{}{\qquad\rule[-1ex]{0.5pt}{3ex}\hspace*{1ex} #6}}$\\\cline{1-4} 
	\end{tabular}
	}



\newcommand{\linievar}[1]{{\color{gray}{\hdashrule{#1}{.5pt}{3pt}}}}%Linien mit variabler Länge: \linie{länge mit Einheit}

%%kleinergleich und größergleich wie in der Realschule verlangt (nicht mehr im LP+!!!) (mit Doppelstrich):
%\renewcommand{\leq}{\leq}\renewcommand{\geq}{\geq}

\newcommand{\laenge}[1]{\ensuremath{\left|\overline{\text{#1}}\right|}}%Die Streckenlänge \laenge{}, muss mit dem neuen Lehrplan geändert werden!
\newcommand{\strecke}[1]{\ensuremath{\overline{\text{#1}}}}%Die Strecke \strecke{}, muss mit dem neuen Lehrplan geändert werden!
\newcommand{\gerade}[1]{#1}
%\newcommand{\winkel}[1]{\ensuremath{\sphericalangle \text{#1}}}
\newcommand{\winkel}[1]{\ensuremath{\varangle \text{#1}}}
\newcommand{\dreieck}[1]{\ensuremath{\bigtriangleup\text{#1}}}


%Optionaler Parameter bei \draw: extended line für Geraden und Co:
\tikzset{dot/.style={label={#1},name=#1},
	extended line/.style={shorten >=-#1,shorten <=-#1},
	extended line/.default=1cm]}

%Punkte mit Kreuzchen markiert:(funktioniert nur in Tikz-Picture-Umgebung!) zuerst name, dann x dann y, dann Position
\tikzset{point/.style={thick, draw=black,cross out, inner sep=0pt, minimum width=4pt, minimum height=4pt}}
\newcommand{\punkt}[4]{\coordinate[label=#4:$#1$, point](#1) at (#2,#3);}

\newcommand{\determinante}[4]{\begin{vmatrix} #1 & #2\\#3 & #4\\\end{vmatrix}}%Befehl für Determinanten, zeilenweise: zuerst die x-Werte, dann die y-Werte. Benötigt den Mathemodus

\newcommand{\vektor}[2]{\dbinom{#1}{#2}}%Befehl, um einen Vektor in der Spaltenschreibweise zu erstellen

%\newcommand{\entspricht}{\widehat{=}}%das entspricht-Zeichen

\newcommand\entspricht{\mathrel{\stackon[1.5pt]{=}{\stretchto{\scalerel*[\widthof{=}]{\wedge}{\rule{1ex}{3ex}}}{0.5ex}}}} %Quelle: https://tex.stackexchange.com/questions/103408/symbol-for-corresponds-to-equals-sign-with-hat%


%\newcommand{\platzbedarf}[4]{(Platzbedarf: $#1\leq x\leq #2$; $#3\leq y\leq 4$)}
\newcommand{\platzbedarf}[4]{Für die Zeichnung: \; Längeneinheit \SI{1}{\centi \meter};  \; $#1\leq x\leq #2$; \; $#3\leq y\leq #4$}

\usepackage[school]{pgf-umlcd}%Objektkarten mit dem package pgf-umlcd
\renewcommand{\umltextcolor}{black}%Textfarbe von Objektkarten:
\renewcommand{\umldrawcolor}{black}%Linienfarbe von Objektkarten:
\renewcommand{\umlfillcolor}{white}%Hintergrundfarbe von Objektkarten:

\newcommand{\kosy}[5]{
	\begin{center}
		\tikzstyle{background grid}=[draw, black!20,step=5mm,line width=1pt]
		\begin{tikzpicture}[scale=1.0,samples=400%
		%			,show background grid
		]
		
		%% Größe des KOSYs %%%%%%%%%%%%%%%%%%%%%%%%%%%%%%%%%%%%%%%%%%
		
		\pgfmathsetmacro{\xmin}{#1}
		\pgfmathsetmacro{\xmax}{#2}
		\pgfmathsetmacro{\ymin}{#3}
		\pgfmathsetmacro{\ymax}{#4}
		
		%%%%%%%%%%%%%%%%%%%%%%%%%%%%%%%%%%%%%%%%%%%%%%%%%%%%%%%%%%%%%
		
		\pgfmathtruncatemacro{\xmintick}{\xmin}
		\pgfmathtruncatemacro{\xmaxtick}{\xmax}
		\pgfmathtruncatemacro{\ymintick}{\ymin}
		\pgfmathtruncatemacro{\ymaxtick}{\ymax}
		
		\pgfmathtruncatemacro{\xmingrid}{\xmin-1}
		\pgfmathtruncatemacro{\xmaxgrid}{\xmax+1}
		\pgfmathtruncatemacro{\ymingrid}{\ymin-1}
		\pgfmathtruncatemacro{\ymaxgrid}{\ymax+1}
		
		\pgfmathsetmacro{\xminkosy}{\xmin-0.5}
		\pgfmathsetmacro{\xmaxkosy}{\xmax+0.5}
		\pgfmathsetmacro{\yminkosy}{\ymin-0.5}
		\pgfmathsetmacro{\ymaxkosy}{\ymax+0.5}
		
		% Definition des Gitters und das Gitter selbst
		
		\tikzstyle{dotted}= [dash pattern=on 0.15mm off 0.85mm]
		\draw[help lines,line width=0.15mm,step=0.5,style=dotted,black,line cap=round] (\xmingrid,\ymingrid) grid (\xmaxgrid,\ymaxgrid);
		
		% Ursprung
		
		\node[below left =5pt] (0,0) {O};
		
		% Beschriftung der Koordinatenachsen mit Ticks (muss vor den Achsen selbst kommen, sonst sind die Achsen z. T. abgedeckt)
		
		\foreach \x in {\xmintick,...,-1,1,2,...,\xmaxtick}
		{
			\draw node[anchor=north,fill=white, fill opacity=0.7] at (\x,-0.2) {\x};
			\draw (\x,0pt) -- (\x,-2pt) node[anchor=north] at (\x,-0.2) {\x};
		}		
		
		\foreach \y in {\ymintick,...,-1,1,2,...,\ymaxtick}
		{
			\draw node[anchor=east,fill=white, fill opacity=0.7] at (-0.1,\y) {\y};
			\draw (0pt,\y) -- (-2pt,\y) node[anchor=east] at (-0.1,\y) {\y};
		}
		
		% Koordinatenachsen und Beschriftung mit $x$ und $y$
		
		\draw[-latex,line width=0.2mm] (\xminkosy,0) --(\xmaxkosy,0) node[below right = 4pt]{$x$};   
		\draw[-latex,line width=0.2mm] (0,\yminkosy) --(0,\ymaxkosy) node[above left = 4pt]{$y$};
		
		#5
		
		\end{tikzpicture}
	\end{center}
}


%Braucht man für das kosycustomi:

\usepackage{fp}
\usepackage{pgfkeys,pgfmath,pgfcore}

\pgfkeys{
	/textnumber/.style={
		/pgf/number format/.cd,% <- changes the prefix for the following options
		fixed,
		use comma,
		fixed zerofill,
		precision=0, % Hier kann man die Anzahl der Nachkommastellen für das KOSY eintragen
		1000 sep={.},
	},
}

%Braucht man für das kosycustomii:

\pgfkeys{
	/textnumber2/.style={
		/pgf/number format/.cd,% <- changes the prefix for the following options
%		fixed,
		use comma,
		fixed zerofill,
		precision=2, % Hier kann man die Anzahl der Nachkommastellen für das KOSY eintragen
		1000 sep={.},
	},
}


\newcommand{\kosycustomi}[9]{
	\begin{center}
		\begin{tikzpicture}[scale=1,samples=400]
		
		%% Größe des KOSYs %%%%%%%%%%%%%%%%%%%%%%%%%%%%%%%%%%%%%%%%%%
		
		\pgfmathsetmacro{\xmin}{#1}
		\pgfmathsetmacro{\xmax}{#2}
		\pgfmathsetmacro{\ymin}{#3}
		\pgfmathsetmacro{\ymax}{#4}
		
		%%%%%%%%%%%%%%%%%%%%%%%%%%%%%%%%%%%%%%%%%%%%%%%%%%%%%%%%%%%%%
		
		\pgfmathtruncatemacro{\xmintick}{1+\xmin}
		\pgfmathtruncatemacro{\xmaxtick}{\xmax}
		\pgfmathtruncatemacro{\ymintick}{1+\ymin}
		\pgfmathtruncatemacro{\ymaxtick}{\ymax}
		
		\pgfmathtruncatemacro{\xmingrid}{\xmin-1}
		\pgfmathtruncatemacro{\xmaxgrid}{\xmax+2}
		\pgfmathtruncatemacro{\ymingrid}{\ymin-1}
		\pgfmathtruncatemacro{\ymaxgrid}{\ymax+1}
		
		\pgfmathsetmacro{\xminkosy}{\xmin-0.5}
		\pgfmathsetmacro{\xmaxkosy}{\xmax+0.5}
		\pgfmathsetmacro{\yminkosy}{\ymin-0.5}
		\pgfmathsetmacro{\ymaxkosy}{\ymax+0.5}
		
		% Definition des Gitters und das Gitter selbst
		
		\tikzstyle{dotted}= [dash pattern=on 0.15mm off 0.85mm]
		\draw[help lines,line width=0.15mm,step=0.5,style=dotted,black!20,line cap=round] (\xmingrid,\ymingrid) grid (\xmaxgrid,\ymaxgrid);
		
		% Ursprung
		
		\node[below left=2pt] (0,0) {\footnotesize 0};
		
		% Beschriftung der Koordinatenachsen mit Ticks (muss vor den Achsen selbst kommen, sonst sind die Achsen z. T. abgedeckt)
		
		\foreach \x in {\xmintick,...,1,2,3,...,\xmaxtick}
		{
			\pgfmathsetmacro\resultx{#7*\x}
			%			\draw node[anchor=north,fill=white, fill opacity=0.9] at (\x,0) {\footnotesize \resultx};
			\draw (\x,0pt) -- (\x,-2pt) node[anchor=north] at (\x,-0.1) {\footnotesize \pgfmathprintnumber[/textnumber]{\resultx}};
		}		
		
		\foreach \y in {\ymintick,...,1,2,3,...,\ymaxtick}
		{
			\pgfmathsetmacro\resulty{#8*(\y)}
			%			\draw node[anchor=east,fill=white, fill opacity=0.9] at (0,\y) {\footnotesize \resulty};
			\draw (0pt,\y) -- (-2pt,\y) node[anchor=east] at (-0.1,\y) {\footnotesize \pgfmathprintnumber[/textnumber]{\resulty}};
		}
		
		% Koordinatenachsen und Beschriftung mit $x$ und $y$
		
		\draw[-latex,line width=0.2mm] (\xminkosy,0) --(\xmaxkosy,0) node[right]{#5};   
		\draw[-latex,line width=0.2mm] (0,\yminkosy) --(0,\ymaxkosy) node[above]{#6};
		
		#9
		
		\end{tikzpicture}
	\end{center}
}	

\newcommand{\kosyquali}[7]{ % qualitatives Diagramm ohne Skalierung
	\begin{center}
		\begin{tikzpicture}[scale=0.707,samples=400]
		
		%% Größe des KOSYs %%%%%%%%%%%%%%%%%%%%%%%%%%%%%%%%%%%%%%%%%%
		
		\pgfmathsetmacro{\xmin}{#1}
		\pgfmathsetmacro{\xmax}{#2}
		\pgfmathsetmacro{\ymin}{#3}
		\pgfmathsetmacro{\ymax}{#4}
		
		%%%%%%%%%%%%%%%%%%%%%%%%%%%%%%%%%%%%%%%%%%%%%%%%%%%%%%%%%%%%%
		
		\pgfmathtruncatemacro{\xmintick}{1+\xmin}
		\pgfmathtruncatemacro{\xmaxtick}{\xmax}
		\pgfmathtruncatemacro{\ymintick}{1+\ymin}
		\pgfmathtruncatemacro{\ymaxtick}{\ymax}
		
		\pgfmathtruncatemacro{\xmingrid}{\xmin-1}
		\pgfmathtruncatemacro{\xmaxgrid}{\xmax+2}
		\pgfmathtruncatemacro{\ymingrid}{\ymin-1}
		\pgfmathtruncatemacro{\ymaxgrid}{\ymax+1}
		
		\pgfmathsetmacro{\xminkosy}{\xmin-0.5}
		\pgfmathsetmacro{\xmaxkosy}{\xmax+0.5}
		\pgfmathsetmacro{\yminkosy}{\ymin-0.5}
		\pgfmathsetmacro{\ymaxkosy}{\ymax+0.5}
		
		% Definition des Gitters und das Gitter selbst
		
		\tikzstyle{dotted}= [dash pattern=on 0.15mm off 0.85mm]
		\draw[help lines,line width=0.15mm,step=0.5,style=dotted,black!80,line cap=round] (\xmingrid,\ymingrid) grid (\xmaxgrid,\ymaxgrid);
		
%		% Ursprung
%		
%		\node[below left=2pt] (0,0) {\footnotesize 0};
%		
%		% Beschriftung der Koordinatenachsen mit Ticks (muss vor den Achsen selbst kommen, sonst sind die Achsen z. T. abgedeckt)
%		
%		\foreach \x in {\xmintick,...,1,2,3,...,\xmaxtick}
%		{
%			\pgfmathsetmacro\resultx{#7*\x}
%			%			\draw node[anchor=north,fill=white, fill opacity=0.9] at (\x,0) {\footnotesize \resultx};
%			\draw (\x,0pt) -- (\x,-2pt) node[anchor=north] at (\x,-0.1) {\footnotesize \pgfmathprintnumber[/textnumber]{\resultx}};
%		}		
%		
%		\foreach \y in {\ymintick,...,1,2,3,...,\ymaxtick}
%		{
%			\pgfmathsetmacro\resulty{#8*(\y)}
%			%			\draw node[anchor=east,fill=white, fill opacity=0.9] at (0,\y) {\footnotesize \resulty};
%			\draw (0pt,\y) -- (-2pt,\y) node[anchor=east] at (-0.1,\y) {\footnotesize \pgfmathprintnumber[/textnumber]{\resulty}};
%		}
		
		% Koordinatenachsen und Beschriftung mit $x$ und $y$
		
		\draw[-latex,line width=0.2mm] (\xminkosy,0) --(\xmaxkosy,0) node[right]{#5};   
		\draw[-latex,line width=0.2mm] (0,\yminkosy) --(0,\ymaxkosy) node[above, yshift=0.5ex]{#6};
			
		#7
						
		\end{tikzpicture}
	\end{center}
}	

\newcommand{\kosycustomii}[9]{
	\begin{center}
		\begin{tikzpicture}[scale=1,samples=400]
		
		%% Größe des KOSYs %%%%%%%%%%%%%%%%%%%%%%%%%%%%%%%%%%%%%%%%%%
		
		\pgfmathsetmacro{\xmin}{#1}
		\pgfmathsetmacro{\xmax}{#2}
		\pgfmathsetmacro{\ymin}{#3}
		\pgfmathsetmacro{\ymax}{#4}
		
		%%%%%%%%%%%%%%%%%%%%%%%%%%%%%%%%%%%%%%%%%%%%%%%%%%%%%%%%%%%%%
		
		\pgfmathtruncatemacro{\xmintick}{1+\xmin}
		\pgfmathtruncatemacro{\xmaxtick}{\xmax}
		\pgfmathtruncatemacro{\ymintick}{1+\ymin}
		\pgfmathtruncatemacro{\ymaxtick}{\ymax}
		
		\pgfmathtruncatemacro{\xmingrid}{\xmin-1}
		\pgfmathtruncatemacro{\xmaxgrid}{\xmax+1}
		\pgfmathtruncatemacro{\ymingrid}{\ymin-1}
		\pgfmathtruncatemacro{\ymaxgrid}{\ymax+1}
		
		\pgfmathsetmacro{\xminkosy}{\xmin-0.5}
		\pgfmathsetmacro{\xmaxkosy}{\xmax+0.5}
		\pgfmathsetmacro{\yminkosy}{\ymin-0.5}
		\pgfmathsetmacro{\ymaxkosy}{\ymax+0.5}
		
		% Definition des Gitters und das Gitter selbst
		
		\tikzstyle{dotted}= [dash pattern=on 0.15mm off 0.85mm]
		\draw[help lines,line width=0.15mm,step=0.5,style=dotted,black!20,line cap=round] (\xmingrid,\ymingrid) grid (\xmaxgrid,\ymaxgrid);
		
		% Ursprung
		
		\node[below left=2pt] (0,0) {\footnotesize 0};
		
		% Beschriftung der Koordinatenachsen mit Ticks (muss vor den Achsen selbst kommen, sonst sind die Achsen z. T. abgedeckt)
		
		\foreach \x in {\xmintick,...,1,2,3,...,\xmaxtick}
		{
			\pgfmathsetmacro\resultx{#7*\x}
			\draw (\x,0pt) -- (\x,-2pt) node[anchor=north] at (\x,-0.1) 
				{\footnotesize \num[round-mode=figures,round-precision=2]{\resultx}};
		}		
		
		\foreach \y in {\ymintick,...,1,2,3,...,\ymaxtick}
		{
			\pgfmathsetmacro\resulty{#8*\y}
			\draw (0pt,\y) -- (-2pt,\y) node[anchor=east] at (-0.1,\y) 
				{\footnotesize \num[round-mode=figures,round-precision=2]{\resulty}};
		}
		
		% Koordinatenachsen und Beschriftung mit $x$ und $y$
		
		\draw[-latex,line width=0.2mm] (\xminkosy,0) --(\xmaxkosy,0) node[left, xshift=-0mm, yshift=-8mm]{#5};   
		\draw[-latex,line width=0.2mm] (0,\yminkosy) --(0,\ymaxkosy) node[above, , yshift=1mm]{#6};
		
		#9
		
		\end{tikzpicture}
	\end{center}
}	

\newcommand{\kosycustomiii}[9]{
	\begin{center}
		\begin{tikzpicture}[scale=1,samples=400]
		
		%% Größe des KOSYs %%%%%%%%%%%%%%%%%%%%%%%%%%%%%%%%%%%%%%%%%%
		
		\pgfmathsetmacro{\xmin}{#1}
		\pgfmathsetmacro{\xmax}{#2}
		\pgfmathsetmacro{\ymin}{#3}
		\pgfmathsetmacro{\ymax}{#4}
		
		%%%%%%%%%%%%%%%%%%%%%%%%%%%%%%%%%%%%%%%%%%%%%%%%%%%%%%%%%%%%%
		
		\pgfmathtruncatemacro{\xmintick}{1+\xmin}
		\pgfmathtruncatemacro{\xmaxtick}{\xmax}
		\pgfmathtruncatemacro{\ymintick}{1+\ymin}
		\pgfmathtruncatemacro{\ymaxtick}{\ymax}
		
		\pgfmathtruncatemacro{\xmingrid}{\xmin-1}
		\pgfmathtruncatemacro{\xmaxgrid}{\xmax+1}
		\pgfmathtruncatemacro{\ymingrid}{\ymin-1}
		\pgfmathtruncatemacro{\ymaxgrid}{\ymax+1}
		
		\pgfmathsetmacro{\xminkosy}{\xmin-0.5}
		\pgfmathsetmacro{\xmaxkosy}{\xmax+0.5}
		\pgfmathsetmacro{\yminkosy}{\ymin-0.5}
		\pgfmathsetmacro{\ymaxkosy}{\ymax+0.5}
		
		% Definition des Gitters und das Gitter selbst
		
		\tikzstyle{dotted}= [dash pattern=on 0.15mm off 0.85mm]
		\draw[help lines,line width=0.15mm,step=0.5,style=dotted,black!20,line cap=round] (\xmingrid,\ymingrid) grid (\xmaxgrid,\ymaxgrid);
		
		% Ursprung
		
		\node[below left=2pt] (0,0) {\footnotesize 0};
		
		% Beschriftung der Koordinatenachsen mit Ticks (muss vor den Achsen selbst kommen, sonst sind die Achsen z. T. abgedeckt)
		
		\foreach \x in {\xmintick,...,1,2,3,...,\xmaxtick}
		{
			\pgfmathsetmacro\resultx{#7*\x}
			\draw (\x,0pt) -- (\x,-2pt) node[anchor=north] at (\x,-0.1) 
			{\footnotesize \num[round-mode=figures,round-precision=3]{\resultx}};
		}		
		
		\foreach \y in {\ymintick,...,1,2,3,...,\ymaxtick}
		{
			\pgfmathsetmacro\resulty{#8*\y}
			\draw (0pt,\y) -- (-2pt,\y) node[anchor=east] at (-0.1,\y) 
			{\footnotesize \num[round-mode=figures,round-precision=2]{\resulty}};
		}
		
		% Koordinatenachsen und Beschriftung mit $x$ und $y$
		
		\draw[-latex,line width=0.2mm] (\xminkosy,0) --(\xmaxkosy,0) node[left, xshift=-0mm, yshift=-8mm]{#5};   
		\draw[-latex,line width=0.2mm] (0,\yminkosy) --(0,\ymaxkosy) node[above]{#6};
		
		#9
		
		\end{tikzpicture}
	\end{center}
}	

\newcommand{\kosycustomiv}[9]{
	\begin{center}
		\begin{tikzpicture}[scale=1,samples=400]
		
		%% Größe des KOSYs %%%%%%%%%%%%%%%%%%%%%%%%%%%%%%%%%%%%%%%%%%
		
		\pgfmathsetmacro{\xmin}{#1}
		\pgfmathsetmacro{\xmax}{#2}
		\pgfmathsetmacro{\ymin}{#3}
		\pgfmathsetmacro{\ymax}{#4}
		
		%%%%%%%%%%%%%%%%%%%%%%%%%%%%%%%%%%%%%%%%%%%%%%%%%%%%%%%%%%%%%
		
		\pgfmathtruncatemacro{\xmintick}{1+\xmin}
		\pgfmathtruncatemacro{\xmaxtick}{\xmax}
		\pgfmathtruncatemacro{\ymintick}{1+\ymin}
		\pgfmathtruncatemacro{\ymaxtick}{\ymax}
		
		\pgfmathtruncatemacro{\xmingrid}{\xmin-1}
		\pgfmathtruncatemacro{\xmaxgrid}{\xmax+1}
		\pgfmathtruncatemacro{\ymingrid}{\ymin-1}
		\pgfmathtruncatemacro{\ymaxgrid}{\ymax+1}
		
		\pgfmathsetmacro{\xminkosy}{\xmin-0.5}
		\pgfmathsetmacro{\xmaxkosy}{\xmax+0.5}
		\pgfmathsetmacro{\yminkosy}{\ymin-0.5}
		\pgfmathsetmacro{\ymaxkosy}{\ymax+0.5}
		
		% Definition des Gitters und das Gitter selbst
		
		\tikzstyle{dotted}= [dash pattern=on 0.15mm off 0.85mm]
		\draw[help lines,line width=0.15mm,step=0.5,style=dotted,black!20,line cap=round] (\xmingrid,\ymingrid) grid (\xmaxgrid,\ymaxgrid);
		
		% Ursprung
		
		\node[below left=2pt] (0,0) {\footnotesize 0};
		
		% Beschriftung der Koordinatenachsen mit Ticks (muss vor den Achsen selbst kommen, sonst sind die Achsen z. T. abgedeckt)
		
		\foreach \x in {\xmintick,...,1,2,3,...,\xmaxtick}
		{
			\pgfmathsetmacro\resultx{#7*\x}
			\draw (\x,0pt) -- (\x,-2pt) node[anchor=north] at (\x,-0.1) 
			{\footnotesize \num[round-mode=places,round-precision=1]{\resultx}};
		}		
		
		\foreach \y in {\ymintick,...,1,2,3,...,\ymaxtick}
		{
			\pgfmathsetmacro\resulty{#8*\y}
			\draw (0pt,\y) -- (-2pt,\y) node[anchor=east] at (-0.1,\y) 
			{\footnotesize \num[round-mode=places,round-precision=1]{\resulty}};
		}
		
		% Koordinatenachsen und Beschriftung mit $x$ und $y$
		
		\draw[-latex,line width=0.2mm] (\xminkosy,0) --(\xmaxkosy,0) node[left, xshift=-0mm, yshift=-8mm]{#5};   
		\draw[-latex,line width=0.2mm] (0,\yminkosy) --(0,\ymaxkosy) node[above]{#6};
		
		#9
		
		\end{tikzpicture}
	\end{center}
}	

\newcommand{\kosycustomiiinkii}[9]{
	\begin{center}
		\begin{tikzpicture}[scale=1,samples=400]
		
		%% Größe des KOSYs %%%%%%%%%%%%%%%%%%%%%%%%%%%%%%%%%%%%%%%%%%
		
		\pgfmathsetmacro{\xmin}{#1}
		\pgfmathsetmacro{\xmax}{#2}
		\pgfmathsetmacro{\ymin}{#3}
		\pgfmathsetmacro{\ymax}{#4}
		
		%%%%%%%%%%%%%%%%%%%%%%%%%%%%%%%%%%%%%%%%%%%%%%%%%%%%%%%%%%%%%
		
		\pgfmathtruncatemacro{\xmintick}{1+\xmin}
		\pgfmathtruncatemacro{\xmaxtick}{\xmax}
		\pgfmathtruncatemacro{\ymintick}{1+\ymin}
		\pgfmathtruncatemacro{\ymaxtick}{\ymax}
		
		\pgfmathtruncatemacro{\xmingrid}{\xmin-1}
		\pgfmathtruncatemacro{\xmaxgrid}{\xmax+1}
		\pgfmathtruncatemacro{\ymingrid}{\ymin-1}
		\pgfmathtruncatemacro{\ymaxgrid}{\ymax+1}
		
		\pgfmathsetmacro{\xminkosy}{\xmin-0.5}
		\pgfmathsetmacro{\xmaxkosy}{\xmax+0.5}
		\pgfmathsetmacro{\yminkosy}{\ymin-0.5}
		\pgfmathsetmacro{\ymaxkosy}{\ymax+0.5}
		
		% Definition des Gitters und das Gitter selbst
		
		\tikzstyle{dotted}= [dash pattern=on 0.15mm off 0.85mm]
		\draw[help lines,line width=0.15mm,step=0.5,style=dotted,black!20,line cap=round] (\xmingrid,\ymingrid) grid (\xmaxgrid,\ymaxgrid);
		
		% Ursprung
		
		\node[below left=2pt] (0,0) {\footnotesize 0};
		
		% Beschriftung der Koordinatenachsen mit Ticks (muss vor den Achsen selbst kommen, sonst sind die Achsen z. T. abgedeckt)
		
		\foreach \x in {\xmintick,...,1,2,3,...,\xmaxtick}
		{
			\pgfmathsetmacro\resultx{#7*\x}
			\draw (\x,0pt) -- (\x,-2pt) node[anchor=north] at (\x,-0.1) 
			{\footnotesize \num[round-mode=places,round-precision=3]{\resultx}};
		}		
		
		\foreach \y in {\ymintick,...,1,2,3,...,\ymaxtick}
		{
			\pgfmathsetmacro\resulty{#8*\y}
			\draw (0pt,\y) -- (-2pt,\y) node[anchor=east] at (-0.1,\y) 
			{\footnotesize \num[round-mode=places,round-precision=2]{\resulty}};
		}
		
		% Koordinatenachsen und Beschriftung mit $x$ und $y$
		
		\draw[-latex,line width=0.2mm] (\xminkosy,0) --(\xmaxkosy,0) node[left, xshift=-0mm, yshift=-8mm]{#5};   
		\draw[-latex,line width=0.2mm] (0,\yminkosy) --(0,\ymaxkosy) node[above]{#6};
		
		#9
		
		\end{tikzpicture}
	\end{center}
}	
\newcommand{\kosycustomiink}[9]{
	\begin{center}
		\begin{tikzpicture}[scale=1,samples=400]
		
		%% Größe des KOSYs %%%%%%%%%%%%%%%%%%%%%%%%%%%%%%%%%%%%%%%%%%
		
		\pgfmathsetmacro{\xmin}{#1}
		\pgfmathsetmacro{\xmax}{#2}
		\pgfmathsetmacro{\ymin}{#3}
		\pgfmathsetmacro{\ymax}{#4}
		
		%%%%%%%%%%%%%%%%%%%%%%%%%%%%%%%%%%%%%%%%%%%%%%%%%%%%%%%%%%%%%
		
		\pgfmathtruncatemacro{\xmintick}{1+\xmin}
		\pgfmathtruncatemacro{\xmaxtick}{\xmax}
		\pgfmathtruncatemacro{\ymintick}{1+\ymin}
		\pgfmathtruncatemacro{\ymaxtick}{\ymax}
		
		\pgfmathtruncatemacro{\xmingrid}{\xmin-1}
		\pgfmathtruncatemacro{\xmaxgrid}{\xmax+1}
		\pgfmathtruncatemacro{\ymingrid}{\ymin-1}
		\pgfmathtruncatemacro{\ymaxgrid}{\ymax+1}
		
		\pgfmathsetmacro{\xminkosy}{\xmin-0.5}
		\pgfmathsetmacro{\xmaxkosy}{\xmax+0.5}
		\pgfmathsetmacro{\yminkosy}{\ymin-0.5}
		\pgfmathsetmacro{\ymaxkosy}{\ymax+0.5}
		
		% Definition des Gitters und das Gitter selbst
		
		\tikzstyle{dotted}= [dash pattern=on 0.15mm off 0.85mm]
		\draw[help lines,line width=0.15mm,step=0.5,style=dotted,black!20,line cap=round] (\xmingrid,\ymingrid) grid (\xmaxgrid,\ymaxgrid);
		
		% Ursprung
		
		\node[below left=2pt] (0,0) {\footnotesize 0};
		
		% Beschriftung der Koordinatenachsen mit Ticks (muss vor den Achsen selbst kommen, sonst sind die Achsen z. T. abgedeckt)
		
		\foreach \x in {\xmintick,...,1,2,3,...,\xmaxtick}
		{
			\pgfmathsetmacro\resultx{#7*\x}
			\draw (\x,0pt) -- (\x,-2pt) node[anchor=north] at (\x,-0.1) 
			{\footnotesize \num[round-mode=places,round-precision=2]{\resultx}};
		}		
		
		\foreach \y in {\ymintick,...,1,2,3,...,\ymaxtick}
		{
			\pgfmathsetmacro\resulty{#8*\y}
			\draw (0pt,\y) -- (-2pt,\y) node[anchor=east] at (-0.1,\y) 
			{\footnotesize \num[round-mode=places,round-precision=2]{\resulty}};
		}
		
		% Koordinatenachsen und Beschriftung mit $x$ und $y$
		
		\draw[-latex,line width=0.2mm] (\xminkosy,0) --(\xmaxkosy,0) node[left, xshift=-0mm, yshift=-3ex]{#5};   
		\draw[-latex,line width=0.2mm] (0,\yminkosy) --(0,\ymaxkosy) node[above]{#6};
		
		#9
		
		\end{tikzpicture}
	\end{center}
}	

\newcommand{\kosycustomdreizweink}[9]{
	\begin{center}
		\begin{tikzpicture}[scale=0.72,samples=400]
			
			%% Größe des KOSYs %%%%%%%%%%%%%%%%%%%%%%%%%%%%%%%%%%%%%%%%%%
			
			\pgfmathsetmacro{\xmin}{#1}
			\pgfmathsetmacro{\xmax}{#2}
			\pgfmathsetmacro{\ymin}{#3}
			\pgfmathsetmacro{\ymax}{#4}
			
			%%%%%%%%%%%%%%%%%%%%%%%%%%%%%%%%%%%%%%%%%%%%%%%%%%%%%%%%%%%%%
			
			\pgfmathtruncatemacro{\xmintick}{1+\xmin}
			\pgfmathtruncatemacro{\xmaxtick}{\xmax}
			\pgfmathtruncatemacro{\ymintick}{1+\ymin}
			\pgfmathtruncatemacro{\ymaxtick}{\ymax}
			
			\pgfmathtruncatemacro{\xmingrid}{\xmin-1}
			\pgfmathtruncatemacro{\xmaxgrid}{\xmax+1}
			\pgfmathtruncatemacro{\ymingrid}{\ymin-1}
			\pgfmathtruncatemacro{\ymaxgrid}{\ymax+1}
			
			\pgfmathsetmacro{\xminkosy}{\xmin-0.5}
			\pgfmathsetmacro{\xmaxkosy}{\xmax+0.5}
			\pgfmathsetmacro{\yminkosy}{\ymin-0.5}
			\pgfmathsetmacro{\ymaxkosy}{\ymax+0.5}
			
			% Definition des Gitters und das Gitter selbst
			
			\tikzstyle{dotted}= [dash pattern=on 0.15mm off 0.85mm]
			\draw[help lines,line width=0.15mm,step=0.5,style=dotted,black!20,line cap=round] (\xmingrid,\ymingrid) grid (\xmaxgrid,\ymaxgrid);
			
			% Ursprung
			
			\node[below left=2pt] (0,0) {\footnotesize 0};
			
			% Beschriftung der Koordinatenachsen mit Ticks (muss vor den Achsen selbst kommen, sonst sind die Achsen z. T. abgedeckt)
			
			\foreach \x in {\xmintick,...,1,2,3,...,\xmaxtick}
			{
				\pgfmathsetmacro\resultx{#7*\x}
				\draw (\x,0pt) -- (\x,-2pt) node[anchor=north] at (\x,-0.1) 
				{\scriptsize \num[round-mode=places,round-precision=3]{\resultx}};
			}		
			
			\foreach \y in {\ymintick,...,1,2,3,...,\ymaxtick}
			{
				\pgfmathsetmacro\resulty{#8*\y}
				\draw (0pt,\y) -- (-2pt,\y) node[anchor=east] at (-0.1,\y) 
				{\scriptsize \num[round-mode=places,round-precision=2]{\resulty}};
			}
			
			% Koordinatenachsen und Beschriftung mit $x$ und $y$
			
			\draw[-latex,line width=0.2mm] (\xminkosy,0) --(\xmaxkosy,0) node[left, xshift=-0mm, yshift=-3ex]{#5};   
			\draw[-latex,line width=0.2mm] (0,\yminkosy) --(0,\ymaxkosy) node[above]{#6};
			
			#9
			
		\end{tikzpicture}
	\end{center}
}	


\newcommand{\kosycustomdreieinsnk}[9]{
%	\begin{center}
		\begin{tikzpicture}[scale=0.72,samples=400]
			
			%% Größe des KOSYs %%%%%%%%%%%%%%%%%%%%%%%%%%%%%%%%%%%%%%%%%%
			
			\pgfmathsetmacro{\xmin}{#1}
			\pgfmathsetmacro{\xmax}{#2}
			\pgfmathsetmacro{\ymin}{#3}
			\pgfmathsetmacro{\ymax}{#4}
			
			%%%%%%%%%%%%%%%%%%%%%%%%%%%%%%%%%%%%%%%%%%%%%%%%%%%%%%%%%%%%%
			
			\pgfmathtruncatemacro{\xmintick}{1+\xmin}
			\pgfmathtruncatemacro{\xmaxtick}{\xmax}
			\pgfmathtruncatemacro{\ymintick}{1+\ymin}
			\pgfmathtruncatemacro{\ymaxtick}{\ymax}
			
			\pgfmathtruncatemacro{\xmingrid}{\xmin-1}
			\pgfmathtruncatemacro{\xmaxgrid}{\xmax+1}
			\pgfmathtruncatemacro{\ymingrid}{\ymin-1}
			\pgfmathtruncatemacro{\ymaxgrid}{\ymax+1}
			
			\pgfmathsetmacro{\xminkosy}{\xmin-0.5}
			\pgfmathsetmacro{\xmaxkosy}{\xmax+0.5}
			\pgfmathsetmacro{\yminkosy}{\ymin-0.5}
			\pgfmathsetmacro{\ymaxkosy}{\ymax+0.5}
			
			% Definition des Gitters und das Gitter selbst
			
			\tikzstyle{dotted}= [dash pattern=on 0.15mm off 0.85mm]
			\draw[help lines,line width=0.15mm,step=0.5,style=dotted,black!20,line cap=round] (\xmingrid,\ymingrid) grid (\xmaxgrid,\ymaxgrid);
			
			% Ursprung
			
			\node[below left=2pt] (0,0) {\footnotesize 0};
			
			% Beschriftung der Koordinatenachsen mit Ticks (muss vor den Achsen selbst kommen, sonst sind die Achsen z. T. abgedeckt)
			
			\foreach \x in {\xmintick,...,1,2,3,...,\xmaxtick}
			{
				\pgfmathsetmacro\resultx{#7*\x}
				\draw (\x,0pt) -- (\x,-2pt) node[anchor=north] at (\x,-0.1) 
				{\scriptsize \num[round-mode=places,round-precision=3]{\resultx}};
			}		
			
			\foreach \y in {\ymintick,...,1,2,3,...,\ymaxtick}
			{
				\pgfmathsetmacro\resulty{#8*\y}
				\draw (0pt,\y) -- (-2pt,\y) node[anchor=east] at (-0.1,\y) 
				{\scriptsize \num[round-mode=places,round-precision=1]{\resulty}};
			}
			
			% Koordinatenachsen und Beschriftung mit $x$ und $y$
			
			\draw[-latex,line width=0.2mm] (\xminkosy,0) --(\xmaxkosy,0) node[left, xshift=-0mm, yshift=-3ex]{#5};   
			\draw[-latex,line width=0.2mm] (0,\yminkosy) --(0,\ymaxkosy) node[above]{#6};
			
			#9
			
		\end{tikzpicture}
%	\end{center}
}	

\newcommand{\kosycustomdreidreink}[9]{
	%	\begin{center}
		\begin{tikzpicture}[scale=0.72,samples=400]
			
			%% Größe des KOSYs %%%%%%%%%%%%%%%%%%%%%%%%%%%%%%%%%%%%%%%%%%
			
			\pgfmathsetmacro{\xmin}{#1}
			\pgfmathsetmacro{\xmax}{#2}
			\pgfmathsetmacro{\ymin}{#3}
			\pgfmathsetmacro{\ymax}{#4}
			
			%%%%%%%%%%%%%%%%%%%%%%%%%%%%%%%%%%%%%%%%%%%%%%%%%%%%%%%%%%%%%
			
			\pgfmathtruncatemacro{\xmintick}{1+\xmin}
			\pgfmathtruncatemacro{\xmaxtick}{\xmax}
			\pgfmathtruncatemacro{\ymintick}{1+\ymin}
			\pgfmathtruncatemacro{\ymaxtick}{\ymax}
			
			\pgfmathtruncatemacro{\xmingrid}{\xmin-1}
			\pgfmathtruncatemacro{\xmaxgrid}{\xmax+1}
			\pgfmathtruncatemacro{\ymingrid}{\ymin-1}
			\pgfmathtruncatemacro{\ymaxgrid}{\ymax+1}
			
			\pgfmathsetmacro{\xminkosy}{\xmin-0.5}
			\pgfmathsetmacro{\xmaxkosy}{\xmax+0.5}
			\pgfmathsetmacro{\yminkosy}{\ymin-0.5}
			\pgfmathsetmacro{\ymaxkosy}{\ymax+0.5}
			
			% Definition des Gitters und das Gitter selbst
			
			\tikzstyle{dotted}= [dash pattern=on 0.15mm off 0.85mm]
			\draw[help lines,line width=0.15mm,step=0.5,style=dotted,black!20,line cap=round] (\xmingrid,\ymingrid) grid (\xmaxgrid,\ymaxgrid);
			
			% Ursprung
			
			\node[below left=2pt] (0,0) {\footnotesize 0};
			
			% Beschriftung der Koordinatenachsen mit Ticks (muss vor den Achsen selbst kommen, sonst sind die Achsen z. T. abgedeckt)
			
			\foreach \x in {\xmintick,...,1,2,3,...,\xmaxtick}
			{
				\pgfmathsetmacro\resultx{#7*\x}
				\draw (\x,0pt) -- (\x,-2pt) node[anchor=north] at (\x,-0.1) 
				{\scriptsize \num[round-mode=places,round-precision=3]{\resultx}};
			}		
			
			\foreach \y in {\ymintick,...,1,2,3,...,\ymaxtick}
			{
				\pgfmathsetmacro\resulty{#8*\y}
				\draw (0pt,\y) -- (-2pt,\y) node[anchor=east] at (-0.1,\y) 
				{\scriptsize \num[round-mode=places,round-precision=3]{\resulty}};
			}
			
			% Koordinatenachsen und Beschriftung mit $x$ und $y$
			
			\draw[-latex,line width=0.2mm] (\xminkosy,0) --(\xmaxkosy,0) node[left, xshift=-0mm, yshift=-3ex]{#5};   
			\draw[-latex,line width=0.2mm] (0,\yminkosy) --(0,\ymaxkosy) node[above]{#6};
			
			#9
			
		\end{tikzpicture}
		%	\end{center}
}	

\newcommand{\kosycustomzweizweink}[9]{
	%	\begin{center}
	\begin{tikzpicture}[scale=1.00,samples=400]
	
	%% Größe des KOSYs %%%%%%%%%%%%%%%%%%%%%%%%%%%%%%%%%%%%%%%%%%
	
	\pgfmathsetmacro{\xmin}{#1}
	\pgfmathsetmacro{\xmax}{#2}
	\pgfmathsetmacro{\ymin}{#3}
	\pgfmathsetmacro{\ymax}{#4}
	
	%%%%%%%%%%%%%%%%%%%%%%%%%%%%%%%%%%%%%%%%%%%%%%%%%%%%%%%%%%%%%
	
	\pgfmathtruncatemacro{\xmintick}{1+\xmin}
	\pgfmathtruncatemacro{\xmaxtick}{\xmax}
	\pgfmathtruncatemacro{\ymintick}{1+\ymin}
	\pgfmathtruncatemacro{\ymaxtick}{\ymax}
	
	\pgfmathtruncatemacro{\xmingrid}{\xmin-1}
	\pgfmathtruncatemacro{\xmaxgrid}{\xmax+1}
	\pgfmathtruncatemacro{\ymingrid}{\ymin-1}
	\pgfmathtruncatemacro{\ymaxgrid}{\ymax+1}
	
	\pgfmathsetmacro{\xminkosy}{\xmin-0.5}
	\pgfmathsetmacro{\xmaxkosy}{\xmax+0.5}
	\pgfmathsetmacro{\yminkosy}{\ymin-0.5}
	\pgfmathsetmacro{\ymaxkosy}{\ymax+0.5}
	
	% Definition des Gitters und das Gitter selbst
	
	\tikzstyle{dotted}= [dash pattern=on 0.15mm off 0.85mm]
	\draw[help lines,line width=0.15mm,step=0.5,style=dotted,black!20,line cap=round] (\xmingrid,\ymingrid) grid (\xmaxgrid,\ymaxgrid);
	
	% Ursprung
	
	\node[below left=2pt] (0,0) {\footnotesize 0};
	
	% Beschriftung der Koordinatenachsen mit Ticks (muss vor den Achsen selbst kommen, sonst sind die Achsen z. T. abgedeckt)
	
	\foreach \x in {\xmintick,...,1,2,3,...,\xmaxtick}
	{
		\pgfmathsetmacro\resultx{#7*\x}
		\draw (\x,0pt) -- (\x,-2pt) node[anchor=north] at (\x,-0.1) 
		{\scriptsize \num[round-mode=places,round-precision=2]{\resultx}};
	}		
	
	\foreach \y in {\ymintick,...,1,2,3,...,\ymaxtick}
	{
		\pgfmathsetmacro\resulty{#8*\y}
		\draw (0pt,\y) -- (-2pt,\y) node[anchor=east] at (-0.1,\y) 
		{\scriptsize \num[round-mode=places,round-precision=2]{\resulty}};
	}
	
	% Koordinatenachsen und Beschriftung mit $x$ und $y$
	
	\draw[-latex,line width=0.2mm] (\xminkosy,0) --(\xmaxkosy,0) node[left, xshift=-0mm, yshift=-3ex]{#5};   
	\draw[-latex,line width=0.2mm] (0,\yminkosy) --(0,\ymaxkosy) node[above]{#6};
	
	#9
	
	\end{tikzpicture}
	%	\end{center}
}	

\newcommand{\kosycustomscale}[9]{
	%	\begin{center}
	\begin{tikzpicture}[scale=#9,samples=400]
	
	%% Größe des KOSYs %%%%%%%%%%%%%%%%%%%%%%%%%%%%%%%%%%%%%%%%%%
	
	\pgfmathsetmacro{\xmin}{#1}
	\pgfmathsetmacro{\xmax}{#2}
	\pgfmathsetmacro{\ymin}{#3}
	\pgfmathsetmacro{\ymax}{#4}
	
	%%%%%%%%%%%%%%%%%%%%%%%%%%%%%%%%%%%%%%%%%%%%%%%%%%%%%%%%%%%%%
	
	\pgfmathtruncatemacro{\xmintick}{1+\xmin}
	\pgfmathtruncatemacro{\xmaxtick}{\xmax}
	\pgfmathtruncatemacro{\ymintick}{1+\ymin}
	\pgfmathtruncatemacro{\ymaxtick}{\ymax}
	
	\pgfmathtruncatemacro{\xmingrid}{\xmin-1}
	\pgfmathtruncatemacro{\xmaxgrid}{\xmax+1}
	\pgfmathtruncatemacro{\ymingrid}{\ymin-1}
	\pgfmathtruncatemacro{\ymaxgrid}{\ymax+1}
	
	\pgfmathsetmacro{\xminkosy}{\xmin-0.5}
	\pgfmathsetmacro{\xmaxkosy}{\xmax+0.5}
	\pgfmathsetmacro{\yminkosy}{\ymin-0.5}
	\pgfmathsetmacro{\ymaxkosy}{\ymax+0.5}
	
	% Definition des Gitters und das Gitter selbst
	
	\tikzstyle{dotted}= [dash pattern=on 0.15mm off 0.85mm]
	\draw[help lines,line width=0.15mm,step=0.5,style=dotted,black!20,line cap=round] (\xmingrid,\ymingrid) grid (\xmaxgrid,\ymaxgrid);
	
	% Ursprung
	
	\node[below left=2pt] (0,0) {\footnotesize 0};
	
	% Beschriftung der Koordinatenachsen mit Ticks (muss vor den Achsen selbst kommen, sonst sind die Achsen z. T. abgedeckt)
	
	\foreach \x in {\xmintick,...,1,2,3,...,\xmaxtick}
	{
		\pgfmathsetmacro\resultx{#7*\x}
		\draw (\x,0pt) -- (\x,-2pt) node[anchor=north] at (\x,-0.1) 
		{\scriptsize \num[round-mode=places,round-precision=0]{\resultx}};
	}		
	
	\foreach \y in {\ymintick,...,1,2,3,...,\ymaxtick}
	{
		\pgfmathsetmacro\resulty{#8*\y}
		\draw (0pt,\y) -- (-2pt,\y) node[anchor=east] at (-0.1,\y) 
		{\scriptsize \num[round-mode=places,round-precision=0]{\resulty}};
	}
	
	% Koordinatenachsen und Beschriftung mit $x$ und $y$
	
	\draw[-latex,line width=0.2mm] (\xminkosy,0) --(\xmaxkosy,0) node[left, xshift=-0mm, yshift=-3ex]{#5};   
	\draw[-latex,line width=0.2mm] (0,\yminkosy) --(0,\ymaxkosy) node[above]{#6};
		
	\end{tikzpicture}
	%	\end{center}
}	

\newcommand{\kosycustommech}[9]{
	%	\begin{center}
	\begin{tikzpicture}[scale=1,samples=400]
	
	%% Größe des KOSYs %%%%%%%%%%%%%%%%%%%%%%%%%%%%%%%%%%%%%%%%%%
	
	\pgfmathsetmacro{\xmin}{#1}
	\pgfmathsetmacro{\xmax}{#2}
	\pgfmathsetmacro{\ymin}{#3}
	\pgfmathsetmacro{\ymax}{#4}
	
	%%%%%%%%%%%%%%%%%%%%%%%%%%%%%%%%%%%%%%%%%%%%%%%%%%%%%%%%%%%%%
	
	\pgfmathtruncatemacro{\xmintick}{1+\xmin}
	\pgfmathtruncatemacro{\xmaxtick}{\xmax}
	\pgfmathtruncatemacro{\ymintick}{1+\ymin}
	\pgfmathtruncatemacro{\ymaxtick}{\ymax}
	
	\pgfmathtruncatemacro{\xmingrid}{\xmin-1}
	\pgfmathtruncatemacro{\xmaxgrid}{\xmax+1}
	\pgfmathtruncatemacro{\ymingrid}{\ymin-1}
	\pgfmathtruncatemacro{\ymaxgrid}{\ymax+1}
	
	\pgfmathsetmacro{\xminkosy}{\xmin-0.5}
	\pgfmathsetmacro{\xmaxkosy}{\xmax+0.5}
	\pgfmathsetmacro{\yminkosy}{\ymin-0.5}
	\pgfmathsetmacro{\ymaxkosy}{\ymax+0.5}
	
	% Definition des Gitters und das Gitter selbst
	
	\tikzstyle{dotted}= [dash pattern=on 0.15mm off 0.85mm]
	\draw[help lines,line width=0.15mm,step=0.5,style=dotted,black!20,line cap=round] (\xmingrid,\ymingrid) grid (\xmaxgrid,\ymaxgrid);
	
	% Ursprung
	
	\node[below left=3pt] (0,0) {\footnotesize 0};
	
	% Beschriftung der Koordinatenachsen mit Ticks (muss vor den Achsen selbst kommen, sonst sind die Achsen z. T. abgedeckt)
	
	\foreach \x in {\xmintick,...,1,2,3,...,\xmaxtick}
	{
		\pgfmathsetmacro\resultx{#7*\x}
		\draw (\x,0pt) -- (\x,-2pt) node[anchor=north] at (\x,-0.1) 
		{\scriptsize \num[round-mode=places,round-precision=#9]{\resultx}};
	}		
	
	\foreach \y in {\ymintick,...,1,2,3,...,\ymaxtick}
	{
		\pgfmathsetmacro\resulty{#8*\y}
		\draw (0pt,\y) -- (-2pt,\y) node[anchor=east] at (-0.1,\y) 
		{\scriptsize \num[round-mode=places,round-precision=#9]{\resulty}};
	}
	
	% Koordinatenachsen und Beschriftung mit $x$ und $y$
	
	\draw[-latex,line width=0.2mm] (\xminkosy,0) --(\xmaxkosy,0) node[xshift=-0mm, yshift=-4.5ex]{#5};   
	\draw[-latex,line width=0.2mm] (0,\yminkosy) --(0,\ymaxkosy) node[yshift=1.5ex]{#6};
	
	\end{tikzpicture}
	%	\end{center}
}	

\newcommand{\kosycustommechleer}[9]{
	%	\begin{center}
	\begin{tikzpicture}[scale=1,samples=400]
	
	%% Größe des KOSYs %%%%%%%%%%%%%%%%%%%%%%%%%%%%%%%%%%%%%%%%%%
	
	\pgfmathsetmacro{\xmin}{#1}
	\pgfmathsetmacro{\xmax}{#2}
	\pgfmathsetmacro{\ymin}{#3}
	\pgfmathsetmacro{\ymax}{#4}
	
	%%%%%%%%%%%%%%%%%%%%%%%%%%%%%%%%%%%%%%%%%%%%%%%%%%%%%%%%%%%%%
	
	\pgfmathtruncatemacro{\xmintick}{1+\xmin}
	\pgfmathtruncatemacro{\xmaxtick}{\xmax}
	\pgfmathtruncatemacro{\ymintick}{1+\ymin}
	\pgfmathtruncatemacro{\ymaxtick}{\ymax}
	
	\pgfmathtruncatemacro{\xmingrid}{\xmin-1}
	\pgfmathtruncatemacro{\xmaxgrid}{\xmax+1}
	\pgfmathtruncatemacro{\ymingrid}{\ymin-1}
	\pgfmathtruncatemacro{\ymaxgrid}{\ymax+1}
	
	\pgfmathsetmacro{\xminkosy}{\xmin-0.5}
	\pgfmathsetmacro{\xmaxkosy}{\xmax+0.5}
	\pgfmathsetmacro{\yminkosy}{\ymin-0.5}
	\pgfmathsetmacro{\ymaxkosy}{\ymax+0.5}
	
	% Definition des Gitters und das Gitter selbst
	
	\tikzstyle{dotted}= [dash pattern=on 0.15mm off 0.85mm]
	\draw[help lines,line width=0.15mm,step=0.5,style=dotted,black!20,line cap=round] (\xmingrid,\ymingrid) grid (\xmaxgrid,\ymaxgrid);
	
	% Ursprung
	
	\node[below left=3pt] (0,0) {\footnotesize 0};
	
	% Beschriftung der Koordinatenachsen mit Ticks (muss vor den Achsen selbst kommen, sonst sind die Achsen z. T. abgedeckt)
	
	\foreach \x in {\xmintick,...,1,2,3,...,\xmaxtick}
	{
		\pgfmathsetmacro\resultx{#7*\x}
		\draw (\x,0pt) -- (\x,-2pt) node[anchor=north] at (\x,-0.1) 
		{\scriptsize \textcolor{white}{\num[round-mode=places,round-precision=#9]{\resultx}}};
	}		
	
	\foreach \y in {\ymintick,...,1,2,3,...,\ymaxtick}
	{
		\pgfmathsetmacro\resulty{#8*\y}
		\draw (0pt,\y) -- (-2pt,\y) node[anchor=east] at (-0.1,\y) 
		{\scriptsize \textcolor{white}{\num[round-mode=places,round-precision=#9]{\resulty}}};
	}
	
	% Koordinatenachsen und Beschriftung mit $x$ und $y$
	
	\draw[-latex,line width=0.2mm] (\xminkosy,0) --(\xmaxkosy,0) node[xshift=-1.5ex, yshift=-4.5ex]{#5};   
	\draw[-latex,line width=0.2mm] (0,\yminkosy) --(0,\ymaxkosy) node[yshift=1.5ex]{#6};
	
	\end{tikzpicture}
	%	\end{center}
}	


\newcommand{\D}{\displaystyle}

\usepackage[b]{esvect}	%https://www.ctan.org/pkg/esvect

%\usepackage{nccmath} %nccmath adds an optional argument to  \intertext, which is added to the vertical spacing of this command. Macht einen Fehler bei Lückentext-Formeln (werdn zu hoch gezogen...)

%\tikzset{circuit declare symbol = AC source}
%\tikzset{AC source IEC graphic/.style={
%		circuit symbol lines,
%		circuit symbol size=width 2 height 2,
%		shape=generic circle IEC,
%		/pgf/generic circle IEC/before background={
%			\pgfpathmoveto{\pgfpoint{-0.8pt}{0pt}}
%			\pgfpathsine{\pgfpoint{0.4pt}{0.4pt}}
%			\pgfpathcosine{\pgfpoint{0.4pt}{-0.4pt}}
%			\pgfpathsine{\pgfpoint{0.4pt}{-0.4pt}}
%			\pgfpathcosine{\pgfpoint{0.4pt}{0.4pt}}
%			\pgfusepath{stroke}
%		},
%		transform shape, draw
%	}
%}
%\tikzset{circuit ee IEC/.append style=
%	{set AC source graphic = AC source IEC graphic}
%}

% für dicke Bruchstriche:
\newcommand{\thickfrac}[2]{\genfrac{}{}{1pt}{}{#1}{#2}}
%%%%%%%%%%%%%%%%%%%%%%%%%%%%%%%%
%%%%% Anpassungen für exam %%%%%
%%%%%%%%%%%%%%%%%%%%%%%%%%%%%%%%
\usepackage{anyfontsize}
%\usepackage{color, colortbl}							% leider zerlegt es da meine tabularx-Umgebung...
\usepackage{xcolor}										% Colour control for LATEX documents
%\shadedsolutions
\renewcommand{\solutiontitle}{\noindent}
\SolutionEmphasis{\color{blue}}
\colorgrids
\definecolor{GridColor}{gray}{0.7}
\setlength{\gridsize}{5.03mm}
\setlength{\gridlinewidth}{0.3mm}
\setlength\dottedlinefillheight{5mm}
\colorfillwithdottedlines
\definecolor{FillWithDottedLinesColor}{gray}{0.5}

\renewcommand{\questionlabel}{\textcolor{white}{\arabic{question}.0}}
\renewcommand{\partlabel}{\thequestion.\arabic{partno}}
\renewcommand{\subpartlabel}{\arabic{question}.\arabic{partno}.\arabic{subpart}}

\newcommand{\questionnull}{\renewcommand{\questionlabel}{\arabic{question}.0} \question}
\newcommand{\questionsolo}{\renewcommand{\questionlabel}{\arabic{question}\textcolor{white}{.0}} \question}
\newcommand{\questioneins}{\renewcommand{\questionlabel}{\textcolor{white}{\arabic{question}.0}} \question}

\renewcommand{\questionshook}{%
	\setlength{\labelsep}{4mm}%{20.5pt}
	\setlength{\leftmargin}{26pt}
}

%\renewcommand{\questionshook}{% 
%	\setlength{\leftmargin}{0pt}% 
%	\setlength{\labelwidth}{-\labelsep}%
%}

\renewcommand{\partshook}{%
	\setlength{\labelsep}{4mm}
	\setlength{\leftmargin}{0pt}
}

\renewcommand{\subpartshook}{%
	\setlength{\labelsep}{4mm}
	\setlength{\leftmargin}{0pt}
}



%\renewcommand{\choiceshook}{%
%	\setlength{\labelsep}{20.5pt}%Kopf
%	\usepackage{scrlayer-scrpage}
%	%\pagestyle{scrheadings}
%	\clearpairofpagestyles
%	\DeclareNewLayer[
%	foreground, oddpage,
%	align=br, hoffset=\paperwidth-10mm , voffset=20mm,%\paperheight,
%	width=10mm, height=20mm,
%	contents=
%	{{%
%			\setlength{\fboxsep}{0pt}%
%			\begin{tcolorbox}[
%				sharp corners, 
%				rounded corners=southeast, rounded corners=southwest,
%				colback=white, colframe=black,
%				arc=3mm,
%				height=20mm, width=5mm,
%				text width=5mm,
%				valign=bottom,
%				boxsep=2mm,left=0pt,right=0pt,top=0pt,bottom=0pt
%				]
%				\color{white}\centering{\pagemark}
%			\end{tcolorbox}
%	}}
%	]{pagenumberouterabove.odd}
%	\DeclareNewLayer[
%	clone=pagenumberouterabove.odd,
%	evenpage,
%	align=bl,
%	hoffset=15mm,
%	]{pagenumberouterabove.even}
%	
%	\addtokomafont{pagenumber}{\color{black}\bfseries}
%	\AddLayersToPageStyle{scrheadings}{pagenumberouterabove.odd,pagenumberouterabove.even}
%	\AddLayersToPageStyle{plain}{pagenumberouterabove.odd,pagenumberouterabove.even}
%	\setlength{\leftmargin}{26pt}
%}


\newcommand{\ansredcheck}{\ifprintanswers\ensuremath{\textcolor{red}{\checkmark}}\else\fi}
\newcommand{\redcheck}{\ensuremath{\textcolor{red}{\checkmark}}}

\pointsinrightmargin

\pointformat{\fontsize{10}{12}\selectfont (\quad\,\,\,$\mid$\,\,\,\textsf{\themarginpoints}\,\,\,)}

%%%%% \newcounter %%%%%
\newcounter{nullpunkte}
\newcounter{punkte}[nullpunkte]
\newcounter{aufg}\setcounter{aufg}{0}
%%%%% \newcounter %%%%%

%%%%% newfont %%%%%
\newfont{\calligraphic}{callig15 scaled 1200}
%%%%% newfont %%%%%

%%%%%%%%%%%%%%%%%%%%%%%%%%%%%%%%
%%%%% Anpassungen für exam %%%%%
%%%%%%%%%%%%%%%%%%%%%%%%%%%%%%%%



\newcommand{\ansnp}{%
	\ifprintanswers
	\newpage
	\else
	\fi%
	}

\newcommand{\noansnp}{%
	\ifprintanswers
	\else 
	\newpage
	\fi%
}

\newcommand{\rueckseite}{
\ifprintanswers	
\else
%	\vfill
	
	\begin{flushright}
		Auf der Rückseite geht es weiter!
	\end{flushright}
	
	\newpage
	\thispagestyle{empty} \vspace*{-2em}
\fi
}

\newcommand{\rueckseitesofort}{
%	\vfill
\ifprintanswers	
\else
	\begin{flushright}
		Auf der Rückseite geht es weiter!
	\end{flushright}
	
	\newpage
	\thispagestyle{empty} \vspace*{-2em}
\fi
}

\newcommand{\rueckseitenurangabe}{
	%	\vfill
	\ifprintanswers
	\else
		\begin{flushright}
			Auf der Rückseite geht es weiter!
		\end{flushright}
		
		\newpage
		\thispagestyle{empty} \vspace*{-2em}
	\fi	
}

\newcommand{\naechsteseite}{
	%\vfill
	
	\begin{flushright}
		Auf der nächsten Seite geht es weiter!
	\end{flushright}
	
	\newpage
	\thispagestyle{empty} \vspace*{-2em}}

\newcommand{\naechsteseitenurangabe}{
	%	\vfill
	\ifprintanswers
	\else
	\begin{flushright}
		Auf der nächsten Seite geht es weiter!
	\end{flushright}
	
	\newpage
	\thispagestyle{empty} \vspace*{-2em}
	\fi	
}

\newcommand{\naechstesblatt}{
	%\vfill
	
	\begin{flushright}
		Auf dem nächsten Blatt geht es weiter!
	\end{flushright}
	
	\newpage
	\thispagestyle{empty} \vspace*{-2em}}

\newcommand{\umblaetternnurangabe}{
	%\vfill
	\ifprintanswers
	\else
		\begin{flushright}
			Bitte einmal umblättern!
		\end{flushright}
		
		\newpage
		\thispagestyle{empty} \vspace*{-2em}
	\fi	
}


\newcommand{\ipa}[1]{\ifprintanswers \textcolor{blue}{\;#1} \else \fi}

\newcommand{\spa}{\ifprintanswers \else \\[-2ex] \fi} % Platz vor solutionorgrid

\newcommand{\radier}[1]{\ifprintanswers \textcolor{blue}{#1\!\!} \else \phantom{#1} \fi}

%%%%% \setlength %%%%%
\setlength{\parindent}{0em}       % kein Erstzeileneinzug
\setlength{\textheight}{282mm} 
\setlength{\textwidth}{167mm}
\setlength{\oddsidemargin}{-15mm} 
\setlength{\evensidemargin}{-15mm} 
\setlength{\topmargin}{-28mm}
\setlength{\marginparsep}{8mm}
\setlength{\rightpointsmargin}{3mm}
%%%%% \setlength %%%%%

\usepackage{marginnote}


%Die Gesamtpunkte immer im rechten Rand! Quelle: https://tex.stackexchange.com/questions/69595/marginnote-always-on-right-side-of-the-page
\makeatletter
\patchcmd{\@addmarginpar}{\ifodd\c@page}{\ifodd\c@page\@tempcnta\m@ne}{}{}
\makeatother
\reversemarginpar

%\usepackage{geometry}
\geometry{%
	a4paper,
%	verbose, 
	tmargin=9mm,
	bmargin=18mm,
	lmargin=24mm,
	rmargin=24mm,
	footskip=3mm, % Hiermit wird der vertikale Abstand zwischen der untersten Fußnote und der horizontalen Linie am unteren Ende der Seite angepasst.
	marginpar=0cm, % <====================================================
%		showframe % <= Anzeige des Rahmens auf der Seite - nur zur Kontrolle einschalten
}

% Anpassung der Fußnoten - siehe auch Anpassungen über geometry in Praeambel_AH.tex
\usepackage[hang, flushmargin, bottom]{footmisc}
\addtolength{\footnotesep}{1mm}

\extrafootheight{-6mm}

%\pagestyle{empty}  %auskommentiert, denn sonst wird die Fußzeile nicht angezeigt...

%%%%% \setlength %%%%%
\setlength{\parindent}{0em}       % kein Erstzeileneinzug
%\setlength{\textheight}{282mm} 
%\setlength{\textwidth}{167mm}
%\setlength{\oddsidemargin}{0mm} 
%\setlength{\evensidemargin}{-8mm} 
%\setlength{\topmargin}{-25mm}
%\setlength{\marginparsep}{8mm}
%\setlength{\rightpointsmargin}{3mm}
%%%%% \setlength %%%%%

%Lückentext: \luecke{Text, der in der Lücke steht} und eine Umgebung lueckentext, die dafür sorgt, dass richtig umgebrochen und der Zeilenabstand hochgesetzt wird.  (damit die Schüler Platz zum Schreiben haben)

\usepackage{setspace}
\newenvironment{lueckentext}
{\raggedright\begin{spacing}{1.7}}
	{\end{spacing}}
\newlength{\diebox}
\newcommand{\luecke}[1]{
	\settowidth{\diebox}{#1}
	\ifprintanswers
	\raisebox{0.1ex}{\parbox{2.3\diebox}{\textbf{\textcolor{magenta!100!black}{\rule[-2ex]{0pt}{5ex}#1}}}}
	\else
	%	\raisebox{-0.5ex}{\parbox{2.3\diebox}{\hrulefill}	}
	\raisebox{-0.5ex}{\parbox{2.3\diebox}{\rule[-1ex]{0pt}{5ex}\begin{tikzpicture}   
			\draw[color=gray, text=gray, dashed](0 ,0)--(\linewidth,0 );
			\end{tikzpicture}}	}
	\fi}  



%Umgebung für 1., 2., 3. (Aufgaben und Beispiele):
\newenvironment{aufzaehlung}
{\begin{enumerate}}
	{\end{enumerate}}

%Befehl, um Beispiele, Aufgaben und Herleitungen (Miniüberschriften) hervorzuheben:
\newcommand{\ueber}[1]{\par\medskip\textbf{#1}\par\smallskip}

%Befehl, um Latex beizubringen, was er beim \liniert{}-Befehl tun soll:
\newcommand{\liniert}[1]
{\par\noindent
	\begin{tikzpicture}   
	\clip (0,.85) rectangle(\linewidth, 0.9*(#1+1);
	\foreach \n in {1,...,#1}{\draw[color=gray, text=gray, dashed](0 ,0.9*\n )--(\linewidth,0.9*\n );}
	\end{tikzpicture}}

%Befehl, um Latex beizubringen, was er beim \kariert{}-Befehl tun soll:
\newcommand{\kariert}[1]{
	\par\noindent
	\begin{tikzpicture} 
	\clip (0,-0.05) rectangle (\linewidth-3pt,#1/2) ;   
	\draw[step=0.5,color=gray, dashed] (0,0) grid (\linewidth,#1/2); %Anzahl der Kästchenreihen
	\end{tikzpicture}}

%Befehl, um Latex beizubringen, was er beim \kariertprog{}-Befehl tun soll: (Karopapier mit Einrückungshilfe)
\newcommand{\kariertprog}[1]{
	\par\noindent
	\begin{tikzpicture} 
	\clip (0,-0.05) rectangle (\linewidth,#1/2) ;   
	\draw[step=0.5,color=gray, dashed] (0,0) grid (\linewidth,#1/2); %Anzahl der Kästchenreihen
	%\draw[step=1.5,color=gray] (0,0) grid (\linewidth,#1/2);
	\draw[gray] (1.5,0)--(1.5,#1/2);
	\draw[gray] (3,0)--(3,#1/2);
	\draw[gray] (4.5,0)--(4.5,#1/2);
	\draw[gray] (6,0)--(6,#1/2);
	\end{tikzpicture}}

%Befehl, um Latex beizubringen was er beim \platz{}-Befehl tun soll. Benutzen, um einen Kasten um Freiplatz für Aufgaben zu ziehen. Angabe der Höhe in Zentimeter.
\newcommand{\platz}[1]{
	\par\medskip\noindent
	\begin{tikzpicture}       
	\draw[dashed, color=gray] (0,0) -- (\linewidth,0) -- (\linewidth,#1) -- (0,#1) -- (0,0);
	\end{tikzpicture}}

\newcommand{\filluptopage}[1]{%
	\clearpage
	\loop\ifnum\value{page}<#1\relax
	\null\clearpage
	\repeat
}

%%Das package Version erlaubt verschiedene Modi: bei mir schueler und lehrer, standardmäßig wird die Lehrerversion gezeigt. Arbeitet mit exam und article, Schalter \printanswerstrue bzw. false
\RequirePackage{versions}
\usepackage{soulutf8}
%\includeversion{lehrer}\excludeversion{schueler}\printanswerstrue
\includeversion{eltern}\excludeversion{schueler}\printanswerstrue

%%%Definition der Lehrerumgebung, so dass sie heraussticht:
%\let\FMlehrer\lehrer
%\let\endFMlehrer\endlehrer
%\DeclareOption{hervorheben}{\renewenvironment{lehrer}
%	{\begin{color}{blue}\FMlehrer}
%		{\endFMlehrer \end{color}}
%}


%\m um wichtige Begriffe rot zu setzen:
\newcommand{\m}[1]{\textcolor{red}{#1}}

%\merke setzt einen Rahmen und fügt das Merke hinzu:
\RequirePackage[most]{tcolorbox}
\NewDocumentCommand{\merke}{o m}
{\vspace*{.2cm}
	\begin{tcolorbox}[colback=red!5!white,colframe=red!75!black,title=\IfValueTF{#1}{#1}{Merke},fonttitle=\bfseries]
		#2
	\end{tcolorbox}%
}


%%%%%%%%%%%%%%%%%%%%%%%%%%%%%%%%%%%
%https://tobiw.de/tbdm/lueckentexte
%%%%%%%%%%%%%%%%%%%%%%%%%%%%%%%%%%%

%\def\zz{\leavevmode\vrule height 10pt\nobreak\leaders\vbox{\vskip-.4pt\hrule width .4pt\vskip 10pt \hrule\vskip-.4pt} \hskip 15em minus 15em \nobreak\hbox{\vrule height 10pt}}



%% --- Testbereich --------------------------------
\newcommand{\skriptfigure}[3]{
	\begin{figure}[H]
		\centering 
		\vspace*{-1ex}
		\ifprintanswers
		\includegraphics[width=#1\linewidth]{./Bilder/#2}
		\else
		\includegraphics[width=#1\linewidth]{./Bilder/#3}
		\fi
	\end{figure}
}
%% ------------------------------------------------

%%%%%%%%%%%%%%%%%%%%%%%%%%%%%%%%%%%%%%%%%%%%%%%%%%%%%%%%%%%%%%%%%%%%%%%%%%%%%%%%%
%%%%%%%%%%%%%%%%%%%%%%%%%%%%%%%%%  Lösungen  %%%%%%%%%%%%%%%%%%%%%%%%%%%%%%%%%%%%
	\printanswers 							\fibhideanswerfalse
%	\noprintanswers		\printanswersfalse	\fibhideanswertrue
%%%%%%%%%%%%%%%%%%%%%%%%%%%%%%%%%%%%%%%%%%%%%%%%%%%%%%%%%%%%%%%%%%%%%%%%%%%%%%%%%
%%%%%%%%%%%%%%%%%%%%%%%%%%%%%%%%%%%%%%%%%%%%%%%%%%%%%%%%%%%%%%%%%%%%%%%%%%%%%%%%%

\newcommand{\ex}[4]{
	
	%	\thispagestyle{empty}
	\pagestyle{empty}
	
	\vspace*{-5mm}
	
	\centerline{\huge  3. Schulaufgabe in Physik}
	
	\vspace{6mm}
	
	\large
	
	Klasse:    #4							\hfill \hfill%
	Datum:    04$\cdot$05$\cdot$2023		\hfill \hfill%
	Name:
	\raisebox{-2ex}{$\stackrel{\hrulefill}{\qquad\qquad{\text{\footnotesize
					#2#3(#1)
			}}\qquad\qquad}$}
	
	\vspace{2mm}
	
	\begin{center}%\small
		\textsl{%
			Achte insbesondere auf die Anzahl der sinnvollen Ziffern, die Maßeinheiten sowie die äußere Form deiner Arbeit.\\
			Alle Lösungswege müssen nachvollziehbar und ausreichend dokumentiert sein.
			%\\ Runde stets auf zwei Stellen nach dem Komma!
		}
	\end{center}
	
	\vspace{2mm}
	
	
	\pointsinrightmargin
	\renewcommand{\partlabel}{\thequestion.\arabic{partno}}
	
	\begin{questions}
		\begingradingrange{myrange}
		
		\input{/Users/florian/Documents/Mybox/Physik/Aufgaben/10/Transformator/11.tex}
		
		\endgradingrange{myrange}		
	\end{questions}
	
	%\ansnp	
	%\noansnp
	%\rueckseitenurangabe
	%\naechsteseitenurangabe
	%\rueckseite
	%\newpage
	
	
	Gutes Gelingen, #2\!\!\! !
	\quad {\calligraphic Da}
	
	\marginpar{$\overline{\overline{\mbox{\textbf{\fontsize{10}{12}\selectfont (\rule[-1ex]{0ex}{4ex}\qquad$\arrowvert$\,\textsf{\pointsinrange{myrange}}\,)}}}}$}
	
	
	%%% nur für leere vierte Seite	
	%	\ifprintanswers\else 
	%	
	%	\newpage
	%	
	%	Platz für Nebenrechnungen oder weitere geniale Ideen:
	%	
	%	\smallskip
	%	
	%	\begin{solutionorgrid}[\stretch{1}]
	%		
	%	\end{solutionorgrid}
	%	
	%	\fi
	%%% nur für leere vierte Seite	END
	
	\newpage
}

\input{../../../../LaTeX/Klassen/2022_23_10AI}
