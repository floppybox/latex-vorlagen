\documentclass[11pt,a4paper,twoside,addpoints,color,answers]{exam}
\usepackage[utf8]{inputenc}
\usepackage[T1]{fontenc}

%% -> um Ligaturen beim Kopieren aus dem PDF aufzulösen
%
%\usepackage{libertine}
%\input{glyphtounicode}
%
%\pdfglyphtounicode{f_f}{FB00}
%\pdfglyphtounicode{f_f_i}{FB03}
%\pdfglyphtounicode{f_f_l}{FB04}
%\pdfglyphtounicode{f_i}{FB01}
%
%\pdfgentounicode=1
%
%% <- um Ligaturen beim Kopieren aus dem PDF aufzulösen

\usepackage[ngerman]{babel}
\usepackage{paralist}
\usepackage{extarrows}
\usepackage{graphicx}
\usepackage{wrapfig} % Produces figures which text can flow around !!! funktioniert nicht mit exam !!!
\usepackage[export]{adjustbox}
\usepackage{framed}
\usepackage{tabularx}
	\newcolumntype{Y}{>{\centering\arraybackslash}X}
\usepackage{qrcode}%QR-Codes generieren
\usepackage{multicol}%mehrspaltige Listen
\usepackage{xcolor}
\definecolor{shadecolor}{rgb}{0.9,0.9,0.9}
\usepackage{pifont} %Access to PostScript standard Symbol and Dingbats fonts
\usepackage{blindtext}
\usepackage{caption}
\usepackage[verbose]{linegoal}
\usepackage{subcaption}
\captionsetup[subfigure]{font=footnotesize,
	labelfont=up,
	%%singlelinecheck=false,
	%%format=hang,
	justification=centering % <-- new
}

\usepackage{float}
\usepackage{upgreek} % griechische Buchstaben aufrech z. B. für Präfixe
\usepackage{url}							% Verbatim with URL-sensitive line breaks
\usepackage{accents}
\usepackage{relsize} % für {\larger\textcircled{\smaller[2]1}}

\newcommand*{\zb}{z.\,B. }
\newcommand{\bx}[1]{\fbox{\rule[-0.5ex]{0ex}{3ex} \hspace*{#1ex}}}
\newcommand{\bluebox}[1]{\textcolor{blue}{\fbox{\rule[-1ex]{0ex}{3.5ex} \quad #1 \quad}}}

\newcommand{\grid}[1]{
	%		\newcount\G
	%		\G=#1
	%		\divide\G by 2
	%		\advance\G by 
	
	\begin{tikzpicture}[x=1cm,y=.5cm]
	\tikzset{dashdot/.style={dash pattern=on 0.1mm off 0.9mm on 0.1mm off 0.9mm}}
	\draw[%style=help lines ,upright
	dashdot
	%densely dotted
	%loosely dotted
	] (0,0) grid[step=0.5cm] (16,#1);
	\end{tikzpicture}
}

\newcommand{\gridn}[1]{
	%		\newcount\G
	%		\G=#1
	%		\divide\G by 2
	%		\advance\G by 
	
	\begin{tikzpicture}[x=1cm,y=.5cm]
	\tikzset{dashdot/.style={dash pattern=on 0.1mm off 0.9mm on 0.1mm off 0.9mm}}
	\draw[%style=help lines ,
	dashdot
	%densely dotted
	%loosely dotted
	] (0,0) grid[step=0.5cm] (16,#1);
	\end{tikzpicture}
}

\newcommand{\gridxy}[2]{
	
	\begin{tikzpicture}[x=.5cm,y=.5cm]
	\tikzset{dashdot/.style={dash pattern=on 0.1mm off 0.9mm on 0.1mm off 0.9mm}}
	\draw[%style=help lines ,
	dashdot
	%densely dotted
	%loosely dotted
	] (0,0) grid[step=0.5cm] (#1,#2);
	\end{tikzpicture}
}

\usepackage[most]{tcolorbox}
%package provides commands to produce colorful framed boxes which can also be applied to math environments.
\newtcolorbox{mymbox}[1]{
	%enhanced,
	ams align, 
	colback=yellow!5!white, 
	colframe=red!75!black, 
	fonttitle=\bfseries, 
	drop fuzzy shadow=blue!50!black!50!white,
	title=#1}
\newtcolorbox{mymbox*}[1]{
	%enhanced,
	ams align*, 
	colback=yellow!5!white, 
	colframe=red!75!black, 
	fonttitle=\bfseries, 
	drop fuzzy shadow=blue!50!black!50!white,
	title=#1}
\newtcolorbox{mybox}[1]{colback=red!5!white, colframe=red!75!black, fonttitle=\bfseries, title=#1}
\newtcolorbox{myboxyellow}[1]{colback=yellow!5!white, colframe=red!75!black, fonttitle=\bfseries, title=#1}
\newtcolorbox{myboxblue}[1]{colback=blue!5!white, colframe=blue!75!black, fonttitle=\bfseries, title=#1}

\usepackage{hyperref}						% Extensive support for hypertext in LATEX
\hypersetup{
	colorlinks	= true,
	linkcolor 	= black,
	urlcolor	= blue,
	breaklinks = true		
}
\usepackage[hyphenbreaks]{breakurl}			% Line-breakable \url-like links in hyperref when compiling via dvips/ps2pdf

\newcommand{\hervorheben}[1]{\textsl{#1}}

\usepackage{enumitem}% http://ctan.org/pkg/enumitem zur Anpassung von itemize und enumerate-Umgebungen [leftmargin=5mm,noitemsep,topsep=5pt,parsep=5pt,partopsep=0pt]

\newcommand{\mystrut}{\rule[-3ex]{0ex}{8ex}} % für Tabellen mit tabularx

%\renewcommand*{\dh}{d.~h.}

\usepackage{tabto} %https://tex.stackexchange.com/questions/14038/tabbing-inside-itemize-or-itemize-inside-tabbing

\newcommand*{\linie}{
	
	\vspace*{1ex}
	
	\hrulefill}

% Erweiterung zu \phantom{}, so dass auch negative Abstände "eingefügt" werden können:

\makeatletter
\newlength{\negph@wd}
\DeclareRobustCommand{\negphantom}[1]{%
	\ifmmode
	\mathpalette\negph@math{#1}%
	\else
	\negph@do{#1}%
	\fi
}
\newcommand{\negph@math}[2]{\negph@do{$\m@th#1#2$}}
\newcommand{\negph@do}[1]{%
	\settowidth{\negph@wd}{#1}%
	\hspace*{-\negph@wd}%
}
\makeatother

%\usepackage{stackengine,graphicx}
%\def\stacktype{L}
%\def\useanchorwidth{T}
%\newcommand\cancel[1]{\stackon[0pt]{#1}{\rotatebox{30}{\textcolor{red}{\rule{2.9ex}{1pt}}}}}

\usepackage{cancel}
\renewcommand{\CancelColor}{\color{red}}

\usepackage{ulem}			% Package for underlining

\usepackage{emptypage}

%\usepackage{xparse}
%
%\DeclareDocumentCommand{\hcancel}{mO{0pt}O{0pt}O{0pt}O{0pt}}{%
%	\tikz[baseline=(tocancel.base)]{
%		\node[inner sep=0pt,outer sep=0pt] (tocancel) {#1};
%		\draw[red] ($(tocancel.south west)+(#2,#3)$) -- ($(tocancel.north east)+(#4,#5)$);
%	}%
%}%
%

% zum schrägen Durchstreichen:

\usepackage{keycommand}
% Patch by Joseph Wright ("bug in the definition of \ifcommandkey (2010/04/27 v3.1415)"),
% https://tex.stackexchange.com/a/35794
\begingroup
\makeatletter
\catcode`\/=8 %
\@firstofone
{
	\endgroup
	\renewcommand{\ifcommandkey}[1]{%
		\csname @\expandafter \expandafter \expandafter
		\expandafter \expandafter \expandafter \expandafter
		\kcmd@nbk \commandkey {#1}//{first}{second}//oftwo\endcsname
	}
}
%--------%
\usepackage{tikz}
\usetikzlibrary{calc}
\newkeycommand{\hcancel}[hshiftstart=0pt,vshiftstart=0pt,hshiftend=0pt,vshiftend=0pt,color=red][1]{%
	\tikz[baseline=(tocancel.base)]{
		\node[inner sep=0pt,outer sep=0pt] (tocancel) {#1};
		\draw[\commandkey{color}] ($(tocancel.south west)+(\commandkey{hshiftstart},\commandkey{vshiftstart})$) --
		($(tocancel.north east)+(\commandkey{hshiftend},\commandkey{vshiftend})$);
	}%
}%


% farbig unterstreichen:

\newsavebox\MBox
\newcommand\Cline[2][red]{{\sbox\MBox{$#2$}%
		\rlap{\usebox\MBox}\color{#1}\rule[-1.2\dp\MBox]{\wd\MBox}{0.5pt}}}
	

\def\mathunderline#1#2{\color{#1}\underline{{#2}}}


% farbig hervorheben in math

\newcommand{\highlight}[2][red!50]{\mathpalette{\highlightwithstyle[#1]}{#2}}
\newcommand{\highlightwithstyle}[3][red!50]{
	\begingroup                         %% <- limit scope of \box0 and \fboxsep assignment
	\sbox0{$\mathsurround 0pt #2#3$}% %% <- typeset content in box 0
	\setlength{\fboxsep}{1pt}        %% <- set (smaller) framebox margins
	\sbox2{\hspace{-.5pt}%            %% <- create box 2, undo margin
		\colorbox{#1}{\usebox0}%        %% <- print the contents of box 0 in a \colorbox
	}%
	\dp2=\dp0 \ht2=\ht0 \wd2=\wd0     %% <- set dimensions of box 2 to match box 0
	\box2                             %% <- print box 2
	\endgroup                           %% <- revert old definitions of the boxes and \fboxsep
}

% für Tabellen
\usepackage{array,booktabs,tabularx}
\usepackage{ragged2e}

% für dem Befehlt \Longunderstack{} in Tabellen 
% Quelle: https://tex.stackexchange.com/a/183873/232548
\usepackage[usestackEOL]{stackengine}
\edef\tmp{\the\baselineskip}
\setstackgap{L}{\tmp}

% Funktioniert leider nicht:
%% Für das Malteserkreuz - Quelle: https://tex.stackexchange.com/questions/202746/how-to-draw-an-actual-maltese-cross/202856#202856
%\usepackage{fontspec}
%\usepackage{mathtools}
%\newcommand*{\amalfi}{\ensuremath{\text{\fontspec{code2000.ttf}\symbol{"2720}}}}

\usepackage{scalerel}
\newcommand{\maltcross}{\scalerel*{%
		\tikz\fill
		(0.02,0.02)    -- (0.2,0.5)   -- (0,0.4)  -- (-0.2,0.5)  --
		(-0.02,0.02)   -- (-0.5,0.2)  -- (-0.4,0) -- (-0.5,-0.2) --
		(-0.02,-0.02)  -- (-0.2,-0.5) -- (0,-0.4) -- (0.2,-0.5)  --
		(0.02,-0.02)   -- (0.5,-0.2)  -- (0.4,0)  -- (0.5,0.2)   --
		cycle;}% Scale this picture...
	{0}% ...to the size of this symbol.
}

\usepackage{tcolorbox}

%\usepackage[table]{xcolor}

%\usepackage{booktabs} 

% Text mit geschweifter Klammer und Text darunter
\newcommand{\undertext}[2] {$\underbrace{\textrm{#1}}_{\textrm{#2}}$} % Quelle: https://tex.stackexchange.com/questions/128064/how-to-overbrace-underbrace-in-text-mode-using-tikz

\usepackage{pifont}% http://ctan.org/pkg/pifont für \cmark und \xmark
\newcommand{\cmark}{\textcolor{green!50!black}{\ding{51}}}%
\newcommand{\xmark}{\textcolor{red!100!black}{\ding{55}}}%

\usepackage[autostyle,babel=true]{csquotes}	% Context sensitive quotation facilities
\renewcommand{\enquote}[1]{\glqq #1\grqq{}}
