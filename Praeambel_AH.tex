\usepackage{geometry}
\geometry{%
	a4paper,
%	verbose, 
	tmargin=9mm,
	bmargin=18mm,
	lmargin=24mm,
	rmargin=24mm,
	footskip=3mm, % Hiermit wird der vertikale Abstand zwischen der untersten Fußnote und der horizontalen Linie am unteren Ende der Seite angepasst.
	marginpar=0cm, % <====================================================
%		showframe % <= Anzeige des Rahmens auf der Seite - nur zur Kontrolle einschalten
}

% Anpassung der Fußnoten - siehe auch Anpassungen über geometry in Praeambel_AH.tex
\usepackage[hang, flushmargin, bottom]{footmisc}
\addtolength{\footnotesep}{1mm}

\extrafootheight{-6mm}

%\pagestyle{empty}  %auskommentiert, denn sonst wird die Fußzeile nicht angezeigt...

%%%%% \setlength %%%%%
\setlength{\parindent}{0em}       % kein Erstzeileneinzug
%\setlength{\textheight}{282mm} 
%\setlength{\textwidth}{167mm}
%\setlength{\oddsidemargin}{0mm} 
%\setlength{\evensidemargin}{-8mm} 
%\setlength{\topmargin}{-25mm}
%\setlength{\marginparsep}{8mm}
%\setlength{\rightpointsmargin}{3mm}
%%%%% \setlength %%%%%

%Lückentext: \luecke{Text, der in der Lücke steht} und eine Umgebung lueckentext, die dafür sorgt, dass richtig umgebrochen und der Zeilenabstand hochgesetzt wird.  (damit die Schüler Platz zum Schreiben haben)

\usepackage{setspace}
\newenvironment{lueckentext}
{\raggedright\begin{spacing}{1.7}}
	{\end{spacing}}
\newlength{\diebox}
\newcommand{\luecke}[1]{
	\settowidth{\diebox}{#1}
	\ifprintanswers
	\raisebox{0.1ex}{\parbox{2.3\diebox}{\textbf{\textcolor{magenta!100!black}{\rule[-2ex]{0pt}{5ex}#1}}}}
	\else
	%	\raisebox{-0.5ex}{\parbox{2.3\diebox}{\hrulefill}	}
	\raisebox{-0.5ex}{\parbox{2.3\diebox}{\rule[-1ex]{0pt}{5ex}\begin{tikzpicture}   
			\draw[color=gray, text=gray, dashed](0 ,0)--(\linewidth,0 );
			\end{tikzpicture}}	}
	\fi}  



%Umgebung für 1., 2., 3. (Aufgaben und Beispiele):
\newenvironment{aufzaehlung}
{\begin{enumerate}}
	{\end{enumerate}}

%Befehl, um Beispiele, Aufgaben und Herleitungen (Miniüberschriften) hervorzuheben:
\newcommand{\ueber}[1]{\par\medskip\textbf{#1}\par\smallskip}

%Befehl, um Latex beizubringen, was er beim \liniert{}-Befehl tun soll:
\newcommand{\liniert}[1]
{\par\noindent
	\begin{tikzpicture}   
	\clip (0,.85) rectangle(\linewidth, 0.9*(#1+1);
	\foreach \n in {1,...,#1}{\draw[color=gray, text=gray, dashed](0 ,0.9*\n )--(\linewidth,0.9*\n );}
	\end{tikzpicture}}

%Befehl, um Latex beizubringen, was er beim \kariert{}-Befehl tun soll:
\newcommand{\kariert}[1]{
	\par\noindent
	\begin{tikzpicture} 
	\clip (0,-0.05) rectangle (\linewidth-3pt,#1/2) ;   
	\draw[step=0.5,color=gray, dashed] (0,0) grid (\linewidth,#1/2); %Anzahl der Kästchenreihen
	\end{tikzpicture}}

%Befehl, um Latex beizubringen, was er beim \kariertprog{}-Befehl tun soll: (Karopapier mit Einrückungshilfe)
\newcommand{\kariertprog}[1]{
	\par\noindent
	\begin{tikzpicture} 
	\clip (0,-0.05) rectangle (\linewidth,#1/2) ;   
	\draw[step=0.5,color=gray, dashed] (0,0) grid (\linewidth,#1/2); %Anzahl der Kästchenreihen
	%\draw[step=1.5,color=gray] (0,0) grid (\linewidth,#1/2);
	\draw[gray] (1.5,0)--(1.5,#1/2);
	\draw[gray] (3,0)--(3,#1/2);
	\draw[gray] (4.5,0)--(4.5,#1/2);
	\draw[gray] (6,0)--(6,#1/2);
	\end{tikzpicture}}

%Befehl, um Latex beizubringen was er beim \platz{}-Befehl tun soll. Benutzen, um einen Kasten um Freiplatz für Aufgaben zu ziehen. Angabe der Höhe in Zentimeter.
\newcommand{\platz}[1]{
	\par\medskip\noindent
	\begin{tikzpicture}       
	\draw[dashed, color=gray] (0,0) -- (\linewidth,0) -- (\linewidth,#1) -- (0,#1) -- (0,0);
	\end{tikzpicture}}

\newcommand{\filluptopage}[1]{%
	\clearpage
	\loop\ifnum\value{page}<#1\relax
	\null\clearpage
	\repeat
}

%%Das package Version erlaubt verschiedene Modi: bei mir schueler und lehrer, standardmäßig wird die Lehrerversion gezeigt. Arbeitet mit exam und article, Schalter \printanswerstrue bzw. false
\RequirePackage{versions}
\usepackage{soulutf8}
%\includeversion{lehrer}\excludeversion{schueler}\printanswerstrue
\includeversion{eltern}\excludeversion{schueler}\printanswerstrue

%%%Definition der Lehrerumgebung, so dass sie heraussticht:
%\let\FMlehrer\lehrer
%\let\endFMlehrer\endlehrer
%\DeclareOption{hervorheben}{\renewenvironment{lehrer}
%	{\begin{color}{blue}\FMlehrer}
%		{\endFMlehrer \end{color}}
%}


%\m um wichtige Begriffe rot zu setzen:
\newcommand{\m}[1]{\textcolor{red}{#1}}

%\merke setzt einen Rahmen und fügt das Merke hinzu:
\RequirePackage[most]{tcolorbox}
\NewDocumentCommand{\merke}{o m}
{\vspace*{.2cm}
	\begin{tcolorbox}[colback=red!5!white,colframe=red!75!black,title=\IfValueTF{#1}{#1}{Merke},fonttitle=\bfseries]
		#2
	\end{tcolorbox}%
}


%%%%%%%%%%%%%%%%%%%%%%%%%%%%%%%%%%%
%https://tobiw.de/tbdm/lueckentexte
%%%%%%%%%%%%%%%%%%%%%%%%%%%%%%%%%%%

%\def\zz{\leavevmode\vrule height 10pt\nobreak\leaders\vbox{\vskip-.4pt\hrule width .4pt\vskip 10pt \hrule\vskip-.4pt} \hskip 15em minus 15em \nobreak\hbox{\vrule height 10pt}}

