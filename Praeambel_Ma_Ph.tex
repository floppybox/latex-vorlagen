\usepackage{expl3}							% für Berechnungen - siehe https://tex.stackexchange.com/questions/196171/inaccurate-pgf-calculation
\ExplSyntaxOn
\cs_set_eq:NN \fpeval \fp_eval:n
\ExplSyntaxOff

\usepackage{pgf}
\usepackage{tkz-euclide}
%\usepackage{amsfonts}						% TEX fonts from the American Mathematical Society
\usepackage{wasysym}   						% Electrical and physical symbols \AC \HF \VHF \photon \gluon
%\usepackage{mathtools}						% Mathematical tools to use with amsmath 
\usepackage{amsmath}						% AMS mathematical facilities for LATEX
\usepackage{amssymb}
\usepackage{isotope}						% A package for typesetting isotopes
%\usepackage{bohr}							% simple atom representation according to the Bohr model
%\usepackage{elements}
\usepackage[version=4]{mhchem}				% \ce{H2O}
\usepackage{fixmath}						% für $\mathbold{x}$
\usepackage{dsfont}							% Type­set math­e­mat­i­cal dou­ble stroke sym­bols
\usepackage{float}
%\usepackage{xfp}

\usepackage{scalerel}
\usepackage{stackengine}

%\usepackage{bm} % für fette Mathematik

\newcommand\reallywidetilde[1]{\ThisStyle{%
		\setbox0=\hbox{$\SavedStyle#1$}%
		\stackengine{-.5\LMpt}{$\SavedStyle#1$}{%
			\stretchto{\scaleto{\SavedStyle\mkern.2mu\frown}{.4\wd0}}{.5\ht0}%
		}{O}{c}{F}{T}{S}%
}}

\def\bogen#1{\ensuremath{\reallywidetilde{\text{#1}}}}

\usepackage{mathtools}						% für \begin{spreadlines}{2ex} Abstand in align ändern \end{spreadlines}


\usepackage{xstring}

%\newcommand{\pkt}[3]{%
%		\FPset\xpkt{#2}%
%		\FPset\ypkt{#3}%
%	\ensuremath{\text{#1}(\StrLeft{#2}{1}[\firstchar]\IfStrEq{\firstchar}{-}{\,}{\,}\num{\FPprint\xpkt}\,|\StrLeft{#3}{1}[\firstchari]\IfStrEq{\firstchari}{-}{\,}{\,}\num{\FPprint\ypkt}\,)}}

\newcommand{\pkt}[3]{\ensuremath{\mathrm{#1}\left(\,\SIx{#2}{}\,\left|\,\SIx{#3}{}\right.\,\right)}}
	
%\newcommand{\pkti}[3]{\ensuremath{\text{#1}(\;#3\;|\;#2\;)}}
%
%\newcommand{\pktii}[3]{P$(-2\,|-8\,)$}


%\newcommand{\pkt}[3]{% Funktioniert leider nur für Zahlen als Koordinaten...
%	\FPset\x{#2}%
%	\FPset\y{#3}%
%	\ensuremath{\text{#1}%
%	(%
%	\FPiflt{\x}{0}
%	%
%	\else
%	\,
%	\fi%
%	\num[round-mode=places,round-precision=2,zero-decimal-to-integer]{\FPprint\x}\,|%
%	\FPiflt{\y}{0}
%	\else
%	\,
%	\fi%
%	\num[round-mode=places,round-precision=2,zero-decimal-to-integer]{\FPprint\y}\,)}
%}

%Taschenrechnerbefehl für Quad. ergänzen und Nullstellen:
\newcommand{\trquad}[3]{
	\FPset\a{#1}
	\FPset\b{#2}
	\FPset\c{#3}
	\ensuremath{\textsf{TR  \; Menu \; A} \rightarrow 2 \rightarrow 2  \quad\; a=\num{\a}\; \rightarrow \; b=\num{\b}\; \rightarrow \; c=\num{\c}}}


%\usepackage{xfp}					% Interface to the LATEX3 floating point unit
%Taschenrechnerbefehl für Nullstellen und Extremwerte:
\newcommand{\trquadl}[3]{			% "Taschenrechner quadratische Gleichung mit Lösung" Eingabe der Koeffizienten mit Dezimalpunkt! 
	\FPset\a{#1}
	\FPset\b{#2}
	\FPset\c{#3}
	\FPeval\d{b*b-4*a*c}			% Berechnung der Diskriminante
	\ensuremath{
	\textsf{TR  \; Menu \; A} \rightarrow 2 \rightarrow 2  \quad\; a=\num{\a}\; \rightarrow \; b=\num{\b}\; \rightarrow \; c=\num{\c} \\[0.5ex]  			% Eingabe am TaschenRechner
%		(a=\a \quad b=\b \quad c=\c \quad D=\num[round-mode=places,round-precision=2]{\k}) \quad 
	\FPifgt{\d}{0} 					% wenn D>0
			\FPeval\e{round((-b+d^0.5)/(2*a),2)} % ohne round(...,2) kann auch mal "-0" rauskommen...
			\FPeval\f{round((-b-d^0.5)/(2*a),2)}
			\sisetup{round-mode=places,round-integer-to-decimal}
			x_1=\num[round-mode=places,round-precision=2,zero-decimal-to-integer]{\FPprint\e} \quad x_2=\num[round-mode=places,round-precision=2,zero-decimal-to-integer]{\FPprint\f} \qquad \mathds{L}=\{\num[round-mode=places,round-precision=2,zero-decimal-to-integer]{\FPprint\e}; \num[round-mode=places,round-precision=2,zero-decimal-to-integer]{\FPprint\f}\} \quad (D=\num[round-mode=places,round-precision=2,zero-decimal-to-integer]{\FPprint\d}>0)
		\else 
		\fi
	\FPifeq{\d}{0} 					% wenn D=0
			\FPeval\e{round((-b)/(2*a),2)} % 
			x=\num[round-mode=places,round-precision=2,zero-decimal-to-integer]{\FPprint\e} \qquad \mathds{L}=\{\num[round-mode=places,round-precision=2,zero-decimal-to-integer]{\FPprint\e}\} \quad (D=\num[round-mode=places,round-precision=2,zero-decimal-to-integer]{\FPprint\d})
		\else 
		\fi
	\FPiflt{\d}{0} 					% wenn D<0
			\mathds{L}=\varnothing \quad (D=\num[round-mode=places,round-precision=2,zero-decimal-to-integer]{\FPprint\d}<0)
		\else 
		\fi
}	
}

\newcommand{\trquadnl}[5]{			% "Taschenrechner quadratische Gleichung mit Lösung" Eingabe der Koeffizienten mit Dezimalpunkt! 
	\FPset\a{#1}
	\FPset\b{#2}
	\FPset\c{#3}
	\FPeval\d{b*b-4*a*c}			% Berechnung der Diskriminante
	\ensuremath{
		\textsf{TR  \; Menu \; A} \rightarrow 2 \rightarrow 2  \quad\; a=\num{\a}\; \rightarrow \; b=\num{\b}\; \rightarrow \; c=\num{\c} \\[0.5ex]  			% Eingabe am TaschenRechner
		%		(a=\a \quad b=\b \quad c=\c \quad D=\num[round-mode=places,round-precision=2]{\k}) \quad 
		\FPifgt{\d}{0} 					% wenn D>0
		\FPeval\e{round((-b+d^0.5)/(2*a),2)} % ohne round(...,2) kann auch mal "-0" rauskommen...
		\FPeval\f{round((-b-d^0.5)/(2*a),2)}
		\sisetup{round-mode=places,round-integer-to-decimal}
		x_#4=\num[round-mode=places,round-precision=2,zero-decimal-to-integer]{\FPprint\e} \quad x_#5=\num[round-mode=places,round-precision=2,zero-decimal-to-integer]{\FPprint\f} \qquad \mathds{L}=\{\num[round-mode=places,round-precision=2,zero-decimal-to-integer]{\FPprint\e}; \num[round-mode=places,round-precision=2,zero-decimal-to-integer]{\FPprint\f}\} \quad (D=\num[round-mode=places,round-precision=2,zero-decimal-to-integer]{\FPprint\d}>0)
		\else 
		\fi
		\FPifeq{\d}{0} 					% wenn D=0
		\FPeval\e{round((-b)/(2*a),2)} % 
		x=\num[round-mode=places,round-precision=2,zero-decimal-to-integer]{\FPprint\e} \qquad \mathds{L}=\{\num[round-mode=places,round-precision=2,zero-decimal-to-integer]{\FPprint\e}\} \quad (D=\num[round-mode=places,round-precision=2,zero-decimal-to-integer]{\FPprint\d})
		\else 
		\fi
		\FPiflt{\d}{0} 					% wenn D<0
		\mathds{L}=\varnothing \quad (D=\num[round-mode=places,round-precision=2,zero-decimal-to-integer]{\FPprint\d}<0)
		\else 
		\fi
	}	
}

%Taschenrechnerbefehl für Lineares Gleichungssystem:
\newcommand{\trlgs}[6]{\ensuremath{\textsf{TR  \; Menu \; A} \rightarrow 1 \rightarrow 2  \quad\; #1 \; \rightarrow \; #2 \; \rightarrow \; #3 \quad \rightarrow \quad  #4 \; \rightarrow \; #5 \; \rightarrow \; #6}}

\usepackage{siunitx}						% A comprehensive (SI) units package
	\sisetup{output-decimal-marker = {,}}	% Komma als Dezimaltrennzeichen für siunitx
	\sisetup{exponent-product = \cdot}		% Punkt statt x zwischen Zahl und Zehnerpotenz
	\sisetup{per-mode = fraction, fraction-function=\tfrac}			% Bruchstrich in Einheit
%	\sisetup{inter-unit-product = \ensuremath{{}\cdot{}}}
	\sisetup{inter-unit-product = \cdot}
	\sisetup{detect-weight=true, detect-family=true} 
	\sisetup{detect-all=true} % für fettes SI - Quelle: https://tex.stackexchange.com/questions/610211/how-to-have-bold-unit-with-siunitx
%	\sisetup{detect-weight=true, detect-family=true, mode=text} 
		% für fettes SI Quelle: https://tex.stackexchange.com/questions/612182/siunitx-bold-units-in-math-mode
		% Quelle: https://tex.stackexchange.com/questions/193389/siunitx-per-mode-fraction-in-displaystyle-with-tfrac
%	\sisetup{round-integer-to-decimal}
	\DeclareSIUnit\As{\text{As}}
	\DeclareSIUnit\Ah{\text{Ah}}
	\DeclareSIUnit\mAh{\text{mAh}}
	\DeclareSIUnit{\liter}{\ell}
	\DeclareSIPrefix\hekto{h}{h}
	\DeclareSIPrefix\zpmzwei{10^{-2}\,}{-2}
	\DeclareSIPrefix\zpmdrei{10^{-3}\,}{-3}
	\DeclareSIPrefix\zpmvier{10^{-4}\,}{-4}
	\DeclareSIPrefix\zpmsechs{10^{-6}\,}{-6}
%
	\DeclareSIPrefix\zppzwei{10^{2}\,}{2}
	\DeclareSIPrefix\zppdrei{10^{3}\,}{3}
	\DeclareSIPrefix\zppvier{10^{4}\,}{4}
	\DeclareSIPrefix\zppzweicenti{10^{2}\,c}{4}
	\DeclareSIPrefix\zppzweideci{10^{2}\,d}{4}
	\DeclareSIPrefix\zppdreideci{10^{3}\,d}{4}
	\DeclareSIPrefix\zppviercenti{10^{4}\,c}{4}
	\DeclareSIPrefix\zppfuenf{10^{5}\,}{5}
	\DeclareSIPrefix\zppsechs{10^{6}\,}{6}
\newcommand{\SIx}[2]{\SI[parse-numbers = false]{\text{$#1$}}{#2}}%damit gehen auch beliebige Terme als \num in \SI{\num}{\unit}

\AtBeginDocument{\DeclareSIUnit{\Wh}{Wh}}
\AtBeginDocument{\DeclareSIUnit{\kWh}{kWh}}
\AtBeginDocument{\DeclareSIUnit{\MWh}{MWh}}
\AtBeginDocument{\DeclareSIUnit{\GWh}{GWh}}
\AtBeginDocument{\DeclareSIUnit{\ct}{ct}}
\AtBeginDocument{\DeclareSIUnit{\EUR}{EUR}}
\AtBeginDocument{\DeclareSIUnit{\impuls}{Impulse}}
\AtBeginDocument{\DeclareSIUnit{\year}{a}}
%\AtBeginDocument{\DeclareSIUnit{\hekto}{h}}
\AtBeginDocument{\DeclareSIUnit{\c}{c}}
\AtBeginDocument{\DeclareSIUnit{\e}{e}}
\AtBeginDocument{\DeclareSIUnit{\MeV}{MeV}}
%\AtBeginDocument{\DeclareSIUnit{\giga}{G}}
%\AtBeginDocument{\DeclareSIUnit{\mega}{M}}
%\AtBeginDocument{\DeclareSIUnit{\kilo}{k}}
%\AtBeginDocument{\DeclareSIUnit{\deca}{da}}
%\AtBeginDocument{\DeclareSIUnit{\deka}{da}}
%\AtBeginDocument{\DeclareSIUnit{\deci}{d}}
%\AtBeginDocument{\DeclareSIUnit{\centi}{c}}
%\AtBeginDocument{\DeclareSIUnit{\milli}{m}}
%\AtBeginDocument{\DeclareSIUnit{\micro}{\text{$\umu$}}}
%\AtBeginDocument{\DeclareSIUnit{\nano}{n}}









%Befehl für (leere) Koordinatensysteme: \kos{links}{unten}{rechts}{oben}. Beschriftet mit der Eins in jede Richtung und dem minimalen bzw. maximalen Wert.   
\newcommand{\kos}[4]{
	\clip (#1-0.5,#2-0.5) rectangle (#3+1,#4+1);
	\draw [color=gray!50]  [step=5mm] (#1-1,#2-1) grid (#3+1,#4+1); %Raster zeichnen
	\draw[->,thick] (#1-0.5,0) -- (#3+0.5,0) node[right] {$x$}; % x-Achse zeichnen
	\draw[->,thick] (0,#2-0.5) -- (0,#4+0.5) node[above] {$y$}; % y-Achse zeichnen
	\foreach \x in {#1,-1,1,#3} \draw (\x,-.1)--(\x,.1) node[below=4pt] {$\scriptstyle\x$};%x-Achse beschriften
	\foreach \y in {#2,-1,1,#4} \draw (-.1,\y)--(.1,\y) node[left=4pt] {$\scriptstyle\y$};   % y-Achse beschriften
	\draw[color=black] (-4pt,-6pt) node[] {$0$};     % schöner Ursprung
	\path[name path=rahmen,clip](#1-0.5,#2-0.5) rectangle (#3+0.5,#4+0.5);
}

%Lineares Gleichungssystem, 4 Argumente: 1. Zeile links, rechts, 2. Zeile links rechts:
\newcommand{\lgs}[4]{
	\begin{tabular}{|lrcl}
		&$#1$&=&$#2$\\$\land$&$#3$&=&$#4$\\\hline
	\end{tabular}}



\newcommand{\kommando}{\qquad\rule[-1ex]{0.5pt}{3ex}\hspace*{1ex}}
\usepackage{etoolbox}% for \ifblank{#3}{true}{false}
%\lings wie \lgs nur mit zwei zusätzlichen Argumenten für etwaige Äquivalenzumformungen
\newcommand{\lings}[6]{
	\begin{tabular}{rrcll}
				I	&$#1$\!\!\!&=&\!\!\!$#2$&$\ensuremath{\ifblank{#3}{}{\qquad\rule[-1ex]{0.5pt}{3ex}\hspace*{1ex} #3}}$\\[0.5ex] 
	$\wedge$	II	&$#4$\!\!\!&=&\!\!\!$#5$&$\ensuremath{\ifblank{#6}{}{\qquad\rule[-1ex]{0.5pt}{3ex}\hspace*{1ex} #6}}$\\\cline{1-4} 
	\end{tabular}
	}



\newcommand{\linievar}[1]{{\color{gray}{\hdashrule{#1}{.5pt}{3pt}}}}%Linien mit variabler Länge: \linie{länge mit Einheit}

%%kleinergleich und größergleich wie in der Realschule verlangt (nicht mehr im LP+!!!) (mit Doppelstrich):
%\renewcommand{\leq}{\leq}\renewcommand{\geq}{\geq}

\newcommand{\laenge}[1]{\ensuremath{\left|\overline{\text{#1}}\right|}}%Die Streckenlänge \laenge{}, muss mit dem neuen Lehrplan geändert werden!
\newcommand{\strecke}[1]{\ensuremath{\overline{\text{#1}}}}%Die Strecke \strecke{}, muss mit dem neuen Lehrplan geändert werden!
\newcommand{\gerade}[1]{#1}
%\newcommand{\winkel}[1]{\ensuremath{\sphericalangle \text{#1}}}
\newcommand{\winkel}[1]{\ensuremath{\varangle \text{#1}}}
\newcommand{\dreieck}[1]{\ensuremath{\bigtriangleup\text{#1}}}


%Optionaler Parameter bei \draw: extended line für Geraden und Co:
\tikzset{dot/.style={label={#1},name=#1},
	extended line/.style={shorten >=-#1,shorten <=-#1},
	extended line/.default=1cm]}

%Punkte mit Kreuzchen markiert:(funktioniert nur in Tikz-Picture-Umgebung!) zuerst name, dann x dann y, dann Position
\tikzset{point/.style={thick, draw=black,cross out, inner sep=0pt, minimum width=4pt, minimum height=4pt}}
\newcommand{\punkt}[4]{\coordinate[label=#4:$#1$, point](#1) at (#2,#3);}

\newcommand{\determinante}[4]{\begin{vmatrix} #1 & #2\\#3 & #4\\\end{vmatrix}}%Befehl für Determinanten, zeilenweise: zuerst die x-Werte, dann die y-Werte. Benötigt den Mathemodus

\newcommand{\vektor}[2]{\dbinom{#1}{#2}}%Befehl, um einen Vektor in der Spaltenschreibweise zu erstellen

%\newcommand{\entspricht}{\widehat{=}}%das entspricht-Zeichen

\newcommand\entspricht{\mathrel{\stackon[1.5pt]{=}{\stretchto{\scalerel*[\widthof{=}]{\wedge}{\rule{1ex}{3ex}}}{0.5ex}}}} %Quelle: https://tex.stackexchange.com/questions/103408/symbol-for-corresponds-to-equals-sign-with-hat%


%\newcommand{\platzbedarf}[4]{(Platzbedarf: $#1\leq x\leq #2$; $#3\leq y\leq 4$)}
\newcommand{\platzbedarf}[4]{Für die Zeichnung: \; Längeneinheit \SI{1}{\centi \meter};  \; $#1\leq x\leq #2$; \; $#3\leq y\leq #4$}

\usepackage[school]{pgf-umlcd}%Objektkarten mit dem package pgf-umlcd
\renewcommand{\umltextcolor}{black}%Textfarbe von Objektkarten:
\renewcommand{\umldrawcolor}{black}%Linienfarbe von Objektkarten:
\renewcommand{\umlfillcolor}{white}%Hintergrundfarbe von Objektkarten:

\newcommand{\kosy}[5]{
	\begin{center}
		\tikzstyle{background grid}=[draw, black!20,step=5mm,line width=1pt]
		\begin{tikzpicture}[scale=1.0,samples=400%
		%			,show background grid
		]
		
		%% Größe des KOSYs %%%%%%%%%%%%%%%%%%%%%%%%%%%%%%%%%%%%%%%%%%
		
		\pgfmathsetmacro{\xmin}{#1}
		\pgfmathsetmacro{\xmax}{#2}
		\pgfmathsetmacro{\ymin}{#3}
		\pgfmathsetmacro{\ymax}{#4}
		
		%%%%%%%%%%%%%%%%%%%%%%%%%%%%%%%%%%%%%%%%%%%%%%%%%%%%%%%%%%%%%
		
		\pgfmathtruncatemacro{\xmintick}{\xmin}
		\pgfmathtruncatemacro{\xmaxtick}{\xmax}
		\pgfmathtruncatemacro{\ymintick}{\ymin}
		\pgfmathtruncatemacro{\ymaxtick}{\ymax}
		
		\pgfmathtruncatemacro{\xmingrid}{\xmin-1}
		\pgfmathtruncatemacro{\xmaxgrid}{\xmax+1}
		\pgfmathtruncatemacro{\ymingrid}{\ymin-1}
		\pgfmathtruncatemacro{\ymaxgrid}{\ymax+1}
		
		\pgfmathsetmacro{\xminkosy}{\xmin-0.5}
		\pgfmathsetmacro{\xmaxkosy}{\xmax+0.5}
		\pgfmathsetmacro{\yminkosy}{\ymin-0.5}
		\pgfmathsetmacro{\ymaxkosy}{\ymax+0.5}
		
		% Definition des Gitters und das Gitter selbst
		
		\tikzstyle{dotted}= [dash pattern=on 0.15mm off 0.85mm]
		\draw[help lines,line width=0.15mm,step=0.5,style=dotted,black,line cap=round] (\xmingrid,\ymingrid) grid (\xmaxgrid,\ymaxgrid);
		
		% Ursprung
		
		\node[below left =5pt] (0,0) {O};
		
		% Beschriftung der Koordinatenachsen mit Ticks (muss vor den Achsen selbst kommen, sonst sind die Achsen z. T. abgedeckt)
		
		\foreach \x in {\xmintick,...,-1,1,2,...,\xmaxtick}
		{
			\draw node[anchor=north,fill=white, fill opacity=0.7] at (\x,-0.2) {\x};
			\draw (\x,0pt) -- (\x,-2pt) node[anchor=north] at (\x,-0.2) {\x};
		}		
		
		\foreach \y in {\ymintick,...,-1,1,2,...,\ymaxtick}
		{
			\draw node[anchor=east,fill=white, fill opacity=0.7] at (-0.1,\y) {\y};
			\draw (0pt,\y) -- (-2pt,\y) node[anchor=east] at (-0.1,\y) {\y};
		}
		
		% Koordinatenachsen und Beschriftung mit $x$ und $y$
		
		\draw[-latex,line width=0.2mm] (\xminkosy,0) --(\xmaxkosy,0) node[below right = 4pt]{$x$};   
		\draw[-latex,line width=0.2mm] (0,\yminkosy) --(0,\ymaxkosy) node[above left = 4pt]{$y$};
		
		#5
		
		\end{tikzpicture}
	\end{center}
}


%Braucht man für das kosycustomi:

\usepackage{fp}
\usepackage{pgfkeys,pgfmath,pgfcore}

\pgfkeys{
	/textnumber/.style={
		/pgf/number format/.cd,% <- changes the prefix for the following options
		fixed,
		use comma,
		fixed zerofill,
		precision=0, % Hier kann man die Anzahl der Nachkommastellen für das KOSY eintragen
		1000 sep={.},
	},
}

%Braucht man für das kosycustomii:

\pgfkeys{
	/textnumber2/.style={
		/pgf/number format/.cd,% <- changes the prefix for the following options
%		fixed,
		use comma,
		fixed zerofill,
		precision=2, % Hier kann man die Anzahl der Nachkommastellen für das KOSY eintragen
		1000 sep={.},
	},
}


\newcommand{\kosycustomi}[9]{
	\begin{center}
		\begin{tikzpicture}[scale=1,samples=400]
		
		%% Größe des KOSYs %%%%%%%%%%%%%%%%%%%%%%%%%%%%%%%%%%%%%%%%%%
		
		\pgfmathsetmacro{\xmin}{#1}
		\pgfmathsetmacro{\xmax}{#2}
		\pgfmathsetmacro{\ymin}{#3}
		\pgfmathsetmacro{\ymax}{#4}
		
		%%%%%%%%%%%%%%%%%%%%%%%%%%%%%%%%%%%%%%%%%%%%%%%%%%%%%%%%%%%%%
		
		\pgfmathtruncatemacro{\xmintick}{1+\xmin}
		\pgfmathtruncatemacro{\xmaxtick}{\xmax}
		\pgfmathtruncatemacro{\ymintick}{1+\ymin}
		\pgfmathtruncatemacro{\ymaxtick}{\ymax}
		
		\pgfmathtruncatemacro{\xmingrid}{\xmin-1}
		\pgfmathtruncatemacro{\xmaxgrid}{\xmax+2}
		\pgfmathtruncatemacro{\ymingrid}{\ymin-1}
		\pgfmathtruncatemacro{\ymaxgrid}{\ymax+1}
		
		\pgfmathsetmacro{\xminkosy}{\xmin-0.5}
		\pgfmathsetmacro{\xmaxkosy}{\xmax+0.5}
		\pgfmathsetmacro{\yminkosy}{\ymin-0.5}
		\pgfmathsetmacro{\ymaxkosy}{\ymax+0.5}
		
		% Definition des Gitters und das Gitter selbst
		
		\tikzstyle{dotted}= [dash pattern=on 0.15mm off 0.85mm]
		\draw[help lines,line width=0.15mm,step=0.5,style=dotted,black!20,line cap=round] (\xmingrid,\ymingrid) grid (\xmaxgrid,\ymaxgrid);
		
		% Ursprung
		
		\node[below left=2pt] (0,0) {\footnotesize 0};
		
		% Beschriftung der Koordinatenachsen mit Ticks (muss vor den Achsen selbst kommen, sonst sind die Achsen z. T. abgedeckt)
		
		\foreach \x in {\xmintick,...,1,2,3,...,\xmaxtick}
		{
			\pgfmathsetmacro\resultx{#7*\x}
			%			\draw node[anchor=north,fill=white, fill opacity=0.9] at (\x,0) {\footnotesize \resultx};
			\draw (\x,0pt) -- (\x,-2pt) node[anchor=north] at (\x,-0.1) {\footnotesize \pgfmathprintnumber[/textnumber]{\resultx}};
		}		
		
		\foreach \y in {\ymintick,...,1,2,3,...,\ymaxtick}
		{
			\pgfmathsetmacro\resulty{#8*(\y)}
			%			\draw node[anchor=east,fill=white, fill opacity=0.9] at (0,\y) {\footnotesize \resulty};
			\draw (0pt,\y) -- (-2pt,\y) node[anchor=east] at (-0.1,\y) {\footnotesize \pgfmathprintnumber[/textnumber]{\resulty}};
		}
		
		% Koordinatenachsen und Beschriftung mit $x$ und $y$
		
		\draw[-latex,line width=0.2mm] (\xminkosy,0) --(\xmaxkosy,0) node[right]{#5};   
		\draw[-latex,line width=0.2mm] (0,\yminkosy) --(0,\ymaxkosy) node[above]{#6};
		
		#9
		
		\end{tikzpicture}
	\end{center}
}	

\newcommand{\kosyquali}[7]{ % qualitatives Diagramm ohne Skalierung
	\begin{center}
		\begin{tikzpicture}[scale=0.707,samples=400]
		
		%% Größe des KOSYs %%%%%%%%%%%%%%%%%%%%%%%%%%%%%%%%%%%%%%%%%%
		
		\pgfmathsetmacro{\xmin}{#1}
		\pgfmathsetmacro{\xmax}{#2}
		\pgfmathsetmacro{\ymin}{#3}
		\pgfmathsetmacro{\ymax}{#4}
		
		%%%%%%%%%%%%%%%%%%%%%%%%%%%%%%%%%%%%%%%%%%%%%%%%%%%%%%%%%%%%%
		
		\pgfmathtruncatemacro{\xmintick}{1+\xmin}
		\pgfmathtruncatemacro{\xmaxtick}{\xmax}
		\pgfmathtruncatemacro{\ymintick}{1+\ymin}
		\pgfmathtruncatemacro{\ymaxtick}{\ymax}
		
		\pgfmathtruncatemacro{\xmingrid}{\xmin-1}
		\pgfmathtruncatemacro{\xmaxgrid}{\xmax+2}
		\pgfmathtruncatemacro{\ymingrid}{\ymin-1}
		\pgfmathtruncatemacro{\ymaxgrid}{\ymax+1}
		
		\pgfmathsetmacro{\xminkosy}{\xmin-0.5}
		\pgfmathsetmacro{\xmaxkosy}{\xmax+0.5}
		\pgfmathsetmacro{\yminkosy}{\ymin-0.5}
		\pgfmathsetmacro{\ymaxkosy}{\ymax+0.5}
		
		% Definition des Gitters und das Gitter selbst
		
		\tikzstyle{dotted}= [dash pattern=on 0.15mm off 0.85mm]
		\draw[help lines,line width=0.15mm,step=0.5,style=dotted,black!80,line cap=round] (\xmingrid,\ymingrid) grid (\xmaxgrid,\ymaxgrid);
		
%		% Ursprung
%		
%		\node[below left=2pt] (0,0) {\footnotesize 0};
%		
%		% Beschriftung der Koordinatenachsen mit Ticks (muss vor den Achsen selbst kommen, sonst sind die Achsen z. T. abgedeckt)
%		
%		\foreach \x in {\xmintick,...,1,2,3,...,\xmaxtick}
%		{
%			\pgfmathsetmacro\resultx{#7*\x}
%			%			\draw node[anchor=north,fill=white, fill opacity=0.9] at (\x,0) {\footnotesize \resultx};
%			\draw (\x,0pt) -- (\x,-2pt) node[anchor=north] at (\x,-0.1) {\footnotesize \pgfmathprintnumber[/textnumber]{\resultx}};
%		}		
%		
%		\foreach \y in {\ymintick,...,1,2,3,...,\ymaxtick}
%		{
%			\pgfmathsetmacro\resulty{#8*(\y)}
%			%			\draw node[anchor=east,fill=white, fill opacity=0.9] at (0,\y) {\footnotesize \resulty};
%			\draw (0pt,\y) -- (-2pt,\y) node[anchor=east] at (-0.1,\y) {\footnotesize \pgfmathprintnumber[/textnumber]{\resulty}};
%		}
		
		% Koordinatenachsen und Beschriftung mit $x$ und $y$
		
		\draw[-latex,line width=0.2mm] (\xminkosy,0) --(\xmaxkosy,0) node[right]{#5};   
		\draw[-latex,line width=0.2mm] (0,\yminkosy) --(0,\ymaxkosy) node[above, yshift=0.5ex]{#6};
			
		#7
						
		\end{tikzpicture}
	\end{center}
}	

\newcommand{\kosycustomii}[9]{
	\begin{center}
		\begin{tikzpicture}[scale=1,samples=400]
		
		%% Größe des KOSYs %%%%%%%%%%%%%%%%%%%%%%%%%%%%%%%%%%%%%%%%%%
		
		\pgfmathsetmacro{\xmin}{#1}
		\pgfmathsetmacro{\xmax}{#2}
		\pgfmathsetmacro{\ymin}{#3}
		\pgfmathsetmacro{\ymax}{#4}
		
		%%%%%%%%%%%%%%%%%%%%%%%%%%%%%%%%%%%%%%%%%%%%%%%%%%%%%%%%%%%%%
		
		\pgfmathtruncatemacro{\xmintick}{1+\xmin}
		\pgfmathtruncatemacro{\xmaxtick}{\xmax}
		\pgfmathtruncatemacro{\ymintick}{1+\ymin}
		\pgfmathtruncatemacro{\ymaxtick}{\ymax}
		
		\pgfmathtruncatemacro{\xmingrid}{\xmin-1}
		\pgfmathtruncatemacro{\xmaxgrid}{\xmax+1}
		\pgfmathtruncatemacro{\ymingrid}{\ymin-1}
		\pgfmathtruncatemacro{\ymaxgrid}{\ymax+1}
		
		\pgfmathsetmacro{\xminkosy}{\xmin-0.5}
		\pgfmathsetmacro{\xmaxkosy}{\xmax+0.5}
		\pgfmathsetmacro{\yminkosy}{\ymin-0.5}
		\pgfmathsetmacro{\ymaxkosy}{\ymax+0.5}
		
		% Definition des Gitters und das Gitter selbst
		
		\tikzstyle{dotted}= [dash pattern=on 0.15mm off 0.85mm]
		\draw[help lines,line width=0.15mm,step=0.5,style=dotted,black!20,line cap=round] (\xmingrid,\ymingrid) grid (\xmaxgrid,\ymaxgrid);
		
		% Ursprung
		
		\node[below left=2pt] (0,0) {\footnotesize 0};
		
		% Beschriftung der Koordinatenachsen mit Ticks (muss vor den Achsen selbst kommen, sonst sind die Achsen z. T. abgedeckt)
		
		\foreach \x in {\xmintick,...,1,2,3,...,\xmaxtick}
		{
			\pgfmathsetmacro\resultx{#7*\x}
			\draw (\x,0pt) -- (\x,-2pt) node[anchor=north] at (\x,-0.1) 
				{\footnotesize \num[round-mode=figures,round-precision=2]{\resultx}};
		}		
		
		\foreach \y in {\ymintick,...,1,2,3,...,\ymaxtick}
		{
			\pgfmathsetmacro\resulty{#8*\y}
			\draw (0pt,\y) -- (-2pt,\y) node[anchor=east] at (-0.1,\y) 
				{\footnotesize \num[round-mode=figures,round-precision=2]{\resulty}};
		}
		
		% Koordinatenachsen und Beschriftung mit $x$ und $y$
		
		\draw[-latex,line width=0.2mm] (\xminkosy,0) --(\xmaxkosy,0) node[left, xshift=-0mm, yshift=-8mm]{#5};   
		\draw[-latex,line width=0.2mm] (0,\yminkosy) --(0,\ymaxkosy) node[above, , yshift=1mm]{#6};
		
		#9
		
		\end{tikzpicture}
	\end{center}
}	

\newcommand{\kosycustomiii}[9]{
	\begin{center}
		\begin{tikzpicture}[scale=1,samples=400]
		
		%% Größe des KOSYs %%%%%%%%%%%%%%%%%%%%%%%%%%%%%%%%%%%%%%%%%%
		
		\pgfmathsetmacro{\xmin}{#1}
		\pgfmathsetmacro{\xmax}{#2}
		\pgfmathsetmacro{\ymin}{#3}
		\pgfmathsetmacro{\ymax}{#4}
		
		%%%%%%%%%%%%%%%%%%%%%%%%%%%%%%%%%%%%%%%%%%%%%%%%%%%%%%%%%%%%%
		
		\pgfmathtruncatemacro{\xmintick}{1+\xmin}
		\pgfmathtruncatemacro{\xmaxtick}{\xmax}
		\pgfmathtruncatemacro{\ymintick}{1+\ymin}
		\pgfmathtruncatemacro{\ymaxtick}{\ymax}
		
		\pgfmathtruncatemacro{\xmingrid}{\xmin-1}
		\pgfmathtruncatemacro{\xmaxgrid}{\xmax+1}
		\pgfmathtruncatemacro{\ymingrid}{\ymin-1}
		\pgfmathtruncatemacro{\ymaxgrid}{\ymax+1}
		
		\pgfmathsetmacro{\xminkosy}{\xmin-0.5}
		\pgfmathsetmacro{\xmaxkosy}{\xmax+0.5}
		\pgfmathsetmacro{\yminkosy}{\ymin-0.5}
		\pgfmathsetmacro{\ymaxkosy}{\ymax+0.5}
		
		% Definition des Gitters und das Gitter selbst
		
		\tikzstyle{dotted}= [dash pattern=on 0.15mm off 0.85mm]
		\draw[help lines,line width=0.15mm,step=0.5,style=dotted,black!20,line cap=round] (\xmingrid,\ymingrid) grid (\xmaxgrid,\ymaxgrid);
		
		% Ursprung
		
		\node[below left=2pt] (0,0) {\footnotesize 0};
		
		% Beschriftung der Koordinatenachsen mit Ticks (muss vor den Achsen selbst kommen, sonst sind die Achsen z. T. abgedeckt)
		
		\foreach \x in {\xmintick,...,1,2,3,...,\xmaxtick}
		{
			\pgfmathsetmacro\resultx{#7*\x}
			\draw (\x,0pt) -- (\x,-2pt) node[anchor=north] at (\x,-0.1) 
			{\footnotesize \num[round-mode=figures,round-precision=3]{\resultx}};
		}		
		
		\foreach \y in {\ymintick,...,1,2,3,...,\ymaxtick}
		{
			\pgfmathsetmacro\resulty{#8*\y}
			\draw (0pt,\y) -- (-2pt,\y) node[anchor=east] at (-0.1,\y) 
			{\footnotesize \num[round-mode=figures,round-precision=2]{\resulty}};
		}
		
		% Koordinatenachsen und Beschriftung mit $x$ und $y$
		
		\draw[-latex,line width=0.2mm] (\xminkosy,0) --(\xmaxkosy,0) node[left, xshift=-0mm, yshift=-8mm]{#5};   
		\draw[-latex,line width=0.2mm] (0,\yminkosy) --(0,\ymaxkosy) node[above]{#6};
		
		#9
		
		\end{tikzpicture}
	\end{center}
}	

\newcommand{\kosycustomiv}[9]{
	\begin{center}
		\begin{tikzpicture}[scale=1,samples=400]
		
		%% Größe des KOSYs %%%%%%%%%%%%%%%%%%%%%%%%%%%%%%%%%%%%%%%%%%
		
		\pgfmathsetmacro{\xmin}{#1}
		\pgfmathsetmacro{\xmax}{#2}
		\pgfmathsetmacro{\ymin}{#3}
		\pgfmathsetmacro{\ymax}{#4}
		
		%%%%%%%%%%%%%%%%%%%%%%%%%%%%%%%%%%%%%%%%%%%%%%%%%%%%%%%%%%%%%
		
		\pgfmathtruncatemacro{\xmintick}{1+\xmin}
		\pgfmathtruncatemacro{\xmaxtick}{\xmax}
		\pgfmathtruncatemacro{\ymintick}{1+\ymin}
		\pgfmathtruncatemacro{\ymaxtick}{\ymax}
		
		\pgfmathtruncatemacro{\xmingrid}{\xmin-1}
		\pgfmathtruncatemacro{\xmaxgrid}{\xmax+1}
		\pgfmathtruncatemacro{\ymingrid}{\ymin-1}
		\pgfmathtruncatemacro{\ymaxgrid}{\ymax+1}
		
		\pgfmathsetmacro{\xminkosy}{\xmin-0.5}
		\pgfmathsetmacro{\xmaxkosy}{\xmax+0.5}
		\pgfmathsetmacro{\yminkosy}{\ymin-0.5}
		\pgfmathsetmacro{\ymaxkosy}{\ymax+0.5}
		
		% Definition des Gitters und das Gitter selbst
		
		\tikzstyle{dotted}= [dash pattern=on 0.15mm off 0.85mm]
		\draw[help lines,line width=0.15mm,step=0.5,style=dotted,black!20,line cap=round] (\xmingrid,\ymingrid) grid (\xmaxgrid,\ymaxgrid);
		
		% Ursprung
		
		\node[below left=2pt] (0,0) {\footnotesize 0};
		
		% Beschriftung der Koordinatenachsen mit Ticks (muss vor den Achsen selbst kommen, sonst sind die Achsen z. T. abgedeckt)
		
		\foreach \x in {\xmintick,...,1,2,3,...,\xmaxtick}
		{
			\pgfmathsetmacro\resultx{#7*\x}
			\draw (\x,0pt) -- (\x,-2pt) node[anchor=north] at (\x,-0.1) 
			{\footnotesize \num[round-mode=places,round-precision=1]{\resultx}};
		}		
		
		\foreach \y in {\ymintick,...,1,2,3,...,\ymaxtick}
		{
			\pgfmathsetmacro\resulty{#8*\y}
			\draw (0pt,\y) -- (-2pt,\y) node[anchor=east] at (-0.1,\y) 
			{\footnotesize \num[round-mode=places,round-precision=1]{\resulty}};
		}
		
		% Koordinatenachsen und Beschriftung mit $x$ und $y$
		
		\draw[-latex,line width=0.2mm] (\xminkosy,0) --(\xmaxkosy,0) node[left, xshift=-0mm, yshift=-8mm]{#5};   
		\draw[-latex,line width=0.2mm] (0,\yminkosy) --(0,\ymaxkosy) node[above]{#6};
		
		#9
		
		\end{tikzpicture}
	\end{center}
}	

\newcommand{\kosycustomiiinkii}[9]{
	\begin{center}
		\begin{tikzpicture}[scale=1,samples=400]
		
		%% Größe des KOSYs %%%%%%%%%%%%%%%%%%%%%%%%%%%%%%%%%%%%%%%%%%
		
		\pgfmathsetmacro{\xmin}{#1}
		\pgfmathsetmacro{\xmax}{#2}
		\pgfmathsetmacro{\ymin}{#3}
		\pgfmathsetmacro{\ymax}{#4}
		
		%%%%%%%%%%%%%%%%%%%%%%%%%%%%%%%%%%%%%%%%%%%%%%%%%%%%%%%%%%%%%
		
		\pgfmathtruncatemacro{\xmintick}{1+\xmin}
		\pgfmathtruncatemacro{\xmaxtick}{\xmax}
		\pgfmathtruncatemacro{\ymintick}{1+\ymin}
		\pgfmathtruncatemacro{\ymaxtick}{\ymax}
		
		\pgfmathtruncatemacro{\xmingrid}{\xmin-1}
		\pgfmathtruncatemacro{\xmaxgrid}{\xmax+1}
		\pgfmathtruncatemacro{\ymingrid}{\ymin-1}
		\pgfmathtruncatemacro{\ymaxgrid}{\ymax+1}
		
		\pgfmathsetmacro{\xminkosy}{\xmin-0.5}
		\pgfmathsetmacro{\xmaxkosy}{\xmax+0.5}
		\pgfmathsetmacro{\yminkosy}{\ymin-0.5}
		\pgfmathsetmacro{\ymaxkosy}{\ymax+0.5}
		
		% Definition des Gitters und das Gitter selbst
		
		\tikzstyle{dotted}= [dash pattern=on 0.15mm off 0.85mm]
		\draw[help lines,line width=0.15mm,step=0.5,style=dotted,black!20,line cap=round] (\xmingrid,\ymingrid) grid (\xmaxgrid,\ymaxgrid);
		
		% Ursprung
		
		\node[below left=2pt] (0,0) {\footnotesize 0};
		
		% Beschriftung der Koordinatenachsen mit Ticks (muss vor den Achsen selbst kommen, sonst sind die Achsen z. T. abgedeckt)
		
		\foreach \x in {\xmintick,...,1,2,3,...,\xmaxtick}
		{
			\pgfmathsetmacro\resultx{#7*\x}
			\draw (\x,0pt) -- (\x,-2pt) node[anchor=north] at (\x,-0.1) 
			{\footnotesize \num[round-mode=places,round-precision=3]{\resultx}};
		}		
		
		\foreach \y in {\ymintick,...,1,2,3,...,\ymaxtick}
		{
			\pgfmathsetmacro\resulty{#8*\y}
			\draw (0pt,\y) -- (-2pt,\y) node[anchor=east] at (-0.1,\y) 
			{\footnotesize \num[round-mode=places,round-precision=2]{\resulty}};
		}
		
		% Koordinatenachsen und Beschriftung mit $x$ und $y$
		
		\draw[-latex,line width=0.2mm] (\xminkosy,0) --(\xmaxkosy,0) node[left, xshift=-0mm, yshift=-8mm]{#5};   
		\draw[-latex,line width=0.2mm] (0,\yminkosy) --(0,\ymaxkosy) node[above]{#6};
		
		#9
		
		\end{tikzpicture}
	\end{center}
}	
\newcommand{\kosycustomiink}[9]{
	\begin{center}
		\begin{tikzpicture}[scale=1,samples=400]
		
		%% Größe des KOSYs %%%%%%%%%%%%%%%%%%%%%%%%%%%%%%%%%%%%%%%%%%
		
		\pgfmathsetmacro{\xmin}{#1}
		\pgfmathsetmacro{\xmax}{#2}
		\pgfmathsetmacro{\ymin}{#3}
		\pgfmathsetmacro{\ymax}{#4}
		
		%%%%%%%%%%%%%%%%%%%%%%%%%%%%%%%%%%%%%%%%%%%%%%%%%%%%%%%%%%%%%
		
		\pgfmathtruncatemacro{\xmintick}{1+\xmin}
		\pgfmathtruncatemacro{\xmaxtick}{\xmax}
		\pgfmathtruncatemacro{\ymintick}{1+\ymin}
		\pgfmathtruncatemacro{\ymaxtick}{\ymax}
		
		\pgfmathtruncatemacro{\xmingrid}{\xmin-1}
		\pgfmathtruncatemacro{\xmaxgrid}{\xmax+1}
		\pgfmathtruncatemacro{\ymingrid}{\ymin-1}
		\pgfmathtruncatemacro{\ymaxgrid}{\ymax+1}
		
		\pgfmathsetmacro{\xminkosy}{\xmin-0.5}
		\pgfmathsetmacro{\xmaxkosy}{\xmax+0.5}
		\pgfmathsetmacro{\yminkosy}{\ymin-0.5}
		\pgfmathsetmacro{\ymaxkosy}{\ymax+0.5}
		
		% Definition des Gitters und das Gitter selbst
		
		\tikzstyle{dotted}= [dash pattern=on 0.15mm off 0.85mm]
		\draw[help lines,line width=0.15mm,step=0.5,style=dotted,black!20,line cap=round] (\xmingrid,\ymingrid) grid (\xmaxgrid,\ymaxgrid);
		
		% Ursprung
		
		\node[below left=2pt] (0,0) {\footnotesize 0};
		
		% Beschriftung der Koordinatenachsen mit Ticks (muss vor den Achsen selbst kommen, sonst sind die Achsen z. T. abgedeckt)
		
		\foreach \x in {\xmintick,...,1,2,3,...,\xmaxtick}
		{
			\pgfmathsetmacro\resultx{#7*\x}
			\draw (\x,0pt) -- (\x,-2pt) node[anchor=north] at (\x,-0.1) 
			{\footnotesize \num[round-mode=places,round-precision=2]{\resultx}};
		}		
		
		\foreach \y in {\ymintick,...,1,2,3,...,\ymaxtick}
		{
			\pgfmathsetmacro\resulty{#8*\y}
			\draw (0pt,\y) -- (-2pt,\y) node[anchor=east] at (-0.1,\y) 
			{\footnotesize \num[round-mode=places,round-precision=2]{\resulty}};
		}
		
		% Koordinatenachsen und Beschriftung mit $x$ und $y$
		
		\draw[-latex,line width=0.2mm] (\xminkosy,0) --(\xmaxkosy,0) node[left, xshift=-0mm, yshift=-3ex]{#5};   
		\draw[-latex,line width=0.2mm] (0,\yminkosy) --(0,\ymaxkosy) node[above]{#6};
		
		#9
		
		\end{tikzpicture}
	\end{center}
}	

\newcommand{\kosycustomdreizweink}[9]{
	\begin{center}
		\begin{tikzpicture}[scale=0.72,samples=400]
			
			%% Größe des KOSYs %%%%%%%%%%%%%%%%%%%%%%%%%%%%%%%%%%%%%%%%%%
			
			\pgfmathsetmacro{\xmin}{#1}
			\pgfmathsetmacro{\xmax}{#2}
			\pgfmathsetmacro{\ymin}{#3}
			\pgfmathsetmacro{\ymax}{#4}
			
			%%%%%%%%%%%%%%%%%%%%%%%%%%%%%%%%%%%%%%%%%%%%%%%%%%%%%%%%%%%%%
			
			\pgfmathtruncatemacro{\xmintick}{1+\xmin}
			\pgfmathtruncatemacro{\xmaxtick}{\xmax}
			\pgfmathtruncatemacro{\ymintick}{1+\ymin}
			\pgfmathtruncatemacro{\ymaxtick}{\ymax}
			
			\pgfmathtruncatemacro{\xmingrid}{\xmin-1}
			\pgfmathtruncatemacro{\xmaxgrid}{\xmax+1}
			\pgfmathtruncatemacro{\ymingrid}{\ymin-1}
			\pgfmathtruncatemacro{\ymaxgrid}{\ymax+1}
			
			\pgfmathsetmacro{\xminkosy}{\xmin-0.5}
			\pgfmathsetmacro{\xmaxkosy}{\xmax+0.5}
			\pgfmathsetmacro{\yminkosy}{\ymin-0.5}
			\pgfmathsetmacro{\ymaxkosy}{\ymax+0.5}
			
			% Definition des Gitters und das Gitter selbst
			
			\tikzstyle{dotted}= [dash pattern=on 0.15mm off 0.85mm]
			\draw[help lines,line width=0.15mm,step=0.5,style=dotted,black!20,line cap=round] (\xmingrid,\ymingrid) grid (\xmaxgrid,\ymaxgrid);
			
			% Ursprung
			
			\node[below left=2pt] (0,0) {\footnotesize 0};
			
			% Beschriftung der Koordinatenachsen mit Ticks (muss vor den Achsen selbst kommen, sonst sind die Achsen z. T. abgedeckt)
			
			\foreach \x in {\xmintick,...,1,2,3,...,\xmaxtick}
			{
				\pgfmathsetmacro\resultx{#7*\x}
				\draw (\x,0pt) -- (\x,-2pt) node[anchor=north] at (\x,-0.1) 
				{\scriptsize \num[round-mode=places,round-precision=3]{\resultx}};
			}		
			
			\foreach \y in {\ymintick,...,1,2,3,...,\ymaxtick}
			{
				\pgfmathsetmacro\resulty{#8*\y}
				\draw (0pt,\y) -- (-2pt,\y) node[anchor=east] at (-0.1,\y) 
				{\scriptsize \num[round-mode=places,round-precision=2]{\resulty}};
			}
			
			% Koordinatenachsen und Beschriftung mit $x$ und $y$
			
			\draw[-latex,line width=0.2mm] (\xminkosy,0) --(\xmaxkosy,0) node[left, xshift=-0mm, yshift=-3ex]{#5};   
			\draw[-latex,line width=0.2mm] (0,\yminkosy) --(0,\ymaxkosy) node[above]{#6};
			
			#9
			
		\end{tikzpicture}
	\end{center}
}	


\newcommand{\kosycustomdreieinsnk}[9]{
%	\begin{center}
		\begin{tikzpicture}[scale=0.72,samples=400]
			
			%% Größe des KOSYs %%%%%%%%%%%%%%%%%%%%%%%%%%%%%%%%%%%%%%%%%%
			
			\pgfmathsetmacro{\xmin}{#1}
			\pgfmathsetmacro{\xmax}{#2}
			\pgfmathsetmacro{\ymin}{#3}
			\pgfmathsetmacro{\ymax}{#4}
			
			%%%%%%%%%%%%%%%%%%%%%%%%%%%%%%%%%%%%%%%%%%%%%%%%%%%%%%%%%%%%%
			
			\pgfmathtruncatemacro{\xmintick}{1+\xmin}
			\pgfmathtruncatemacro{\xmaxtick}{\xmax}
			\pgfmathtruncatemacro{\ymintick}{1+\ymin}
			\pgfmathtruncatemacro{\ymaxtick}{\ymax}
			
			\pgfmathtruncatemacro{\xmingrid}{\xmin-1}
			\pgfmathtruncatemacro{\xmaxgrid}{\xmax+1}
			\pgfmathtruncatemacro{\ymingrid}{\ymin-1}
			\pgfmathtruncatemacro{\ymaxgrid}{\ymax+1}
			
			\pgfmathsetmacro{\xminkosy}{\xmin-0.5}
			\pgfmathsetmacro{\xmaxkosy}{\xmax+0.5}
			\pgfmathsetmacro{\yminkosy}{\ymin-0.5}
			\pgfmathsetmacro{\ymaxkosy}{\ymax+0.5}
			
			% Definition des Gitters und das Gitter selbst
			
			\tikzstyle{dotted}= [dash pattern=on 0.15mm off 0.85mm]
			\draw[help lines,line width=0.15mm,step=0.5,style=dotted,black!20,line cap=round] (\xmingrid,\ymingrid) grid (\xmaxgrid,\ymaxgrid);
			
			% Ursprung
			
			\node[below left=2pt] (0,0) {\footnotesize 0};
			
			% Beschriftung der Koordinatenachsen mit Ticks (muss vor den Achsen selbst kommen, sonst sind die Achsen z. T. abgedeckt)
			
			\foreach \x in {\xmintick,...,1,2,3,...,\xmaxtick}
			{
				\pgfmathsetmacro\resultx{#7*\x}
				\draw (\x,0pt) -- (\x,-2pt) node[anchor=north] at (\x,-0.1) 
				{\scriptsize \num[round-mode=places,round-precision=3]{\resultx}};
			}		
			
			\foreach \y in {\ymintick,...,1,2,3,...,\ymaxtick}
			{
				\pgfmathsetmacro\resulty{#8*\y}
				\draw (0pt,\y) -- (-2pt,\y) node[anchor=east] at (-0.1,\y) 
				{\scriptsize \num[round-mode=places,round-precision=1]{\resulty}};
			}
			
			% Koordinatenachsen und Beschriftung mit $x$ und $y$
			
			\draw[-latex,line width=0.2mm] (\xminkosy,0) --(\xmaxkosy,0) node[left, xshift=-0mm, yshift=-3ex]{#5};   
			\draw[-latex,line width=0.2mm] (0,\yminkosy) --(0,\ymaxkosy) node[above]{#6};
			
			#9
			
		\end{tikzpicture}
%	\end{center}
}	

\newcommand{\kosycustomdreidreink}[9]{
	%	\begin{center}
		\begin{tikzpicture}[scale=0.72,samples=400]
			
			%% Größe des KOSYs %%%%%%%%%%%%%%%%%%%%%%%%%%%%%%%%%%%%%%%%%%
			
			\pgfmathsetmacro{\xmin}{#1}
			\pgfmathsetmacro{\xmax}{#2}
			\pgfmathsetmacro{\ymin}{#3}
			\pgfmathsetmacro{\ymax}{#4}
			
			%%%%%%%%%%%%%%%%%%%%%%%%%%%%%%%%%%%%%%%%%%%%%%%%%%%%%%%%%%%%%
			
			\pgfmathtruncatemacro{\xmintick}{1+\xmin}
			\pgfmathtruncatemacro{\xmaxtick}{\xmax}
			\pgfmathtruncatemacro{\ymintick}{1+\ymin}
			\pgfmathtruncatemacro{\ymaxtick}{\ymax}
			
			\pgfmathtruncatemacro{\xmingrid}{\xmin-1}
			\pgfmathtruncatemacro{\xmaxgrid}{\xmax+1}
			\pgfmathtruncatemacro{\ymingrid}{\ymin-1}
			\pgfmathtruncatemacro{\ymaxgrid}{\ymax+1}
			
			\pgfmathsetmacro{\xminkosy}{\xmin-0.5}
			\pgfmathsetmacro{\xmaxkosy}{\xmax+0.5}
			\pgfmathsetmacro{\yminkosy}{\ymin-0.5}
			\pgfmathsetmacro{\ymaxkosy}{\ymax+0.5}
			
			% Definition des Gitters und das Gitter selbst
			
			\tikzstyle{dotted}= [dash pattern=on 0.15mm off 0.85mm]
			\draw[help lines,line width=0.15mm,step=0.5,style=dotted,black!20,line cap=round] (\xmingrid,\ymingrid) grid (\xmaxgrid,\ymaxgrid);
			
			% Ursprung
			
			\node[below left=2pt] (0,0) {\footnotesize 0};
			
			% Beschriftung der Koordinatenachsen mit Ticks (muss vor den Achsen selbst kommen, sonst sind die Achsen z. T. abgedeckt)
			
			\foreach \x in {\xmintick,...,1,2,3,...,\xmaxtick}
			{
				\pgfmathsetmacro\resultx{#7*\x}
				\draw (\x,0pt) -- (\x,-2pt) node[anchor=north] at (\x,-0.1) 
				{\scriptsize \num[round-mode=places,round-precision=3]{\resultx}};
			}		
			
			\foreach \y in {\ymintick,...,1,2,3,...,\ymaxtick}
			{
				\pgfmathsetmacro\resulty{#8*\y}
				\draw (0pt,\y) -- (-2pt,\y) node[anchor=east] at (-0.1,\y) 
				{\scriptsize \num[round-mode=places,round-precision=3]{\resulty}};
			}
			
			% Koordinatenachsen und Beschriftung mit $x$ und $y$
			
			\draw[-latex,line width=0.2mm] (\xminkosy,0) --(\xmaxkosy,0) node[left, xshift=-0mm, yshift=-3ex]{#5};   
			\draw[-latex,line width=0.2mm] (0,\yminkosy) --(0,\ymaxkosy) node[above]{#6};
			
			#9
			
		\end{tikzpicture}
		%	\end{center}
}	

\newcommand{\kosycustomzweizweink}[9]{
	%	\begin{center}
	\begin{tikzpicture}[scale=1.00,samples=400]
	
	%% Größe des KOSYs %%%%%%%%%%%%%%%%%%%%%%%%%%%%%%%%%%%%%%%%%%
	
	\pgfmathsetmacro{\xmin}{#1}
	\pgfmathsetmacro{\xmax}{#2}
	\pgfmathsetmacro{\ymin}{#3}
	\pgfmathsetmacro{\ymax}{#4}
	
	%%%%%%%%%%%%%%%%%%%%%%%%%%%%%%%%%%%%%%%%%%%%%%%%%%%%%%%%%%%%%
	
	\pgfmathtruncatemacro{\xmintick}{1+\xmin}
	\pgfmathtruncatemacro{\xmaxtick}{\xmax}
	\pgfmathtruncatemacro{\ymintick}{1+\ymin}
	\pgfmathtruncatemacro{\ymaxtick}{\ymax}
	
	\pgfmathtruncatemacro{\xmingrid}{\xmin-1}
	\pgfmathtruncatemacro{\xmaxgrid}{\xmax+1}
	\pgfmathtruncatemacro{\ymingrid}{\ymin-1}
	\pgfmathtruncatemacro{\ymaxgrid}{\ymax+1}
	
	\pgfmathsetmacro{\xminkosy}{\xmin-0.5}
	\pgfmathsetmacro{\xmaxkosy}{\xmax+0.5}
	\pgfmathsetmacro{\yminkosy}{\ymin-0.5}
	\pgfmathsetmacro{\ymaxkosy}{\ymax+0.5}
	
	% Definition des Gitters und das Gitter selbst
	
	\tikzstyle{dotted}= [dash pattern=on 0.15mm off 0.85mm]
	\draw[help lines,line width=0.15mm,step=0.5,style=dotted,black!20,line cap=round] (\xmingrid,\ymingrid) grid (\xmaxgrid,\ymaxgrid);
	
	% Ursprung
	
	\node[below left=2pt] (0,0) {\footnotesize 0};
	
	% Beschriftung der Koordinatenachsen mit Ticks (muss vor den Achsen selbst kommen, sonst sind die Achsen z. T. abgedeckt)
	
	\foreach \x in {\xmintick,...,1,2,3,...,\xmaxtick}
	{
		\pgfmathsetmacro\resultx{#7*\x}
		\draw (\x,0pt) -- (\x,-2pt) node[anchor=north] at (\x,-0.1) 
		{\scriptsize \num[round-mode=places,round-precision=2]{\resultx}};
	}		
	
	\foreach \y in {\ymintick,...,1,2,3,...,\ymaxtick}
	{
		\pgfmathsetmacro\resulty{#8*\y}
		\draw (0pt,\y) -- (-2pt,\y) node[anchor=east] at (-0.1,\y) 
		{\scriptsize \num[round-mode=places,round-precision=2]{\resulty}};
	}
	
	% Koordinatenachsen und Beschriftung mit $x$ und $y$
	
	\draw[-latex,line width=0.2mm] (\xminkosy,0) --(\xmaxkosy,0) node[left, xshift=-0mm, yshift=-3ex]{#5};   
	\draw[-latex,line width=0.2mm] (0,\yminkosy) --(0,\ymaxkosy) node[above]{#6};
	
	#9
	
	\end{tikzpicture}
	%	\end{center}
}	

\newcommand{\kosycustomscale}[9]{
	%	\begin{center}
	\begin{tikzpicture}[scale=#9,samples=400]
	
	%% Größe des KOSYs %%%%%%%%%%%%%%%%%%%%%%%%%%%%%%%%%%%%%%%%%%
	
	\pgfmathsetmacro{\xmin}{#1}
	\pgfmathsetmacro{\xmax}{#2}
	\pgfmathsetmacro{\ymin}{#3}
	\pgfmathsetmacro{\ymax}{#4}
	
	%%%%%%%%%%%%%%%%%%%%%%%%%%%%%%%%%%%%%%%%%%%%%%%%%%%%%%%%%%%%%
	
	\pgfmathtruncatemacro{\xmintick}{1+\xmin}
	\pgfmathtruncatemacro{\xmaxtick}{\xmax}
	\pgfmathtruncatemacro{\ymintick}{1+\ymin}
	\pgfmathtruncatemacro{\ymaxtick}{\ymax}
	
	\pgfmathtruncatemacro{\xmingrid}{\xmin-1}
	\pgfmathtruncatemacro{\xmaxgrid}{\xmax+1}
	\pgfmathtruncatemacro{\ymingrid}{\ymin-1}
	\pgfmathtruncatemacro{\ymaxgrid}{\ymax+1}
	
	\pgfmathsetmacro{\xminkosy}{\xmin-0.5}
	\pgfmathsetmacro{\xmaxkosy}{\xmax+0.5}
	\pgfmathsetmacro{\yminkosy}{\ymin-0.5}
	\pgfmathsetmacro{\ymaxkosy}{\ymax+0.5}
	
	% Definition des Gitters und das Gitter selbst
	
	\tikzstyle{dotted}= [dash pattern=on 0.15mm off 0.85mm]
	\draw[help lines,line width=0.15mm,step=0.5,style=dotted,black!20,line cap=round] (\xmingrid,\ymingrid) grid (\xmaxgrid,\ymaxgrid);
	
	% Ursprung
	
	\node[below left=2pt] (0,0) {\footnotesize 0};
	
	% Beschriftung der Koordinatenachsen mit Ticks (muss vor den Achsen selbst kommen, sonst sind die Achsen z. T. abgedeckt)
	
	\foreach \x in {\xmintick,...,1,2,3,...,\xmaxtick}
	{
		\pgfmathsetmacro\resultx{#7*\x}
		\draw (\x,0pt) -- (\x,-2pt) node[anchor=north] at (\x,-0.1) 
		{\scriptsize \num[round-mode=places,round-precision=0]{\resultx}};
	}		
	
	\foreach \y in {\ymintick,...,1,2,3,...,\ymaxtick}
	{
		\pgfmathsetmacro\resulty{#8*\y}
		\draw (0pt,\y) -- (-2pt,\y) node[anchor=east] at (-0.1,\y) 
		{\scriptsize \num[round-mode=places,round-precision=0]{\resulty}};
	}
	
	% Koordinatenachsen und Beschriftung mit $x$ und $y$
	
	\draw[-latex,line width=0.2mm] (\xminkosy,0) --(\xmaxkosy,0) node[left, xshift=-0mm, yshift=-3ex]{#5};   
	\draw[-latex,line width=0.2mm] (0,\yminkosy) --(0,\ymaxkosy) node[above]{#6};
		
	\end{tikzpicture}
	%	\end{center}
}	

\newcommand{\kosycustommech}[9]{
	%	\begin{center}
	\begin{tikzpicture}[scale=1,samples=400]
	
	%% Größe des KOSYs %%%%%%%%%%%%%%%%%%%%%%%%%%%%%%%%%%%%%%%%%%
	
	\pgfmathsetmacro{\xmin}{#1}
	\pgfmathsetmacro{\xmax}{#2}
	\pgfmathsetmacro{\ymin}{#3}
	\pgfmathsetmacro{\ymax}{#4}
	
	%%%%%%%%%%%%%%%%%%%%%%%%%%%%%%%%%%%%%%%%%%%%%%%%%%%%%%%%%%%%%
	
	\pgfmathtruncatemacro{\xmintick}{1+\xmin}
	\pgfmathtruncatemacro{\xmaxtick}{\xmax}
	\pgfmathtruncatemacro{\ymintick}{1+\ymin}
	\pgfmathtruncatemacro{\ymaxtick}{\ymax}
	
	\pgfmathtruncatemacro{\xmingrid}{\xmin-1}
	\pgfmathtruncatemacro{\xmaxgrid}{\xmax+1}
	\pgfmathtruncatemacro{\ymingrid}{\ymin-1}
	\pgfmathtruncatemacro{\ymaxgrid}{\ymax+1}
	
	\pgfmathsetmacro{\xminkosy}{\xmin-0.5}
	\pgfmathsetmacro{\xmaxkosy}{\xmax+0.5}
	\pgfmathsetmacro{\yminkosy}{\ymin-0.5}
	\pgfmathsetmacro{\ymaxkosy}{\ymax+0.5}
	
	% Definition des Gitters und das Gitter selbst
	
	\tikzstyle{dotted}= [dash pattern=on 0.15mm off 0.85mm]
	\draw[help lines,line width=0.15mm,step=0.5,style=dotted,black!20,line cap=round] (\xmingrid,\ymingrid) grid (\xmaxgrid,\ymaxgrid);
	
	% Ursprung
	
	\node[below left=3pt] (0,0) {\footnotesize 0};
	
	% Beschriftung der Koordinatenachsen mit Ticks (muss vor den Achsen selbst kommen, sonst sind die Achsen z. T. abgedeckt)
	
	\foreach \x in {\xmintick,...,1,2,3,...,\xmaxtick}
	{
		\pgfmathsetmacro\resultx{#7*\x}
		\draw (\x,0pt) -- (\x,-2pt) node[anchor=north] at (\x,-0.1) 
		{\scriptsize \num[round-mode=places,round-precision=#9]{\resultx}};
	}		
	
	\foreach \y in {\ymintick,...,1,2,3,...,\ymaxtick}
	{
		\pgfmathsetmacro\resulty{#8*\y}
		\draw (0pt,\y) -- (-2pt,\y) node[anchor=east] at (-0.1,\y) 
		{\scriptsize \num[round-mode=places,round-precision=#9]{\resulty}};
	}
	
	% Koordinatenachsen und Beschriftung mit $x$ und $y$
	
	\draw[-latex,line width=0.2mm] (\xminkosy,0) --(\xmaxkosy,0) node[xshift=-0mm, yshift=-4.5ex]{#5};   
	\draw[-latex,line width=0.2mm] (0,\yminkosy) --(0,\ymaxkosy) node[yshift=1.5ex]{#6};
	
	\end{tikzpicture}
	%	\end{center}
}	

\newcommand{\kosycustommechleer}[9]{
	%	\begin{center}
	\begin{tikzpicture}[scale=1,samples=400]
	
	%% Größe des KOSYs %%%%%%%%%%%%%%%%%%%%%%%%%%%%%%%%%%%%%%%%%%
	
	\pgfmathsetmacro{\xmin}{#1}
	\pgfmathsetmacro{\xmax}{#2}
	\pgfmathsetmacro{\ymin}{#3}
	\pgfmathsetmacro{\ymax}{#4}
	
	%%%%%%%%%%%%%%%%%%%%%%%%%%%%%%%%%%%%%%%%%%%%%%%%%%%%%%%%%%%%%
	
	\pgfmathtruncatemacro{\xmintick}{1+\xmin}
	\pgfmathtruncatemacro{\xmaxtick}{\xmax}
	\pgfmathtruncatemacro{\ymintick}{1+\ymin}
	\pgfmathtruncatemacro{\ymaxtick}{\ymax}
	
	\pgfmathtruncatemacro{\xmingrid}{\xmin-1}
	\pgfmathtruncatemacro{\xmaxgrid}{\xmax+1}
	\pgfmathtruncatemacro{\ymingrid}{\ymin-1}
	\pgfmathtruncatemacro{\ymaxgrid}{\ymax+1}
	
	\pgfmathsetmacro{\xminkosy}{\xmin-0.5}
	\pgfmathsetmacro{\xmaxkosy}{\xmax+0.5}
	\pgfmathsetmacro{\yminkosy}{\ymin-0.5}
	\pgfmathsetmacro{\ymaxkosy}{\ymax+0.5}
	
	% Definition des Gitters und das Gitter selbst
	
	\tikzstyle{dotted}= [dash pattern=on 0.15mm off 0.85mm]
	\draw[help lines,line width=0.15mm,step=0.5,style=dotted,black!20,line cap=round] (\xmingrid,\ymingrid) grid (\xmaxgrid,\ymaxgrid);
	
	% Ursprung
	
	\node[below left=3pt] (0,0) {\footnotesize 0};
	
	% Beschriftung der Koordinatenachsen mit Ticks (muss vor den Achsen selbst kommen, sonst sind die Achsen z. T. abgedeckt)
	
	\foreach \x in {\xmintick,...,1,2,3,...,\xmaxtick}
	{
		\pgfmathsetmacro\resultx{#7*\x}
		\draw (\x,0pt) -- (\x,-2pt) node[anchor=north] at (\x,-0.1) 
		{\scriptsize \textcolor{white}{\num[round-mode=places,round-precision=#9]{\resultx}}};
	}		
	
	\foreach \y in {\ymintick,...,1,2,3,...,\ymaxtick}
	{
		\pgfmathsetmacro\resulty{#8*\y}
		\draw (0pt,\y) -- (-2pt,\y) node[anchor=east] at (-0.1,\y) 
		{\scriptsize \textcolor{white}{\num[round-mode=places,round-precision=#9]{\resulty}}};
	}
	
	% Koordinatenachsen und Beschriftung mit $x$ und $y$
	
	\draw[-latex,line width=0.2mm] (\xminkosy,0) --(\xmaxkosy,0) node[xshift=-1.5ex, yshift=-4.5ex]{#5};   
	\draw[-latex,line width=0.2mm] (0,\yminkosy) --(0,\ymaxkosy) node[yshift=1.5ex]{#6};
	
	\end{tikzpicture}
	%	\end{center}
}	


\newcommand{\D}{\displaystyle}

\usepackage[b]{esvect}	%https://www.ctan.org/pkg/esvect

%\usepackage{nccmath} %nccmath adds an optional argument to  \intertext, which is added to the vertical spacing of this command. Macht einen Fehler bei Lückentext-Formeln (werdn zu hoch gezogen...)

%\tikzset{circuit declare symbol = AC source}
%\tikzset{AC source IEC graphic/.style={
%		circuit symbol lines,
%		circuit symbol size=width 2 height 2,
%		shape=generic circle IEC,
%		/pgf/generic circle IEC/before background={
%			\pgfpathmoveto{\pgfpoint{-0.8pt}{0pt}}
%			\pgfpathsine{\pgfpoint{0.4pt}{0.4pt}}
%			\pgfpathcosine{\pgfpoint{0.4pt}{-0.4pt}}
%			\pgfpathsine{\pgfpoint{0.4pt}{-0.4pt}}
%			\pgfpathcosine{\pgfpoint{0.4pt}{0.4pt}}
%			\pgfusepath{stroke}
%		},
%		transform shape, draw
%	}
%}
%\tikzset{circuit ee IEC/.append style=
%	{set AC source graphic = AC source IEC graphic}
%}

% für dicke Bruchstriche:
\newcommand{\thickfrac}[2]{\genfrac{}{}{1pt}{}{#1}{#2}}