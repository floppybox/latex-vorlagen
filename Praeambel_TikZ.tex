
%\usepackage[upright]{fourier}				% Für aufrechte Labels der Punkte in tkz-euclide - Quelle: https://tikz.fr/tkz-euclide-simple-example/

\newcommand{\pfeilrunterbeschriftungrechts}[1]{\begin{tikzpicture} \draw[-Stealth, ultra thick] (0,0.5) -- node [right, xshift=2em] {#1} (0,-0.5); \end{tikzpicture}}

\usepackage{nicematrix} % Für schöne Tabellen mit dem Befehl \begin{NiceTabular}{|c|c|c|}[cell-space-limits=4pt] ... \end{NiceTabular}

\usepackage{tkz-euclide}

\usepackage[compat=1.1.0]{tikz-feynman}

\usepackage{tikz}							% Create PostScript and PDF graphics in TEX
	\usetikzlibrary{3d, angles, arrows, arrows.meta, babel, backgrounds, bending, calc, circuits.logic.US, decorations.markings, decorations.pathmorphing, decorations.pathreplacing, decorations.text, decorations, er, fit, graphs, intersections, matrix, mindmap, patterns, plotmarks, positioning, quotes, svg.path, shadows.blur, shapes.misc, tikzmark, trees, shapes, shadows, through, circuits.logic.US, circuits.logic.IEC, circuits.logic.CDH, circuits.ee.IEC}

\usepackage{tikz-qtree}

%%\usepackage[upright]{fourier} 
%\usepackage{fourier} 

\usetikzlibrary{positioning}

\usepackage[american, europeanresistors,americanvoltages,emptydiodes,siunitx]{circuitikz}
	\tikzset{circuit declare symbol = ammeter}
	\tikzset{set ammeter graphic ={draw,generic circle IEC, minimum size=5.5mm,info=center:A}}
	\tikzset{circuit declare symbol = voltmeter}
	\tikzset{set voltmeter graphic ={draw,generic circle IEC, minimum size=5.5mm,info=center:V}}

\usepackage{pgf}
\usepgflibrary{decorations.text}
\usepackage{pgfplots}
\pgfplotsset{compat=newest}

%  Für Schraffur (https://tex.stackexchange.com/questions/164773/graphics-area-between-curves)
\usepgfplotslibrary{fillbetween}
\usetikzlibrary{patterns}

	
\makeatletter
\tikzset{%
	remember picture with id/.style={%
		remember picture,
		overlay,
		save picture id=#1,
	},
	save picture id/.code={%
		\edef\pgf@temp{#1}%
		\immediate\write\pgfutil@auxout{%
			\noexpand\savepointas{\pgf@temp}{\pgfpictureid}}%
	},
	if picture id/.code args={#1#2#3}{%
		\@ifundefined{save@pt@#1}{%
			\pgfkeysalso{#3}%
		}{
			\pgfkeysalso{#2}%
		}
	}
}

\def\savepointas#1#2{%
	\expandafter\gdef\csname save@pt@#1\endcsname{#2}%
}

\def\tmk@labeldef#1,#2\@nil{%
	\def\tmk@label{#1}%
	\def\tmk@def{#2}%
}

\tikzdeclarecoordinatesystem{pic}{%
	\pgfutil@in@,{#1}%
	\ifpgfutil@in@%
	\tmk@labeldef#1\@nil
	\else
	\tmk@labeldef#1,(0pt,0pt)\@nil
	\fi
	\@ifundefined{save@pt@\tmk@label}{%
		\tikz@scan@one@point\pgfutil@firstofone\tmk@def
	}{%
		\pgfsys@getposition{\csname save@pt@\tmk@label\endcsname}\save@orig@pic%
		\pgfsys@getposition{\pgfpictureid}\save@this@pic%
		\pgf@process{\pgfpointorigin\save@this@pic}%
		\pgf@xa=\pgf@x
		\pgf@ya=\pgf@y
		\pgf@process{\pgfpointorigin\save@orig@pic}%
		\advance\pgf@x by -\pgf@xa
		\advance\pgf@y by -\pgf@ya
	}%
}
\makeatother

\newcommand{\tikzmarkindouble}[2][]{%
	\tikz[remember picture,overlay,baseline=1ex]
	\draw[line width=0.5pt,#1,double]
	(pic cs:#2) -- (0,0)
	;}

\newcommand{\tikzmarkin}[2][]{%
	\tikz[remember picture,overlay,baseline=1ex]
	\draw[line width=0.5pt,#1]
	(pic cs:#2) -- (0,0)
	;}

\newcommand\tikzmarkend[2][]{%
	\tikz[remember picture with id=#2,baseline=1ex] 
	#1;}

\newcommand{\tikzmarkindoublebaseline}[3][]{%
	\tikz[remember picture,overlay,baseline=#3ex]
	\draw[line width=0.5pt,#1,double]
	(pic cs:#2) -- (0,0)
	;}

\newcommand{\tikzmarkinbaseline}[3][]{%
	\tikz[remember picture,overlay,baseline=#3ex]
	\draw[line width=0.5pt,#1]
	(pic cs:#2) -- (0,0)
	;}

\newcommand\tikzmarkendbaseline[3][]{%
	\tikz[remember picture with id=#2,baseline=#3ex] 
	#1;}


\newcommand{\UpArrow}{\mathord{~\begin{tikzpicture}[baseline=0ex, line width=0.7, scale=0.12, ->, >=latex]
		\draw (0,0) -- (0,3);
		\end{tikzpicture}}}

\newcommand{\DownArrow}{\mathord{~\begin{tikzpicture}[baseline=0ex, line width=0.7, scale=0.12, <-, >=latex]
		\draw (0,0) -- (0,3);
		\end{tikzpicture}}}
	
\newcommand{\RightDownArrow}{\mathord{~\begin{tikzpicture}[baseline=0ex, line width=0.7, scale=0.12, ->, >=latex]
		\draw (0,3) -- (3,0);
		\draw (1,4) -- (4,1);
		\end{tikzpicture}}}

\newcommand{\RightUpArrow}{\mathord{~\begin{tikzpicture}[baseline=0ex, line width=0.7, scale=0.12, <-, >=latex]
		\draw (0,0) -- (0,3);
		\end{tikzpicture}}}

%%%%%%%%%%%%%%%%%%%%%%%%%%%
%% LowUpDep %%%%%%%%%%%%%%%%%%%%
%\tikzset{dependent/.style={annotation arrow/.style = {>=}}}
%%
%%
%%LowDep %%%%%
%\tikzset{LowDep/.style args={#1}{
%		append after command={%
%			\bgroup
%			[current point is local=true]
%			[every LowDep/.try]
%			[annotation arrow,-]
%			(-2.5\tikzcircuitssizeunit,-1.5\tikzcircuitssizeunit) edge[line to]
%			(-1.5\tikzcircuitssizeunit,-1.5\tikzcircuitssizeunit) node[xshift=3.0\tikzcircuitssizeunit]{#1}
%			\egroup%
%	}},
%	%
%	LowDep'/.style args={#1}{
%		append after command={%
%			\bgroup
%			[current point is local=true, yscale=-1]
%			[every LowDep/.try]
%			[annotation arrow,-]
%			(-2.5\tikzcircuitssizeunit,-1.5\tikzcircuitssizeunit) edge[line to]
%			(-1.5\tikzcircuitssizeunit,-1.5\tikzcircuitssizeunit) node[xshift=3.0\tikzcircuitssizeunit]{#1}
%			\egroup%
%	}}
%}
%%
%%
%%
%%UpDep %%%%%
%\tikzset{UpDep/.style args={#1}{
%		append after command={%
%			\bgroup
%			[current point is local=true]
%			[every UpDep/.try]
%			[annotation arrow,-]
%			%
%			(2.5\tikzcircuitssizeunit,1.5\tikzcircuitssizeunit)  edge[line to]
%			(1.5\tikzcircuitssizeunit,1.5\tikzcircuitssizeunit) node[xshift=-3.0\tikzcircuitssizeunit]{#1}
%			\egroup%
%	}},
%	%
%	UpDep'/.style args={#1}{
%		append after command={%
%			\bgroup
%			[current point is local=true, yscale=-1]
%			[every UpDep/.try]
%			[annotation arrow,-]
%			%
%			(2.5\tikzcircuitssizeunit,1.5\tikzcircuitssizeunit)  edge[line to]
%			(1.5\tikzcircuitssizeunit,1.5\tikzcircuitssizeunit) node[xshift=-3.0\tikzcircuitssizeunit]{#1}
%			\egroup%
%	}}
%}
%%%%%%%%%%%%%%%%%%%%%%%%%%%%


%\newcommand\circlearound[1]{\tikz[baseline]\node[draw,shape=circle,anchor=base] {#1} ;}

%\usepackage{relsize} % für {\larger\textcircled{\smaller[2]1}}

\newcommand*\circled[1]{\tikz[baseline=(char.base)]{ \node[shape=circle,draw,inner sep=2pt, minimum width=3ex, line width=0.75pt] (char) {\textsf{#1}};}}


%my-Meter - eigenes Messgerät zum Eintragen:
\tikzset{circuit declare symbol = mymeter}
\tikzset{set mymeter graphic ={draw,generic circle IEC, minimum size=7mm,info=center:}}


%\tikzset{circuit declare symbol = AC source}
%\tikzset{AC source IEC graphic/.style={
%		circuit symbol lines,
%		circuit symbol size=width 2 height 2,
%		shape=generic circle IEC,
%		/pgf/generic circle IEC/before background={
%			\pgfpathmoveto{\pgfpoint{-0.8pt}{0pt}}
%			\pgfpathsine{\pgfpoint{0.4pt}{0.4pt}}
%			\pgfpathcosine{\pgfpoint{0.4pt}{-0.4pt}}
%			\pgfpathsine{\pgfpoint{0.4pt}{-0.4pt}}
%			\pgfpathcosine{\pgfpoint{0.4pt}{0.4pt}}
%			\pgfusepath{stroke}
%		},
%		transform shape, draw
%	}
%}
%\tikzset{circuit ee IEC/.append style=
%	{set AC source graphic = AC source IEC graphic}
%}
%
%\tikzset{circuit declare symbol = AC generator}
%\tikzset{set AC generator graphic ={draw, minimum size=5.5mm,transform shape, info=center:{$\underset{\mathbf{\sim}}{\mathsf{G}}$}}}
%
%%%%%%%%%%%%%%%%%%%%%%%%%%%%%%%%%%%




%%%%% geschweifte Klammern in Tabellen - Aternative		

\usepackage{array, multirow, bigdelim, makecell, booktabs}
%& \hspace{-10em}\rdelim\}{3.5}{*}[\quad $\Rightarrow W_\text{th}  = \SI{4,182}{\kilo\joule}$] 

%%%%% geschweifte Klammern in Tabellen - Aternative		

% Zum extrahieren von Punktkoordinaten und Weiterverwendung
\newdimen\XCoord
\newdimen\YCoord
\newcommand*{\ExtractCoordinate}[1]{\path (#1); \pgfgetlastxy{\XCoord}{\YCoord};}


% für eine schöne Spule (wird aber aktuell nicht genutzt)
% \draw (0,-1) -- node[inductor={0}]{} ++(2,0) node[right]{\verb|inductor={0}|};
% Quelle: https://tex.stackexchange.com/questions/212077/circuitikz-inductor-style

\tikzset{
	inductor/.style args={#1}{
		sloped,
		allow upside down,
		minimum height=4mm,
		minimum width=10mm,
		path picture={%
			\pgfmathsetmacro{\addangle}{#1}
			\pgfmathsetmacro{\radius}{.8/(2+6*cos(\addangle))}
			\draw[white] (-4.5mm,0) -- (4.5mm,0);
			\draw[line join=bevel] (-5mm,0)-- (-4mm,0)
			arc (180:-\addangle:\radius)
			arc (180+\addangle:-\addangle:\radius)
			arc (180+\addangle:-\addangle:\radius)
			arc (180+\addangle:0:\radius)
			-- (5mm,0);
		}
	},
	inductor/.default={40}
}

\tikzstyle{background grid}=[draw, black!20,step=1*5mm]

%TikZ Grid Paper With Node Coordinates

\makeatletter% from https://tex.stackexchange.com/a/39698/121799
\def\grd@save@target#1{%
	\def\grd@target{#1}}
\def\grd@save@start#1{%
	\def\grd@start{#1}}
\tikzset{
	labeled grid/.style={
		to path={%
			\pgfextra{%
				\edef\grd@@target{(\tikztotarget)}%
				\tikz@scan@one@point\grd@save@target\grd@@target\relax
				\edef\grd@@start{(\tikztostart)}%
				\tikz@scan@one@point\grd@save@start\grd@@start\relax
				\draw[minor help lines] (\tikztostart) grid (\tikztotarget);
				\draw[major help lines] (\tikztostart) grid (\tikztotarget);
				\grd@start
				\pgfmathsetmacro{\grd@xa}{\the\pgf@x/1cm}
				\pgfmathsetmacro{\grd@ya}{\the\pgf@y/1cm}
				\grd@target
				\pgfmathsetmacro{\grd@xb}{\the\pgf@x/1cm}
				\pgfmathsetmacro{\grd@yb}{\the\pgf@y/1cm}
				\pgfmathsetmacro{\grd@xc}{\grd@xa + \pgfkeysvalueof{/tikz/grid with coordinates/major step}}
				\pgfmathsetmacro{\grd@yc}{\grd@ya + \pgfkeysvalueof{/tikz/grid with coordinates/major step}}
				%\foreach \x in {\grd@xa,\grd@xc,...,\grd@xb}
				%\node[anchor=north] at (\x,\grd@ya) {\pgfmathprintnumber{\x}};
				%\foreach \y in {\grd@ya,\grd@yc,...,\grd@yb}
				%\node[anchor=east] at (\grd@xa,\y) {\pgfmathprintnumber{\y}};
				\path foreach \x in {\grd@xa,\grd@xc,...,\grd@xb}
				{foreach \y in {\grd@ya,\grd@yc,...,\grd@yb}
					{ (\x,\y) node[grid with coordinates/grid label] {$(\pgfmathprintnumber{\x},\pgfmathprintnumber{\y})$}}};
			}
		}
	},
	minor help lines/.style={
		help lines,
		step=\pgfkeysvalueof{/tikz/grid with coordinates/minor step},
		draw=none
	},
	major help lines/.style={
		help lines,
		line width=\pgfkeysvalueof{/tikz/grid with coordinates/major line width},
		step=\pgfkeysvalueof{/tikz/grid with coordinates/major step},
		draw=none
	},
	grid with coordinates/.cd,
	minor step/.initial=.2,
	major step/.initial=1,
	major line width/.initial=0pt,
	grid label/.style={below,scale=0.3,color = black!20}
}
\makeatother

\newcommand{\lamp}{\!
\begin{tikzpicture}[scale=0.5, every node/.style={transform shape}]
\draw (0,0) to [lamp] (2,0);
\end{tikzpicture}\; }

\tikzstyle{background grid}=[draw, black!20,step=1*5mm]

\makeatletter
\pgfdeclareshape{neon lamp shape}
{
	\inheritsavedanchors[from=circle ee]
	\inheritanchor[from=circle ee]{center}
	\inheritanchor[from=circle ee]{north}
	\inheritanchor[from=circle ee]{south}
	\inheritanchor[from=circle ee]{east}
	\inheritanchor[from=circle ee]{west}
	\inheritanchor[from=circle ee]{north east}
	\inheritanchor[from=circle ee]{north west}
	\inheritanchor[from=circle ee]{south east}
	\inheritanchor[from=circle ee]{south west}
	\inheritanchor[from=circle ee]{input}
	\inheritanchor[from=circle ee]{output}
	\inheritanchorborder[from=circle ee]
	
	\backgroundpath{
		\pgf@process{\radius}
		\pgfutil@tempdima=\radius
		
		\pgfpathcircle{\centerpoint}{\pgfutil@tempdima}
		
		\pgfpathmoveto{\pgfpoint{-\pgfutil@tempdima}{0pt}}
		\pgfpathlineto{\pgfpoint{-0.4\pgfutil@tempdima}{0pt}}
		\pgfpathmoveto{\pgfpoint{-0.4\pgfutil@tempdima}{0.6\pgfutil@tempdima}}
		\pgfpathlineto{\pgfpoint{-0.4\pgfutil@tempdima}{-0.6\pgfutil@tempdima}}
		
		\pgfpathmoveto{\pgfpoint{\pgfutil@tempdima}{0pt}}
		\pgfpathlineto{\pgfpoint{0.4\pgfutil@tempdima}{0pt}}
		\pgfpathmoveto{\pgfpoint{0.4\pgfutil@tempdima}{0.6\pgfutil@tempdima}}
		\pgfpathlineto{\pgfpoint{0.4\pgfutil@tempdima}{-0.6\pgfutil@tempdima}}
		\pgfusepath{stroke}
		
		\pgfpathcircle{\pgfqpoint{0\pgfutil@tempdima}{0.55\pgfutil@tempdima}}{.1\pgfutil@tempdima}
		\pgfusepath{stroke}
	}
}
\makeatother

\newcommand{\mathdirectcurrent}{\mathrel{\mathpalette\mathdirectcurrentinner\relax}}
\newcommand{\mathdirectcurrentinner}[2]{%
	\settowidth{\dimen0}{$#1=$}%
	\vbox to .85ex {\offinterlineskip
		\hbox to \dimen0{\hss\leaders\hrule height 1pt\hskip.85\dimen0\hss}
		\vskip.35ex
		\hbox to \dimen0{\hss
			\leaders\hrule height 1pt\hskip.34\dimen0
			\hskip.17\dimen0
			\leaders\hrule height 1pt\hskip.34\dimen0
			%			\hskip.17\dimen0
			%			\leaders\hrule\hskip.17\dimen0
			\hss}
		%		\vfill
	}%
}
\newcommand{\textdirectcurrent}{\mathdirectcurrentinner{\textstyle}{}}

\newcommand\Ground{%
	\mathbin{\text{\begin{tikzpicture}[circuit ee IEC,yscale=0.6,xscale=0.5]
			\draw (0,2ex) to (0,0) node[ground,rotate=-90,xshift=.65ex] {};
			\end{tikzpicture}}}%
}


% schöne Pfeilspitzen
\tikzset{%
	>=latex,
	inner sep=0pt,%
	outer sep=0pt,%         Das ist auch der Abstand zwischen Messgeräten und Leitungen!!!
	mark coordinate/.style={inner sep=0pt,outer sep=0pt,minimum size=3pt,
		fill=black,circle}%
}


\newcommand\pgfmathsinandcos[3]{%
	\pgfmathsetmacro#1{sin(#3)}%
	\pgfmathsetmacro#2{cos(#3)}%
}
\newcommand\LongitudePlane[3][current plane]{%
	\pgfmathsinandcos\sinEl\cosEl{#2} % elevation
	\pgfmathsinandcos\sint\cost{#3} % azimuth
	\tikzset{#1/.style={cm={\cost,\sint*\sinEl,0,\cosEl,(0,0)}}}
}

\newcommand\LatitudePlane[3][current plane]{%
	\pgfmathsinandcos\sinEl\cosEl{#2} % elevation
	\pgfmathsinandcos\sint\cost{#3} % latitude
	\pgfmathsetmacro\yshift{\RadiusSphere*\cosEl*\sint}
	\tikzset{#1/.style={cm={\cost,0,0,\cost*\sinEl,(0,\yshift)}}} %
}
\newcommand\NewLatitudePlane[4][current plane]{%
	\pgfmathsinandcos\sinEl\cosEl{#3} % elevation
	\pgfmathsinandcos\sint\cost{#4} % latitude
	\pgfmathsetmacro\yshift{#2*\cosEl*\sint}
	\tikzset{#1/.style={cm={\cost,0,0,\cost*\sinEl,(0,\yshift)}}} %
}
\newcommand\DrawLongitudeCircle[2][1]{
	\LongitudePlane{\angEl}{#2}
	\tikzset{current plane/.prefix style={scale=#1}}
	% angle of "visibility"
	\pgfmathsetmacro\angVis{atan(sin(#2)*cos(\angEl)/sin(\angEl))} %
	\draw[current plane] (\angVis:1) arc (\angVis:\angVis+180:1);
	\draw[current plane,opacity=0.4] (\angVis-180:1) arc (\angVis-180:\angVis:1);
}
\newcommand\DrawLongitudeArc[4][black]{
	\LongitudePlane{\angEl}{#2}
	\tikzset{current plane/.prefix style={scale=1}}
	\pgfmathsetmacro\angVis{atan(sin(#2)*cos(\angEl)/sin(\angEl))} %
	\pgfmathsetmacro\angA{mod(max(\angVis,#3),360)} %
	\pgfmathsetmacro\angB{mod(min(\angVis+180,#4),360} %
	\draw[current plane,#1,opacity=0.4] (#3:\RadiusSphere) arc (#3:#4:\RadiusSphere);
	\draw[current plane,#1]  (\angA:\RadiusSphere) arc (\angA:\angB:\RadiusSphere);
}%
\newcommand\DrawLatitudeCircle[2][1]{
	\LatitudePlane{\angEl}{#2}
	\tikzset{current plane/.prefix style={scale=#1}}
	\pgfmathsetmacro\sinVis{sin(#2)/cos(#2)*sin(\angEl)/cos(\angEl)}
	% angle of "visibility"
	\pgfmathsetmacro\angVis{asin(min(1,max(\sinVis,-1)))}
	\draw[current plane] (\angVis:1) arc (\angVis:-\angVis-180:1);
	\draw[current plane,opacity=0.4] (180-\angVis:1) arc (180-\angVis:\angVis:1);
}

\newcommand\DrawLatitudeArc[4][black]{
	\LatitudePlane{\angEl}{#2}
	\tikzset{current plane/.prefix style={scale=1}}
	\pgfmathsetmacro\sinVis{sin(#2)/cos(#2)*sin(\angEl)/cos(\angEl)}
	% angle of "visibility"
	\pgfmathsetmacro\angVis{asin(min(1,max(\sinVis,-1)))}
	\pgfmathsetmacro\angA{max(min(\angVis,#3),-\angVis-180)} %
	\pgfmathsetmacro\angB{min(\angVis,#4)} %
	\draw[current plane,#1,opacity=0.4] (#3:\RadiusSphere) arc (#3:#4:\RadiusSphere);
	\draw[current plane,#1] (\angA:\RadiusSphere) arc (\angA:\angB:\RadiusSphere);
}


% Fading:
\makeatletter
\tikzset{ % based on https://tex.stackexchange.com/q/328433/121799
	/tikz/render blur shadow/.code={
		\pgfbs@savebb
		\pgfsyssoftpath@getcurrentpath{\pgfbs@input@path}%
		\pgfbs@compute@shadow@bbox
		\pgfbs@process@rounding{\pgfbs@input@path}{\pgfbs@fadepath}%
		\pgfbs@apply@canvas@transform
		\colorlet{pstb@shadow@color}{white!\pgfbs@opacity!\my@shadow@color}%
		\pgfdeclarefading{shadowfading}{\pgfbs@paint@fading}%
		\pgfsetfillcolor{\my@shadow@color}%
		\pgfsetfading{shadowfading}%
		{\pgftransformshift{\pgfpoint{\pgfbs@midx}{\pgfbs@midy}}}%
		\pgfbs@usebbox{fill}%
		\pgfbs@restorebb
	},}
\tikzset{
	/tikz/shadow color/.store in=\my@shadow@color,
	/tikz/shadow color=gray,
}
\makeatother

% Text entlang Winkel-Markierung nach 
%	https://tex.stackexchange.com/questions/234826/build-text-path-using-pic-from-library-angles

\newcommand{\escala}{.02} % escala
\tikzset{MeuAngulo/.style={draw=blue,<-,angle eccentricity=1.2,angle radius=25pt,semithick}}

\makeatletter
\tikzset{
	pics/deco angle/.style n args = 2{
		setup code  = \tikz@lib@angle@parse#1\pgf@stop,
		background code = \tikz@lib@angle@background#1\pgf@stop,
		foreground code = \tikz@lib@decoangle@foreground#1{#2}\pgf@stop,
	},
	pics/angle/.default=A--B--C,
	angle eccentricity/.initial=.6,
	angle radius/.initial=5mm
}

\def\tikz@lib@decoangle@foreground#1--#2--#3#4\pgf@stop{%
	\path [name prefix ..] [pic actions, fill=none, shade=none]
	([shift={(\tikz@start@angle@temp:\tikz@lib@angle@rad pt)}]#2.center)
	arc [start angle=\tikz@start@angle@temp, end
	angle=\tikz@end@angle@temp, radius=\tikz@lib@angle@rad pt];
	\path [name prefix ..] [fill=none, shade=none, decorate, decoration={text along path, text align=center, text={#4}}]
	([shift={(\tikz@end@angle@temp:\tikz@lib@angle@rad+2 pt)}]#2.center)
	arc [start angle=\tikz@end@angle@temp, end angle=\tikz@start@angle@temp, radius=\tikz@lib@angle@rad+2 pt];
}
\makeatother
% und jetzt noch für Text auf dem Kopf...
\makeatletter
\tikzset{
	pics/deco angle usd/.style n args = 2{
		setup code  = \tikz@lib@angle@parse#1\pgf@stop,
		background code = \tikz@lib@angle@background#1\pgf@stop,
		foreground code = \tikz@lib@decoangle@foreground#1{#2}\pgf@stop,
	},
	pics/angle/.default=A--B--C,
	angle eccentricity/.initial=.6,
	angle radius/.initial=5mm
}

\def\tikz@lib@decoangle@foreground#1--#2--#3#4\pgf@stop{%
	\path [name prefix ..] [pic actions, fill=none, shade=none] ([shift={(\tikz@start@angle@temp:\tikz@lib@angle@rad pt)}]#2.center)
	arc [start angle=\tikz@start@angle@temp, end angle=\tikz@end@angle@temp, radius=\tikz@lib@angle@rad pt];
	\path [name prefix ..] [fill=none, shade=none, decorate, decoration={text along path, text align=center, text={$|\scriptstyle|~#4~||$}}]
	([shift={(\tikz@end@angle@temp:\tikz@lib@angle@rad+3 pt)}]#2.center)
	arc [start angle=\tikz@end@angle@temp, end angle=\tikz@start@angle@temp, radius=\tikz@lib@angle@rad+2 pt];
}
\makeatother

%%%%%%%%%%%%%%% Highlighter
%%%%%%%%%%%%%%% Quelle. https://tex.stackexchange.com/questions/5959/cool-text-highlighting-in-latex

\usepackage{soulutf8}
\usepackage{atbegshi}
\usepackage{etoolbox}

\usetikzlibrary{tikzmark,calc,decorations.pathmorphing}

\colorlet{tdcolor}{yellow!35}

\makeatletter

\newlength{\txtdec@depth}
\setlength{\txtdec@depth}{.5ex}
\newlength{\txtdec@height}
\setlength{\txtdec@height}{\f@size pt} % sensible default

\newcounter{txtdec@hyphmark}
\newcounter{txtdec@decormark}
\setcounter{txtdec@decormark}{1}
\newcounter{txtdec@hyphdraw}
\setcounter{txtdec@hyphdraw}{1}
\newcounter{txtdec@decordraw}
\setcounter{txtdec@decordraw}{1}

\newcounter{txtdec@thenextpage}

\newtoggle{txtdec@unfinisheddecor}
\newtoggle{txtdec@stayonpage}
\newtoggle{txtdec@stayondecor}
\newtoggle{txtdec@stayonline}

% from https://tex.stackexchange.com/a/33765/105447
\newcommand{\gettikzxy}[3]{%
	\tikz@scan@one@point\pgfutil@firstofone#1\relax
	\edef#2{\the\pgf@x}%
	\edef#3{\the\pgf@y}%
}

% the drawing macros

\tikzset{%
	defaultdecor/.style={%
		fill=tdcolor,
		decoration = {random steps, amplitude=0.5pt, segment length=15pt},
		outer sep = -15pt,
		inner sep = 0pt,
		decorate}%
}

\newcommand{\txtdec@draw@all}{%
	\tikzset{thisdecor/.style/.expanded=\csuse{decor@tikz@style@\thetxtdec@decordraw}}%
	\path[defaultdecor, thisdecor]
	($(\Xbegin,\Ybegin)+(0,-\txtdec@depth)$) rectangle
	($(\Xend,\Yend)+(0,\txtdec@height-\txtdec@depth)$) ;
}

\newcommand{\txtdec@draw@begin}{%
	\tikzset{thisdecor/.style/.expanded=\csuse{decor@tikz@style@\thetxtdec@decordraw}}%
	\path[defaultdecor, thisdecor]
	($(\Xbegin,\Ybegin)+(0,-\txtdec@depth)$) rectangle
	($(\Xlineend,\Ylineend)+(0,\txtdec@height-\txtdec@depth)$) ;
}

\newcommand{\txtdec@draw@middle}{%
	\tikzset{thisdecor/.style/.expanded=\csuse{decor@tikz@style@\thetxtdec@decordraw}}%
	\path[defaultdecor, thisdecor]
	($(\Xlinebegin,\Ylinebegin)+(0,-\txtdec@depth)$) rectangle
	($(\Xlineend,\Ylineend)+(0,\txtdec@height-\txtdec@depth)$) ;
}

\newcommand{\txtdec@draw@end}{%
	\tikzset{thisdecor/.style/.expanded=\csuse{decor@tikz@style@\thetxtdec@decordraw}}%
	\path[defaultdecor, thisdecor]
	($(\Xlinebegin,\Ylinebegin)+(0,-\txtdec@depth)$) rectangle
	($(\Xend,\Yend)+(0,\txtdec@height-\txtdec@depth)$) ;
}


% using soul to set tikzmarks

\def\SOUL@tdleaders{%
	\stepcounter{txtdec@hyphmark}%
	\tikzmark{p\thepage.d\arabic{txtdec@decormark}.\arabic{txtdec@hyphmark}}%
	\leaders\hrule\@depth\z@\@height\z@\relax
}

\def\SOUL@tdunderline#1{{%
		\setbox\z@\hbox{#1}%
		\dimen@=\wd\z@
		\dimen@i=\SOUL@uloverlap
		\advance\dimen@2\dimen@i
		\rlap{%
			\null
			\kern-\dimen@i
			\SOUL@ulcolor{\SOUL@tdleaders\hskip\dimen@}%
			\hskip\dimen@
		}%
		\unhcopy\z@
}}

\def\SOUL@tdpreamble{%
	\spaceskip\SOUL@spaceskip
	\setcounter{txtdec@hyphmark}{0}%
	\tikzmark{p\thepage.d\arabic{txtdec@decormark}.begin}%
}
\def\SOUL@tdeverysyllable{%
	\SOUL@tdunderline{%
		\the\SOUL@syllable
		\SOUL@setkern\SOUL@charkern
	}%
	\stepcounter{txtdec@hyphmark}%
	\tikzmark{p\thepage.d\arabic{txtdec@decormark}.\arabic{txtdec@hyphmark}}%
}
\def\SOUL@tdeveryhyphen{%
	\discretionary{%
		\unkern
		\SOUL@tdunderline{%
			\SOUL@setkern\SOUL@hyphkern
			\SOUL@sethyphenchar
		}%
		\stepcounter{txtdec@hyphmark}%
		\tikzmark{p\thepage.d\arabic{txtdec@decormark}.\arabic{txtdec@hyphmark}}%
	}{}{}%
}
\def\SOUL@tdeveryexhyphen#1{%
	\SOUL@setkern\SOUL@hyphkern
	\SOUL@tdunderline{#1}%
	\stepcounter{txtdec@hyphmark}%
	\tikzmark{p\thepage.d\arabic{txtdec@decormark}.\arabic{txtdec@hyphmark}}%
	\discretionary{}{}{%
		\SOUL@setkern\SOUL@charkern
	}%
}
\def\SOUL@tdpostamble{%
	% create an extra mark, vertically displaced, to create an exit condition for the last line
	\stepcounter{txtdec@hyphmark}%
	\raisebox{-5pt}{\tikzmark{p\thepage.d\arabic{txtdec@decormark}.\arabic{txtdec@hyphmark}}}%
	\tikzmark{p\thepage.d\arabic{txtdec@decormark}.end}%
	\stepcounter{txtdec@decormark}%
}
\def\SOUL@tdsetup{%
	\SOUL@setup
	\let\SOUL@preamble\SOUL@tdpreamble
	\let\SOUL@everysyllable\SOUL@tdeverysyllable
	\let\SOUL@everyhyphen\SOUL@tdeveryhyphen
	\let\SOUL@everyexhyphen\SOUL@tdeveryexhyphen
	\let\SOUL@postamble\SOUL@tdpostamble
}
\DeclareRobustCommand*\textdecor[1][]{%
	\csxdef{decor@tikz@style@\thetxtdec@decormark}{#1}%
	\csxdef{decor@fsize@\thetxtdec@decormark}{\f@size pt}%
	\SOUL@tdsetup\SOUL@}


% get the drawing done AtBeginShipout

\AtBeginShipout{%
	\AtBeginShipoutUpperLeft{%
		% getting the number of the next page
		\setcounter{txtdec@thenextpage}{\thepage}%
		\stepcounter{txtdec@thenextpage}%
		% if the current decoration occurs on this page, stay on it
		\iftikzmark{p\thepage.d\arabic{txtdec@decordraw}.\arabic{txtdec@hyphdraw}}{%
			\toggletrue{txtdec@stayonpage}}{}%
		\whileboolexpr{togl {txtdec@stayonpage}}{%
			\begin{tikzpicture}[remember picture, overlay]
			\setlength{\txtdec@height}{\csuse{decor@fsize@\thetxtdec@decordraw}}%
			\iftikzmark{p\thepage.d\arabic{txtdec@decordraw}.begin}{%
				% if current decor begins in current page, get coordinates
				\gettikzxy{(pic cs:p\thepage.d\arabic{txtdec@decordraw}.begin)}{\Xbegin}{\Ybegin}}{%
				% if current decor begins in previous page, set to top left of the page
				\gettikzxy{(current page.north west)}{\Xbegin}{\Ybegin}}%
			\iftikzmark{p\thepage.d\arabic{txtdec@decordraw}.end}{%
				% if current decor ends in current page, get coordinates
				\gettikzxy{(pic cs:p\thepage.d\arabic{txtdec@decordraw}.end)}{\Xend}{\Yend}}{%
				% if current decor ends in future page, set to bottom right of the page
				\gettikzxy{(current page.south east)}{\Xend}{\Yend}}%
			\ifdim\Ybegin=\Yend % the simplest case, a single line
			\txtdec@draw@all
			\stepcounter{txtdec@decordraw}%
			\else % current textdecor has a line break          
			\toggletrue{txtdec@stayondecor}%
			\whileboolexpr{togl {txtdec@stayondecor}}{%
				\gettikzxy{(pic cs:p\thepage.d\arabic{txtdec@decordraw}.\arabic{txtdec@hyphdraw})}{\Xlinebegin}{\Ylinebegin}%
				\edef\Xcurrent{\Xlinebegin}%
				\edef\Ycurrent{\Ylinebegin}%
				\edef\Xnext{\Xcurrent}%
				\edef\Ynext{\Ycurrent}%
				\toggletrue{txtdec@stayonline}%
				\whileboolexpr{togl {txtdec@stayonline}}{%
					\ifdim\Ycurrent=\Ynext
					\stepcounter{txtdec@hyphdraw}%
					% if the following tikzmark exists, we are at a page break
					\iftikzmark{p\arabic{txtdec@thenextpage}.d\arabic{txtdec@decordraw}.\arabic{txtdec@hyphdraw}}{%
						\edef\Xcurrent{\Xnext}%
						\edef\Ycurrent{\Ynext}%
						\gettikzxy{(current page.south east)}{\Xnext}{\Ynext}
						\togglefalse{txtdec@stayondecor}%
					}{% else, we remain on the same page
						\iftikzmark{p\thepage.d\arabic{txtdec@decordraw}.\arabic{txtdec@hyphdraw}}{%
							\edef\Xcurrent{\Xnext}%
							\edef\Ycurrent{\Ynext}%
							\gettikzxy{(pic cs:p\thepage.d\arabic{txtdec@decordraw}.\arabic{txtdec@hyphdraw})}{\Xnext}{\Ynext}}{}%
					}%
					\else
					\edef\Xlineend{\Xcurrent}%
					\edef\Ylineend{\Ycurrent}%
					% if we are on the first line of the current decoration
					\ifdim\Ylinebegin=\Ybegin
					\txtdec@draw@begin
					\else
					% if we are on the last line of the current decoration
					\ifdim\Ycurrent=\Yend
					\txtdec@draw@end
					\stepcounter{txtdec@decordraw}%
					\setcounter{txtdec@hyphdraw}{1}%
					\togglefalse{txtdec@stayondecor}%
					% if we are in a middle line of the decoration
					\else
					\txtdec@draw@middle
					\fi
					\fi
					\togglefalse{txtdec@stayonline}%  
					\fi
				}%
			}%
			\fi
			\end{tikzpicture}%
			% if the beginning of the next decor does not exist in this page, leave this page
			\iftikzmark{p\thepage.d\arabic{txtdec@decordraw}.begin}{}{%
				\togglefalse{txtdec@stayonpage}}%
			% if the continuation of the current decor exists in the next page, leave this page
			\iftikzmark{p\arabic{txtdec@thenextpage}.d\arabic{txtdec@decordraw}.\arabic{txtdec@hyphdraw}}{%
				\togglefalse{txtdec@stayonpage}}{}% 
		}%
	}%
}

\makeatother

%%%%% geschweifte Klammern von A nach B

%\newcommand{\tikzmark}[2][-3pt]{\tikz[remember picture, overlay, baseline=-0.5ex]\node[#1](#2){};}
%
%\tikzset{brace/.style={decorate, decoration={brace}},
%	brace mirrored/.style={decorate, decoration={brace,mirror}},
%}
%
%\newcounter{brace}
%\setcounter{brace}{0}
%\newcommand{\drawbrace}[3][brace]{%
%	\refstepcounter{brace}
%	\tikz[remember picture, overlay]\draw[#1] (#2.center)--(#3.center)node[pos=0.5, name=brace-\thebrace]{};
%}
%
%\newcounter{arrow}
%\setcounter{arrow}{0}
%\newcommand{\drawcurvedarrow}[3][]{%
%	\refstepcounter{arrow}
%	\tikz[remember picture, overlay]\draw (#2.center)edge[#1]node[coordinate,pos=0.5, name=arrow-\thearrow]{}(#3.center);
%}
%
%% #1 options, #2 position, #3 text 
%\newcommand{\annote}[3][]{%
%	\tikz[remember picture, overlay]\node[#1] at (#2) {#3};
%}

\renewcommand{\tikzmark}[2][-3pt]{\tikz[remember picture, overlay, baseline=-0.5ex]\node[#1](#2){};}

\tikzset{brace/.style={decorate, decoration={brace}},
	brace mirrored/.style={decorate, decoration={brace,mirror}},
}

\newcounter{brace}
\setcounter{brace}{0}
\newcommand{\drawbrace}[3][brace]{%
	\refstepcounter{brace}
	\tikz[remember picture, overlay]\draw[#1] (#2.center)--(#3.center)node[pos=0.5, name=brace-\thebrace]{};
}

\newcounter{arrow}
\setcounter{arrow}{0}
\newcommand{\drawcurvedarrow}[3][]{%
	\refstepcounter{arrow}
	\tikz[remember picture, overlay]\draw (#2.center)edge[#1]node[coordinate,pos=0.5, name=arrow-\thearrow]{}(#3.center);
}

\newcommand{\annote}[3][]{%
	\tikz[remember picture, overlay]\node[#1] at (#2) {#3};
}


%%%%% geschweifte Klammern von A nach B

\newcommand*\mycirc[1]{%
	\begin{tikzpicture}[baseline=(C.base)]
	\node[draw,circle,inner sep=1pt](C) {#1};
	\end{tikzpicture}}

%%%%% Lüchentext \fib nach 
%%%%% https://tobiw.de/tbdm/lueckentexte

\makeatletter

	\newlength\fib@width
	
	\newcommand\fib@widthfactor{1.34}
	
	\newlength\shortfib@width
	
	\newcommand\shortfib@widthfactor{1} % war 0.7
	
	\newif\iffibhideanswer
	\fibhideanswertrue
	
	\tikzset{
		every fill in box/.style={
			inner xsep=5pt,
			minimum height=4ex,
			align=center, %% war "center"
			font={\sffamily\slshape\bfseries},
			color=magenta,
			fill=black!5!white%dazu gedichtet
		},
		colored box/.style={
			every fill in box,
			fill=yellow!50!white,
		},
		framed box/.style={
			every fill in box,
			draw,
		},
		underline style/.style={},
		underlined box/.style={
			every fill in box,
			append after command={%
				\pgfextra{\begin{pgfinterruptpath}
						\draw [underline style,dashed,color=gray] (\tikzlastnode.south west)
						-- (\tikzlastnode.south east);
				\end{pgfinterruptpath}}
			},
		},
		bracket style/.style={},
		underbracked box/.style={
			every fill in box,
			append after command={%
				\pgfextra{\begin{pgfinterruptpath}
						\draw [bracket style]
						($(\tikzlastnode.south west)+(0,2pt)$)
						|- (\tikzlastnode.south)
						-| ($(\tikzlastnode.south east)+(0,2pt)$);
				\end{pgfinterruptpath}}
			},
		},
		%%%
		bracket style/.style={},
		simple box/.style={
			every fill in box,
			append after command={
				\pgfextra{
					%			\begin{pgfinterruptpath}
					%				\draw [bracket style]
					%				($(\tikzlastnode.south west)+(0,2pt)$)
					%				|- (\tikzlastnode.south)
					%				-| ($(\tikzlastnode.south east)+(0,2pt)$);
					%			\end{pgfinterruptpath}
				}
			},
		},
		%%%
		% von mir dazu gedichtet, um z. B. in Merkkästen Lingsbündig zu schreiben
		bracket style/.style={},
		underleft box/.style={
			every fill in box,
			align=left,
			append after command={%
				\pgfextra{\begin{pgfinterruptpath}
						\draw [bracket style]
						($(\tikzlastnode.south west)+(0,2pt)$)
						|- (\tikzlastnode.south)
						-| ($(\tikzlastnode.south east)+(0,2pt)$);
				\end{pgfinterruptpath}}
			},
		},
		%%%%%%%%%%%%%%%%%%%%%%%%%%%%%%%%%%%%%%%%%%%%%%%%%%%%%%%%%%%%%%%%%%%%%%%%%
		underoverbracked box/.style={
			every fill in box,
			append after command={%
				\pgfextra{\begin{pgfinterruptpath}
						\draw [bracket style]
						($(\tikzlastnode.north west)-(0,2pt)$)
						|- (\tikzlastnode.north)
						-| ($(\tikzlastnode.north east)-(0,2pt)$);
						\draw [bracket style]
						($(\tikzlastnode.south west)+(0,2pt)$)
						|- (\tikzlastnode.south)
						-| ($(\tikzlastnode.south east)+(0,2pt)$);
				\end{pgfinterruptpath}}
			},
		},
		fill in/.style={
			colored box,
		},
	}
	\tikzset{
		every fill in box/.style={
			inner xsep=4pt,			% wie groß ist der x-Abstand der Box vom Text
			inner ysep=2pt,			% wie groß ist der y-Abstand der Box vom Text
			minimum height=4ex,
			align=center, %% war "center"
			font={\sffamily\bfseries}, % war: 		font={\sffamily\slshape\bfseries},
			color=magenta,
			color=blue,
			fill=black!5!white%dazu gedichtet
		},
		%%%
		bracket style/.style={},
		simple box/.style={
			every fill in box,
			append after command={
				\pgfextra{
					%			\begin{pgfinterruptpath}
					%				\draw [bracket style]
					%				($(\tikzlastnode.south west)+(0,2pt)$)
					%				|- (\tikzlastnode.south)
					%				-| ($(\tikzlastnode.south east)+(0,2pt)$);
					%			\end{pgfinterruptpath}
				}%
			},
		},
		%%%
		fill in/.style={
			colored box,
		},
	}
		
	\NewDocumentCommand { \fib@hide } { m } {%
		\iffibhideanswer
		\phantom{#1}%
		\else
			{\boldmath\bfseries#1}%
		\fi
	}
	
	\NewDocumentCommand { \shortfib@hide } { m } {%
		\iffibhideanswer
		\phantom{#1}%
		\else
			{\boldmath\bfseries#1}%
		\fi
	}
	
	\NewDocumentCommand { \fib@makebox }{ m }{%
		\settowidth{\fib@width}{\tikz\node[fill in]{#1};}%
		\begin{tikzpicture}[baseline=(fill in node.base)]
		\node (fill in node)
		[text width=\fib@widthfactor*\fib@width,fill in]
		{%
			\fib@hide{#1}%
		};
		\end{tikzpicture}%
	}
	
	\NewDocumentCommand { \shortfib@makebox }{ m }{%
		\settowidth{\shortfib@width}{\tikz\node[fill in]{#1};}%
		\begin{tikzpicture}[baseline=(fill in node.base)]
		\node (fill in node)
		[text width=\shortfib@widthfactor*\shortfib@width,fill in]
		{%
			\shortfib@hide{#1}%
		};
		\end{tikzpicture}%
	}
	
	\NewDocumentCommand { \fib } { s o m }{{%
			\IfBooleanT{#1}{\fibhideanswerfalse}%
			\IfValueT{#2}{\tikzset{fill in/.style={#2}}}%
			\ifmmode
				\mathchoice
					{\fib@makebox{$\displaystyle#3$}}
					{\fib@makebox{$\textstyle#3$}}
					{\fib@makebox{$\scriptstyle#3$}}
					{\fib@makebox{$\scriptscriptstyle#3$}}
			\else
				\fib@makebox{{\boldmath\bfseries#3}}% <---- Dieses Kommentarzeichen war für die Lücken hinter den fibs verantwortlich!!!!
				%https://tex.stackexchange.com/questions/288561/how-to-boldify-a-part-of-text-with-maths
				%https://tex.stackexchange.com/questions/253105/question-about-bolding-math-and-text-simultaneously
			\fi
	}}
	
	\NewDocumentCommand { \shortfib } { s o m }{{%
			\IfBooleanT{#1}{\fibhideanswerfalse}%
			\IfValueT{#2}{\tikzset{fill in/.style={#2}}}%
			\ifmmode
				\mathchoice
					{\shortfib@makebox{$\displaystyle#3$}}
					{\shortfib@makebox{$\textstyle#3$}}
					{\shortfib@makebox{$\scriptstyle#3$}}
					{\shortfib@makebox{$\scriptscriptstyle#3$}}
			\else
				\shortfib@makebox{{\boldmath\bfseries#3}}% <---- Dieses Kommentarzeichen war für die Lücken hinter den fibs verantwortlich!!!!
				%https://tex.stackexchange.com/questions/288561/how-to-boldify-a-part-of-text-with-maths
				%https://tex.stackexchange.com/questions/253105/question-about-bolding-math-and-text-simultaneously
			\fi
	}}

\makeatother

%%%%%%%%%%%%%%%
%oberhalb gibt es wohl den Grund für die Lücke nach einer fib-Lücke!
%%%%%%%%%%%%%%%

% Definition des Standards:

%\tikzset{fill in/.style={colored box}}% = Voreinstellung
%\tikzset{fill in/.style={framed box}}
%\tikzset{fill in/.style={underlined box}}
%\tikzset{fill in/.style={underbracked box}}
\tikzset{fill in/.style={underleft box}}
%\tikzset{fill in/.style={underoverbracked box}}

\tikzset{colored box/.append style={fill=black!5!white}}
\tikzset{underline style/.style={dotted, thick}}
\tikzset{bracket style/.style={gray, thick}}

%%%%%%%%%%%%%%%%%%%%%%%%%%%%%%%%%%%
%https://tobiw.de/tbdm/lueckentexte
%%%%%%%%%%%%%%%%%%%%%%%%%%%%%%%%%%%

% Einfärben eines Ringes:

\tikzset{
	ring shading/.code args={from #1 at #2 to #3 at #4}{
		\pgfmathsetmacro{\inner}{25*#2/#4}
		\pgfmathsetmacro{\innerlow}{\inner-1}
		\pgfdeclareradialshading{ring}{\pgfpoint{0cm}{0cm}}%
		{
			color(0bp)=(white);
			color(\innerlow bp)=(white);
			color(\inner bp)=(#1);
			color(25bp)=(#3);
			color(50bp)=(black)
		}
		\pgfkeysalso{/tikz/shading=ring}
	},
}


%%% Für Baumdiagramme
%\tikzset{
%	basic/.style  = 	{draw, font=\sffamily, rectangle},
%	root/.style   = 	{basic, text width=4cm, rounded corners=2pt, thick, align=center, fill=white!30, minimum height=1cm},
%	level 1/.style = 	{basic, text width=6cm, rounded corners=2pt, thick, align=center, fill=white!60, minimum height=1.5cm, sibling distance = 7cm},
%	level 2/.style = 	{basic, text width=5cm, rounded corners=2pt, thick, align=center, fill=white!60, minimum height=1.5cm, sibling distance = 6cm},
%}

\makeatletter
\newif\if@showgrid@grid
\newif\if@showgrid@left
\newif\if@showgrid@right
\newif\if@showgrid@below
\newif\if@showgrid@above
\tikzset{%
	every show grid/.style={},
	show grid/.style={execute at end picture={\@showgrid{grid=true,#1}}},%
	show grid/.default={true},
	show grid/.cd,
	labels/.style={font={\sffamily\small},help lines},
	xlabels/.style={},
	ylabels/.style={},
	keep bb/.code={\useasboundingbox (current bounding box.south west) rectangle (current bounding box.north west);},
	true/.style={left,below},
	false/.style={left=false,right=false,above=false,below=false,grid=false},
	none/.style={left=false,right=false,above=false,below=false},
	all/.style={left=true,right=true,above=true,below=true},
	grid/.is if=@showgrid@grid,
	left/.is if=@showgrid@left,
	right/.is if=@showgrid@right,
	below/.is if=@showgrid@below,
	above/.is if=@showgrid@above,
	false,
}

\def\@showgrid#1{%
	\begin{scope}[every show grid,show grid/.cd,#1]
		\if@showgrid@grid
		\begin{pgfonlayer}{background}
			\draw [help lines]
			(current bounding box.south west) grid
			(current bounding box.north east);
			%
			\pgfpointxy{1}{1}%
			\edef\xs{\the\pgf@x}%
			\edef\ys{\the\pgf@y}%
			\pgfpointanchor{current bounding box}{south west}
			\edef\xa{\the\pgf@x}%
			\edef\ya{\the\pgf@y}%
			\pgfpointanchor{current bounding box}{north east}
			\edef\xb{\the\pgf@x}%
			\edef\yb{\the\pgf@y}%
			\pgfmathtruncatemacro\xbeg{ceil(\xa/\xs)}
			\pgfmathtruncatemacro\xend{floor(\xb/\xs)}
			\if@showgrid@below
			\foreach \X in {\xbeg,...,\xend} {
				\node [below,show grid/labels,show grid/xlabels] at (\X,\ya) {\X};
			}
			\fi
			\if@showgrid@above
			\foreach \X in {\xbeg,...,\xend} {
				\node [above,show grid/labels,show grid/xlabels] at (\X,\yb) {\X};
			}
			\fi
			\pgfmathtruncatemacro\ybeg{ceil(\ya/\ys)}
			\pgfmathtruncatemacro\yend{floor(\yb/\ys)}
			\if@showgrid@left
			\foreach \Y in {\ybeg,...,\yend} {
				\node [left,show grid/labels,show grid/ylabels] at (\xa,\Y) {\Y};
			}
			\fi
			\if@showgrid@right
			\foreach \Y in {\ybeg,...,\yend} {
				\node [right,show grid/labels,show grid/ylabels] at (\xb,\Y) {\Y};
			}
			\fi
		\end{pgfonlayer}
		\fi
	\end{scope}
}
\makeatother

\usepackage{eqparbox}